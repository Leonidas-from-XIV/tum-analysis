\lecture{2010-01-13}

% In der Vorlesung wurde einiges aus der letzten Vorlesung wiederholt. 
% Eventuell sollten wir Sachen die wirklich 1:1 doppelt vorkommen aus diesem Mitschrieb streichen

\subsubsection*{Erinnerung}
\todo[inline]{redundant -- in entsprechende Textstellen einarbeiten} 
\begin{itemize}
	\item Koch'sche Schneeflockenkurve
	\item Reihe \( \sum_{k=0}^{\infty} a_k \) als Folge der Partialsummen \( (s_n)_{n \in \mathbb{N} } \quad s_n := \sum_{k=0}^{\infty}a_k \)
	\item Folgentheorie: Konvergenz, Divergenz
	\item Zentral: Geometrische Reihe \[
		|x|<1 \qquad \sum_{k=1}^{\infty} x^{k-1} = \frac{1}{-x+1} \qquad \left(\frac{1-x^n}{1-x}\right)^{\infty\rightarrow n}
	\]
	\item Notwendig für Reihenkonvergenz: Folge \( (a_k)_{k \in \mathbb{N}} \) ist Nullfolge
	\item Leider ist dies nicht hinreichend, denn \( \sum_{k=1}^{\infty} \frac{1}{k} \) divergiert ( \(\sum_{k=1}^{\infty}\frac{1}{k^{1+\delta}} \quad   \delta > 0 \) konvergiert) und \( a_k = \frac{1}{k} \) ist Nullfolge
	\item Rechenregeln: \( \pm, \cdot, \) 
	\todo[inline]{Dritte Operation fehlt. Konnte ich nicht entziffern.}
\end{itemize}

\subsection*{Konvergenzkriterien}
\subsubsection*{Idee: Majoranten/Minorantenkriterium}
	
\begin{align*}
	& 0 \leq a_n < b_n & \\
	& \sum_{k=0}^{\infty}b_k \quad\text{konvergiert} &\implies \sum_{k=0}^{\infty} a_k \quad\text{konvergiert} \qquad \text{"`Majorante"'} \\
	& \sum_{k=0}^{\infty}a_k \quad\text{divergiert} &\implies \sum_{k=0}^{\infty} b_k \quadÅ\text{konvergiert} \qquad \text{"`Minorante"'}
\end{align*}

\minisec{Fazit}
Versucht man Konvergenz zu zeigen, konstruiert man eine bekannte Majorante.\newline
Versucht man Divergenz zu zeigen, konstruiert man eine bekannte Minorante. \newline

\noindent Einziges Werkzeug: Geometrische Reihe

\subsubsection*{Idee: vergleiche $ \sum a_k $ mit $x^k$}
Ab festem \( k_0 \):
\begin{align*}
	&0 \leq a_k \leq x^k \qquad k \geq k_0 \\
	\implies &0 \leq \sqrt[k]{a_k} \leq x
\end{align*}

\noindent Allgemein:
\begin{align*}
	&0 \leq \lim_{k\rightarrow\infty} \sqrt[k]{a_k} \leq x \\
	&0 < x < 1 \implies \text{Konvergenz}
\end{align*}

\subsubsection*{Idee: bilde Quotienten} % (fold)
\label{sub:bilde_quotienten}

\[
	\frac{a_{n+1}}{a_n} \leq x < 1 \implies a_{n+1} \leq a_n x
\]
Rekursiv:

\[
	a_{n+1} \leq a_nx \leq a_{n-1}x^2\leq \ldots \leq a_1x^{n+1} \implies a_n \leq a_1x^n
\]

\begin{theorem}[Wurzelkriterium]
		Ist eine Folge \( (a_n)_{n \in \mathbb{N}} \) gegeben mit \( 0 \leq a_n \) und existiert \( r = \lim_{n\rightarrow\infty} \sqrt[n]{a_n} \), so gilt
		\begin{enumerate}
			\item \( r < 1 \implies \quad \text{Konvergenz} \)
			\item \( r > 1 \implies \quad \text{Divergenz} \)
			\item \( r = 1 \implies \quad \text{Nicht entscheidbar} \)
		\end{enumerate}
\end{theorem}

\begin{theorem}[Quotientenkriterium]
	Ist eine Folge \( (a_n)_{n \in \mathbb{N} } \) gegeben mit \( 0 \leq a_n \) und existiert \( s = \lim_{n\rightarrow\infty} \frac{a_{n+1}}{a_n} \), so gilt
	\begin{enumerate}
		\item \( s < 1 \implies \quad \text{Konvergenz} \)
		\item \( s > 1 \implies \quad \text{Divergenz} \)
		\item \( s = 1 \implies \quad \text{Nicht entscheidbar} \)
	\end{enumerate}
\end{theorem}
% subsection 3_idee (end)

\subsubsection{Idee: alternierende Reihen} % (fold)
\label{sub:idee_alternierende_reihen}

\begin{definition}[Alternierende Reihe]
	Eine Reihe, deren Elemente wechselnde Vorzeichen haben, heißt alternierende Reihe.
	\[
		(a_n)_{n \in \mathbb{N} } \quad a_n \geq 0 \quad \sum_{k=1}^\infty (-1)^{k-1}a_k
	\]
\end{definition}

\begin{theorem}[Leibnitz-Kriterium]
	Bilden \( (a_n)_{n\in \mathbb{N}} \) eine monoton fallende Nullfolge (\( a_n>a_{n+1} \)), so ist die alternierende Reihe konvergent.
\end{theorem}

% Das hier ist eigentlich nur eine "Beweis-Idee", kein formaler Beweis
\begin{proof}
	Betrachte Partialsummen mit geraden bzw. ungeraden Indizes.
	\[
		S_{2n+2} = S_{2n} + \underbrace{a_{2n+1}-a_{2n+2}}_{>0} > S_{2n} \quad \text{"`Monoton wachsend"'}
	\]
	
	\[
		S_{2n+1} = S_{2n-1} + \underbrace{-a_{2n}+a_{2n+1}}_{<0} < S_{2n-1} \quad \text{"`Monoton fallend"'}
	\]
	
	\noindent \( S_{2n} \) ist oben durch \( S_1 \) beschränkt. \newline
	\noindent \( S_{2n+1} \) ist unten durch \( S_2 \) beschränkt. \newline\newline
	\noindent Grenzwert = Konvergenz \( S_{n+2}\rightarrow S_{n+1} \)
\end{proof}

\subsubsection{Bemerkung zu alternierenden Reihen (Datenverarbeitung)} % (fold)
\label{ssub:bemerkung_zu_alternierenden_reihen_datenverarbeitung_}
Es gelte \[ s=\sum_{k=1}^\infty (-1)^{k-1}a_k \] 

\noindent In der Praxis relevant: \( 0<(-1)^n(S-S_n) < a_{n+1} \). D.h. falls ich die Reihe nach \( n \) Termen abbreche, ist der Fehler stets kleiner als der letzte Reihenterm. Bis ca. 1985 war dies u.A. für die Implementierung von sin(x) wichtig.
\[
	x \in \mathbb{M} \quad x \rightarrow 2\pi \rightarrow \frac{\pi}{2} \rightarrow \ldots
\]

% subsubsection bemerkung_zu_alternierenden_reihen_datenverarbeitung_ (end)
% subsection 4_idee_alternierende_reihen (end)

\subsection{Absolut konvergente Reihen} % (fold)
\label{sub:absolut_konvergente_reihen}

Bisherige Notation war \( a_n \geq 0 \). Erweiterung auf beliebige reelle bzw. komplexe \( a_n \). 

\begin{definition}[Absolut konvergente Reihe]
	Sei \( (a_n)_{n\in\mathbb{N}} \) eine beliebige Zahlenfolge in \( \mathbb{R} \) bzw. in \( \mathbb{C} \), dann heisst die zugehörige Reihe \( \sum_{k=0}^\infty a_k \) absolut konvergent, falls \( \sum_{k=0}^\infty |a_k| \) konvergiert.
\end{definition}

\noindent Bemerkung: Definition ist sehr krass.
\[
	\sum_{k=1}^\infty(-1)^{k-1}\frac{1}{k} \quad \text{konvergent, aber nicht absolut konvergent}
\]
Allgemein:
\[
	\sum_{k=1}^\infty a_k\leq|\sum a_k| \leq \sum|a_k|
\]
\[
	\sum_{k=1}^\infty a_k\geq-|\sum a_k| \geq -\sum|a_k|
\]
\noindent Jede absolut konvergente Reihe ist implizit eine konvergente Reihe! (nicht umgekehrt!)

\begin{example}
	\[
		\sum_{k=1}^\infty \frac{1}{k^2}z^k \qquad |z|\leq1
	\]
	Versuche Konvergenz über absolute Konvergenz und Majorantenkriterium nachzuweisen.
	\[
		0 \leq \left|\frac{z^k}{k^2}\right|=\frac{|z^k|}{k^2}\leq\frac{1}{k^2}\qquad \text{da} |z|\leq1
	\]
	Frage: Ist \( \sum\frac{1}{k^2} \) eine Majorante?
	
	\begin{enumerate}
		\item Versuch: Quotientenkriterium 
		\[
		   s=\lim_{k\rightarrow\infty}\frac{a_{k+1}}{a_k}
			=\lim_{k\rightarrow\infty}\frac{\frac{1}{(k+1)^2}}{\frac{1}{k^2}}
			=\lim_{k\rightarrow\infty}\frac{n^2}{(k+1)^2}
			=\lim_{k\rightarrow\infty}\left(1-\frac{1}{(k+1)}\right)^2 
			= 1 
		\]
		\[
			s = 1 \implies \text{nicht entscheidbar} 
		\]
		\item Versuch: Wurzelkriterium
		\[
		   r=\lim_{k\rightarrow\infty}\sqrt[k]{\frac{1}{k^2}}
			=\lim_{k\rightarrow\infty}\frac{1}{(\sqrt[k]{k})^2}
			= 1 \implies  \text{nicht entscheidbar}
		\]
		\item Versuch: Finde eine weitere Majorante zur möglichen Majorante \( \sum\frac{1}{k^2} \)
		\[
			\frac{1}{k^2}= \frac{1}{k\cdot k}\leq \frac{1}{k-1}\cdot\frac{1}{k} \qquad \text{Bemerkung:}\frac{1}{k} \text{ divergiert (Minnorant)}
		\]
		Ist \( \sum\frac{1}{k(k-1)} \) konvergent? Umformung:
		\(
			\frac{1}{k(k-1)}=\frac{1}{k-1}-\frac{1}{k}
		\)
		\[
			\sum_{k=2}^\infty\frac{1}{(k-1)k}=\sum_{k=2}^\infty(\frac{1}{k-1}-\frac{1}{k}) \qquad \text{"`Teleskopsumme"'}
		\]
		Bilde Partialsummen
		\[
			S_n=\sum_{k=2}^n(\frac{1}{k-1}-\frac{1}{k})=(1-{\color{red}\frac{1}{2}})+({\color{red}\frac{1}{2}}-{\color{blue}\frac{1}{3}})+({\color{blue}\frac{1}{3}}-\frac{1}{4})+\ldots+(\frac{1}{n-1}-\frac{1}{n})=1-\frac{1}{n}
		\]
	\end{enumerate}
	Ergebnis: Folge der Partialsummen \( S_n = 1- \frac{1}{n}\) ist konvergent: \( lim_{n\rightarrow\infty} S_n=1 \implies \sum\frac{1}{k(k-1)} \) ist konvergenter Majorant zu \( \sum\frac{1}{k^2} \).
\end{example}
% subsection absolut_konvergente_reihen (end)




