\lecture{2010-01-13}

% In der Vorlesung wurde einiges aus der letzten Vorlesung wiederholt. 
% Eventuell sollten wir Sachen die wirklich 1:1 doppelt vorkommen aus diesem Mitschrieb streichen

\subsubsection*{Erinnerung}
\todo[inline]{redundant -- in entsprechende Textstellen einarbeiten}
\begin{itemize}
	\item Koch'sche Schneeflockenkurve
	\item Reihe \( \sum_{k=0}^{\infty} a_k \) als Folge der Partialsummen \( (s_n)_{n \in \mathbb{N} } \quad s_n := \sum_{k=0}^{\infty}a_k \)
	\item Folgentheorie: Konvergenz, Divergenz
	\item Zentral: Geometrische Reihe \[
		|x|<1 \qquad \sum_{k=1}^{\infty} x^{k-1} = \frac{1}{-x+1} \qquad \left(\frac{1-x^n}{1-x}\right)^{\infty\rightarrow n}
	\]
	\item Notwendig für Reihenkonvergenz: Folge \( (a_k)_{k \in \mathbb{N}} \) ist Nullfolge
	\item Leider ist dies nicht hinreichend, denn \( \sum_{k=1}^{\infty} \frac{1}{k} \) divergiert ( \(\sum_{k=1}^{\infty}\frac{1}{k^{1+\delta}} \quad   \delta > 0 \) konvergiert) und \( a_k = \frac{1}{k} \) ist Nullfolge
	\item Rechenregeln: \( \pm, \cdot, \) 
	\todo{Dritte Operation fehlt. Konnte ich nicht entziffern.}
\end{itemize}

\subsection*{Konvergenzkriterien}
\subsubsection*{1. Idee: Majoranten-/Minorantenkriterium}
	
\begin{align*}
	& 0 \leq a_n < b_n & \\
	& \sum_{k=0}^{\infty}b_k \quad\text{konvergiert} &\implies \sum_{k=0}^{\infty} a_k \quad\text{konvergiert} \qquad \text{"`Majorante"'} \\
	& \sum_{k=0}^{\infty}a_k \quad\text{divergiert} &\implies \sum_{k=0}^{\infty} b_k \quad\text{konvergiert} \qquad \text{"`Minorante"'}
\end{align*}

\minisec{Fazit}
Versucht man Konvergenz zu zeigen, konstruiert man eine bekannte Majorante.\newline
Versucht man Divergenz zu zeigen, konstruiert man eine bekannte Minorante. \newline

\noindent Einziges Werkzeug: Geometrische Reihe

\subsubsection*{2. Idee: Vergleiche $ \sum a_k $ mit $x^k$}
Ab festem \( k_0 \):
\begin{align*}
	&0 \leq a_k \leq x^k \qquad k \geq k_0 \\
	\implies &0 \leq \sqrt[k]{a_k} \leq x
\end{align*}

\noindent Allgemein:
\begin{align*}
	&0 \leq \lim_{k\rightarrow\infty} \sqrt[k]{a_k} \leq x \\
	&0 < x < 1 \implies \text{Konvergenz}
\end{align*}

\subsubsection*{3. Idee: Bilde Quotienten} % (fold)
\label{sub:3_idee}

\[
	\frac{a_{n+1}}{a_n} \leq x < 1 \implies a_{n+1} \leq a_n x
\]

% subsection 3_idee (end)