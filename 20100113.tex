\lecture{2010-01-13}

% In der Vorlesung wurde einiges aus der letzten Vorlesung wiederholt. 
% Eventuell sollten wir Sachen die wirklich 1:1 doppelt vorkommen aus diesem Mitschrieb streichen
% eingegliedert --LH

\subsection{Konvergenzkriterien}
\subsubsection*{Idee: Majoranten-/Minorantenkriterium}
	
$0 \leq a_n < b_n$
\[
\begin{array}{lcll}
	\ds\sum_{k=0}^{\infty}b_k\quad\text{konvergiert}&\implies&\ds\sum_{k=0}^{\infty} a_k \quad\text{konvergiert} & \text{"`Majorante"'} \\
	\ds\sum_{k=0}^{\infty}a_k \quad\text{divergiert}&\implies&\ds\sum_{k=0}^{\infty} b_k \quad\text{divergiert} & \text{"`Minorante"'}
\end{array}
\]

\begin{example}[Konvergenz]
    Die Reihe \[ \sum_{k=1}^{\infty} \frac{\sin^2 \left( k^3 + 5 \right) }{3^k + 1} \] ist konvergent, da
    \begin{align*}
        \sin^2 \left(k^3 + 5 \right) &\leq 1 \\
        \frac{1}{3^k+1} &\leq \frac{1}{3^k} \\
        \text{folglich: } \sum_{k=1}^{\infty} \frac{\sin^2 \left( k^3 + 5 \right) }{3^k + 1} &\leq \sum_{k=1}^{\infty} \left( \frac{1}{3} \right)^k
    \end{align*}
    und $\sum_{k=1}^{\infty} \left( \frac{1}{3} \right)^k$ konvergent ist.
\end{example}

\begin{example}[Divergenz]
    \begin{equation*}
      \sum_{k=1}^{\infty} \frac{1}{\sqrt{k}}
    \end{equation*}
    divergent, da
    \begin{equation*}
      0 \leq \frac{1}{k} \leq \frac{1}{\sqrt{k}}
    \end{equation*}
    $\implies$ harmonische Reihe $\sum_{k=1}^{\infty} \frac{1}{k}$ ist Minorante und divergent
\end{example}

\minisec{Fazit}
Versucht man Konvergenz zu zeigen, konstruiert man eine bekannte Majorante.\newline
Versucht man Divergenz zu zeigen, konstruiert man eine bekannte Minorante. \newline

\noindent einziges Werkzeug: Geometrische Reihe

\subsubsection*{Idee: vergleiche $ \sum a_k $ mit $x^k$}
ab festem \( k_0 \):
\begin{align*}
	&0 \leq a_k \leq x^k \qquad k \geq k_0 \\
	\implies &0 \leq \sqrt[k]{a_k} \leq x
\end{align*}

\noindent allgemein:
\begin{align*}
	&0 \leq \lim_{k\rightarrow\infty} \sqrt[k]{a_k} \leq x \\
	&0 < x < 1 \implies \text{Konvergenz}
\end{align*}

\begin{theorem}[Wurzelkriterium]
\label{theorem:Wurzelkriterium}
		Ist eine Folge \( (a_n)_{n \in \mathbb{N}} \) gegeben mit \( 0 \leq a_n \) und existiert \( r = \lim_{n\rightarrow\infty} \sqrt[n]{a_n} \), so gilt
		\begin{enumerate}
			\item \( r < 1 \implies \; \text{Konvergenz} \)
			\item \( r > 1 \implies \; \text{Divergenz} \)
			\item \( r = 1 \implies \; \text{nicht entscheidbar} \)
		\end{enumerate}
\end{theorem}

\subsubsection*{Idee: bilde Quotienten} % (fold)
\label{sub:bilde_quotienten}

\[
	\frac{a_{n+1}}{a_n} \leq x < 1 \implies a_{n+1} \leq a_n x
\]
rekursiv:

\[
	a_{n+1} \leq a_nx \leq a_{n-1}x^2\leq \ldots \leq a_1x^{n+1} \implies a_n \leq a_1x^n
\]

\begin{theorem}[Quotientenkriterium]
	Ist eine Folge \( (a_n)_{n \in \mathbb{N} } \) gegeben mit \( 0 \leq a_n \) und existiert \( s = \lim_{n\rightarrow\infty} \frac{a_{n+1}}{a_n} \), so gilt
	\begin{enumerate}
		\item \( s < 1 \implies \; \text{Konvergenz} \)
		\item \( s > 1 \implies \; \text{Divergenz} \)
		\item \( s = 1 \implies \; \text{nicht entscheidbar} \)
	\end{enumerate}
\end{theorem}
% subsection 3_idee (end)

\subsubsection{Idee: alternierende Reihen} % (fold)
\label{sub:idee_alternierende_reihen}

\begin{definition}[Alternierende Reihe]
	Eine Reihe, deren Elemente wechselnde Vorzeichen haben, heißt alternierende Reihe.
	\[
		(a_n)_{n \in \mathbb{N} } \quad a_n \geq 0 \quad \sum_{k=1}^\infty (-1)^{k-1}a_k
	\]
\end{definition}

\begin{theorem}[Leibniz-Kriterium]
	Bilden \( (a_n)_{n\in \mathbb{N}} \) eine monoton fallende Nullfolge (\( a_n>a_{n+1} \)), so ist die alternierende Reihe konvergent.
\end{theorem}

% Das hier ist eigentlich nur eine "Beweis-Idee", kein formaler Beweis
\begin{proof}[Beweisidee]
	Betrachte Partialsummen mit geraden bzw. ungeraden Indizes:
	\[
		S_{2n+2} = S_{2n} + \underbrace{a_{2n+1}-a_{2n+2}}_{>0} > S_{2n} \quad \text{monoton wachsend}
	\]
	
	\[
		S_{2n+1} = S_{2n-1} + \underbrace{-a_{2n}+a_{2n+1}}_{<0} < S_{2n-1} \quad \text{monoton fallend}
	\]
	
	\noindent \( S_{2n} \) ist oben durch \( S_1 \) beschränkt\newline
	\noindent \( S_{2n+1} \) ist unten durch \( S_2 \) beschränkt \newline\newline
	\noindent Grenzwert = Konvergenz \( S_{n+2}\rightarrow S_{n+1} \)
\end{proof}

\begin{note}[Alternierende Reihen in der Datenverarbeitung]

Es gelte \[ s=\sum_{k=1}^\infty (-1)^{k-1}a_k \] 

\noindent In der Praxis relevant: \( 0<(-1)^n(S-S_n) < a_{n+1} \). D. h. falls die Berechnung der Reihe nach \( n \) Termen abgebrochen wird, ist der Fehler stets kleiner als der letzte Reihenterm. Bis ca. 1985 war dies u. A. für die Implementierung von $\sin(x)$ wichtig.
\[
	x \in \mathbb{M} \quad x \rightarrow 2\pi \rightarrow \frac{\pi}{2} \rightarrow \ldots
\]
\end{note}
% subsection idee_alternierende_reihen (end)

\subsection{Absolut konvergente Reihen} % (fold)
\label{sub:absolut_konvergente_reihen}

bisherige Notation war \( a_n \geq 0 \), Erweiterung auf beliebige reelle bzw. komplexe \( a_n \)

\begin{definition}[Absolut konvergente Reihe]
	Sei \( (a_n)_{n\in\mathbb{N}} \) eine beliebige Zahlenfolge in \( \mathbb{R} \) bzw. in \( \mathbb{C} \), dann heißt die zugehörige Reihe \( \sum_{k=0}^\infty a_k \) absolut konvergent, falls \( \sum_{k=0}^\infty |a_k| \) konvergiert.
\end{definition}

\begin{note} Definition ist sehr rigide:
\[
	\sum_{k=1}^\infty(-1)^{k-1}\frac{1}{k}
\]
ist konvergent, aber nicht absolut konvergent.
\end{note}

\noindent allgemein:
\[
  \begin{array}{ccrcr}
    \ds\sum_{k=1}^\infty a_k &\leq& \ds\left|\sum a_k\right| &\leq& \ds\sum|a_k| \\
    \ds\sum_{k=1}^\infty a_k &\geq&\ds-\left|\sum a_k\right| &\geq& \ds-\sum|a_k|
  \end{array}
\]

\begin{note}
  Jede absolut konvergente Reihe ist implizit eine konvergente Reihe, aber nicht umgekehrt.
\end{note}

\begin{example}
	\[
		\sum_{k=1}^\infty \frac{1}{k^2}z^k \qquad |z|\leq1
	\]
	Versuch: Konvergenz über absolute Konvergenz und Majorantenkriterium nachweisen
	\[
		0 \leq \left|\frac{z^k}{k^2}\right|=\frac{|z^k|}{k^2}\leq\frac{1}{k^2}\qquad \text{da } |z|\leq1
	\]
	Frage: Ist \( \sum\frac{1}{k^2} \) eine Majorante?
	
	\begin{enumerate}
		\item Quotientenkriterium 
		\[
		   s=\lim_{k\rightarrow\infty}\frac{a_{k+1}}{a_k}
			=\lim_{k\rightarrow\infty}\frac{\frac{1}{(k+1)^2}}{\frac{1}{k^2}}
			=\lim_{k\rightarrow\infty}\frac{k^2}{(k+1)^2}
			=\lim_{k\rightarrow\infty}\left(1-\frac{1}{(k+1)}\right)^2 
			= 1 
		\]
		\[
			s = 1 \implies \text{nicht entscheidbar} 
		\]
		\item Wurzelkriterium
		\[
		   r=\lim_{k\rightarrow\infty}\sqrt[k]{\frac{1}{k^2}}
			=\lim_{k\rightarrow\infty}\frac{1}{(\sqrt[k]{k})^2}
			= 1 \implies  \text{nicht entscheidbar}
		\]
		\item finde eine weitere Majorante zur möglichen Majorante \( \sum\frac{1}{k^2} \)
		\[
			\frac{1}{k^2}= \frac{1}{k\cdot k}\leq \frac{1}{k-1}\cdot\frac{1}{k}
%\qquad \text{Bemerkung:}\frac{1}{k} \text{ divergiert (Minnorant)}
		\]
		Ist \( \sum\frac{1}{k(k-1)} \) konvergent? Umformung:
		\[
			\frac{1}{k(k-1)}=\frac{1}{k-1}-\frac{1}{k}
		\]
		\[
			\sum_{k=2}^\infty\frac{1}{(k-1)k}=\sum_{k=2}^\infty\left(\frac{1}{k-1}-\frac{1}{k}\right) \qquad \text{"`Teleskopsumme"'}
		\]
		bilde Partialsummen
		\begin{multline*}
			S_n=\sum_{k=2}^n\left(\frac{1}{k-1}-\frac{1}{k}\right)\\=\left(1-{\color{red}\frac{1}{2}}\right)+\left({\color{red}\frac{1}{2}}-{\color{blue}\frac{1}{3}}\right)+\left({\color{blue}\frac{1}{3}}-\frac{1}{4}\right)+\ldots+\left(\frac{1}{n-1}-\frac{1}{n}\right)=1-\frac{1}{n}
		\end{multline*}
	\end{enumerate}
	Ergebnis: Folge der Partialsummen \( S_n = 1- \frac{1}{n}\) ist konvergent: \( \lim_{n\rightarrow\infty} S_n=1 \implies \sum\frac{1}{k(k-1)} \) ist konvergenter Majorant zu \( \sum\frac{1}{k^2} \).
\end{example}
% subsection absolut_konvergente_reihen (end)
