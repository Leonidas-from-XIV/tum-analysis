\lecture{2009-11-24}
\section{Differenzierbarkeit}
\subsection{Ableitungsbegriff}

bisher: Stetige Funktion, die den Grenzprozess im Definitionsbereich auf den Wertebereich überträgt.\\
Frage: Ist es \emph{lokal} möglich eine Funktion \( f(x) \) genauer zu beschreiben? 

Wie sieht eine Funktion \( p(x) \) bzw. eine Gerade aus, die \( f(x) \) in \( x_0 \) möglichst gut beschreibt?
\[
p(x)=ax+b 
\]
\begin{center}
  \centering
\begin{tikzpicture}[scale=0.7, x=2cm, y=2cm]
	\draw[blue,domain=-2:2] plot ({\x},{(\x)^2}) node[above]{\( f(x)=x^2 \)};
	\draw[red,domain=-2:2.2] plot ({\x},{3}) node[anchor=mid west]{\( p_1 \)};
	\draw[red,domain=-2:0.5] plot ({\x},{(-\x)}) node[anchor=north west]{\( p_2 \)}; 
	\draw[green,domain=0:2.3] plot ({\x},{2*(\x)-1}) node[anchor=south west]{\( p \)};
\end{tikzpicture}
  \captionof{figure}{Die Geraden \( p_1(x) \) bzw. \( p_2(x) \) aus der Interpolation sind keine guten Repräsentanten. \(p(x)\) als Tangente in \(f(x_0)\) liegt hingegen gut.}
  \label{fig:Approximierung}
\end{center}
% LH: Polynominterpolation aus zwei Punkten -> Gerade
%\todo{JP:Welche Interpolation ist gemeint?}

\noindent Konstruktion der Geraden \( p(x) \), die die Funktion \( f(x) \) in der Nähe von \( x_0 \) gut beschreibt:

\[
	f:D\rightarrow\mathbb{R} \qquad D\subseteq\mathbb{R} \qquad x_0 \in D
\]

\begin{description}
	\item[1. Forderung] \( p(x_0)=f(x_0) \) 
	\begin{align*}
			p(x) &= ax+b \\
			f(x_0) &= ax_0+b \\
			\implies b &= f(x_0)-ax_0 \\
			\implies p(x) &= a(x-x_0)+f(x_0)
	\end{align*}
	\item[2. Forderung] \( d(x)=f(x)-p(x) \) soll in der Nähe von \( x_0 \) möglichst klein werden. Damit gilt \( d(x) = f(x) -f(x_0)-a(x-x_0) \)
	\item[1. Versuch] \( f(x) \) stetig  
	\[\lim_{x\rightarrow x_0} d(x)=\lim_{x\rightarrow x_0}(\underbrace{f(x)-f(x_0)}_{\rightarrow 0}-a\underbrace{(x-x_0)}_{\rightarrow 0})=0\]Dies ist nichts Neues, da die Information von \( a \) durch den Faktor \( (x-x_0) \) verdeckt wird.
	\item[2. Versuch] Forderung an Parameter \( a \in \mathbb{R} \)
	\[
		\lim_{x\rightarrow x_0}\left(\frac{f(x)-f(x_0)-a(x-x_0)}{x-x_0}\right)\stackrel{!}{=}0
	\]
	Die Division durch \( x-x_0 \) neutralisiert den Faktor \( a \).
\end{description}

\begin{definition}[Lineare Approximierbarkeit]
	\( f:D \to \mathbb{R} \) heißt linear approximierbar in \( x_0 \in D \), falls ein \( a \in \mathbb{R} \) existiert mit
        \[ \lim_{x\rightarrow x_0}\left(\frac{f(x)-f(x_0)-a(x-x_0)}{x-x_0}\right)=0 \]
\end{definition}

\begin{note}
  \begin{itemize}
    \item Die Herleitung ist auch auf \( \mathbb{R}^n \) übertragbar.
    \item \( f \) heißt auch \emph{differenzierbar} in \( x_0 \)
  \end{itemize}
\end{note}

\begin{example} 
\[
f(x)=x^2 \qquad x_0=1 \implies f(x_0)=1
\]
\begin{align*}
  \frac{f(x)-f(x_0)-a(x-x_0)}{x-x_0} &= \frac{x^2-1-a(x-1)}{x-1} \\
  &= \frac{(x-1)(x+1)-a(x-1)}{x-1} \\
  &= x+1-a
\intertext{nach Def. der linearen Approximierbarkeit -- Grenzwertbildung}
  0 &= \lim_{x \rightarrow x_0} \left(x+1-a\right) \\
  &= \left(x+1-a\right) \\
  \implies a &= 2
\end{align*}
\end{example}
	
\begin{definition}[1. Ableitung]
	\( f:D\rightarrow \mathbb{R} \) heißt differenzierbar in \( x_0 \in D \) falls obiger Grenzwert existiert.
\end{definition}
\begin{note}
  Die durch \( f \) und \( x_0 \) festgelegte Zahl \( a \) heißt 1. Ableitung von \( f \) in \( x_0 \).


\begin{equation*}
	f'(x_0) = \dot{f}(x_0) = \lim_{x\rightarrow x_0}\left(\frac{f(x)-f(x_0)}{x-x_0}\right) =:\text{ "`Differentialquotient"'}
\end{equation*}

\end{note}

\begin{note}
  Graph von \( p(x) \) lässt sich deuten als Tangente an \( f \) im Punkt \( x_0 \) (siehe \ref{fig:Approximierung})
\end{note}

\begin{example}[Einfache Differentiationsgleichung \( f(x)=x^n \)]
\begin{align*}
	\lim_{x \rightarrow x_0} \left(\frac{f(x)-f(x_0)}{x-x_0} \right)
	&= \lim_{x \rightarrow x_0} \left(\frac{x^n-x_0^n}{x-x_0}\right) \\
	&= \lim_{x \rightarrow x_0} \left(\sum_{k=0}^{n-1}x_0^kx^{n-1-k}\right) \\
	&= \sum_{k=0}^{n-1}x_0^kx_0^{n-1-k} \\
	&= \sum_{k=0}^{n-1}x_0^kx_0^{n-1} = nx^{n-1} 
\end{align*}

\[
	f(x)=x^n \quad \text{allg.: } \forall x_0 \in \mathbb{R} \quad f'(x)=nx^{n-1}
\]
\end{example}
\noindent Idee von Leibniz: approximiere die Steigung der Tangente in \( x_0 \) durch Sekanten

\[	\text{Grenzprozess } x \rightarrow x_0 \]
\[	\Leftrightarrow \]
\[	\text{Sekantensteigung } \equals \text{ Tangentensteigung} \]
\[	\Leftrightarrow \]
\[	\text{Differentialquotient} \leftrightarrow \text{Differenzenquotient} \]
\[	\frac{f(x)-f(x_0)}{x-x_0} \rightarrow \frac{\Delta f}{\Delta x}\]

\subsubsection*{Bezeichnungen}
\begin{equation*}
	f'(x_0) \qquad \frac{\diff f}{\diff x}(x_0) \qquad \left.\frac{\diff f}{\diff x}\right|_{x=x_0}
\end{equation*}
%
setze \( x = x_0 \):
\begin{equation*}
  \lim_{x \rightarrow x_0}\left(\frac{f(x)-f(x_0)}{x-x_0}\right) = \lim_{h \rightarrow 0}\left(\frac{f(x_0+h)-f(x_0)}{h}\right)
\end{equation*}
% LH: man geht von x->x0 über zu h->0
%\todo{JP:Zusammenhang ist mir unklar.}

\subsection{Rechenregeln für Funktionen}
\label{sec:rechenregeln_f_funktionen}

\begin{theorem}[Differenzierbarkeit $\Rightarrow$ Stetigkeit]
  Jede in \( x_0 \in D \) differenzierbare Funktion \( f \) ist stetig in \( x_0\).
\end{theorem}
\begin{proof}
  \[ \lim_{x \rightarrow x_0} (f(x)-f(x_0)-\underbrace{f'(x_0)}_{\in \mathbb{R} \text{ und fest} }\underbrace{(x-x_0)}_{\longrightarrow 0}) = 0 \]
\end{proof}

\begin{theorem}[Stetigkeit $\not\Rightarrow$ Differenzierbarkeit]
  Nicht jede stetige Funktion ist differenzierbar.
\end{theorem}

\begin{example}
\[	f(x)=|x| \qquad \text{ in } x_0=0 \text{ hat die Funktion eine "`Spitze"' } \]
\[	\text{formal: } \lim_{h \rightarrow 0}\frac{f(x_0+h)-f(x_0)}{h} = \lim_{h \rightarrow 0}\frac{|h|}{h}  \]
  Fallunterscheidung:
  \begin{itemize}
    \item von links \[\displaystyle\lim_{h \rightarrow 0^-}\frac{|h|}{h} = -1\]
    \item von rechts \[\displaystyle\lim_{h \rightarrow 0^+}\frac{|h|}{h} = 1\]
  \end{itemize}

\end{example}

\begin{center}
  \begin{tikzpicture}[scale=0.7, x=2cm, y=2cm]
    \draw[->,color=black] (-2.5,0) -- (2.5,0);
    \foreach \x in {-2,-1,1,2}
    \draw[semithick,shift={(\x,0)},color=black] (0pt,2pt) -- (0pt,-2pt) node[below] {\footnotesize $\x$};
    \draw[->,color=black] (0,-0.5) -- (0,2.5);
    \foreach \y in {,1,2}
    \draw[semithick,shift={(0,\y)},color=black] (2pt,0pt) -- (-2pt,0pt) node[left] {\footnotesize $\y$};
    \draw[color=black] (0pt,-10pt) node[right] {\footnotesize $0$};
    \draw[blue,samples=200,domain=-2:2] plot ({\x},{(abs(\x))}) node[above]{\( f(x)=|x|\)};
  \end{tikzpicture}
  \captionof{figure}{Betragsfunktion \( f(x)=|x| \)}
  \label{fig:Betragsfunktion}
\end{center}

\subsubsection*{Einfache differenzierbare Funktionen}
\begin{enumerate}
	\item \( f(x)=c \text{ (Konstante)} \implies f'(x)=0 \)
	\item \( f(x)=ax+b \implies f'(x)=a \)
	\item \(f(x)=x^2 \implies f'(x)=2x \)\newline
		  \(f(x)=x^n \implies f'(x)=nx^{n-1} \)
	\item \( f(x)=\sqrt{x} \qquad x>0 \implies f'(x)=\frac{1}{2\sqrt{x}} \)
	 \begin{proof}
	   \[ x_0>0 \qquad \lim_{x \rightarrow x_0}\frac{\sqrt{x}-\sqrt{x_0}}{x-x_0} =
            \lim_{x \rightarrow x_0}\frac{1}{\sqrt{x}+\sqrt{x_0}}=\frac{1}{2\sqrt{x_0}} \]
	 \end{proof}

\end{enumerate}


% Graph der Wurzelfunktion mit der Bemerkung das x_0=0 nicht diffbar ist
\begin{center}
\begin{tikzpicture}[scale=0.7, x=2cm, y=2cm]
	\draw[->,semithick,color=black] (-0.5,0) -- (3.5,0);
	\foreach \x in {,1,2,3}
	\draw[shift={(\x,0)},color=black] (0pt,2pt) -- (0pt,-2pt) node[below] {\footnotesize $\x$};
	
	\draw[->,semithick,color=black] (0,-0.5) -- (0,2);
	\foreach \y in {,1}
	\draw[shift={(0,\y)},color=black] (2pt,0pt) -- (-2pt,0pt) node[left] {\footnotesize $\y$};
	\draw[blue,domain=0:3,samples=200] plot ({\x},{(sqrt(\x))}) node[above]{\( f(x)=\sqrt{x}\)};
\end{tikzpicture}
\captionof{figure}{Wurzelfunktion: an der Stelle \( x_0=0 \) nicht differenzierbar}
\label{Wurzelfunktion}
\end{center}


\subsubsection*{Rechenregeln} % (fold)
\label{sub:rechenregeln}

\( f,g:D\rightarrow\mathbb{R} \) für alle \( x \in D \) differenzierbar

\begin{enumerate}
	\item \( f \pm g \) differenzierbar mit \( (f(x)\pm g(x))'=f'(x)\pm g'(x) \)
	\item \( c \cdot f \) differenzierbar, \( c \in \mathbb{R} \), \( (c \cdot f(x))'=c\cdot f'(x) \)
	\item Produktregel \( (f(x) \cdot g(x))' = f'(x) \cdot g(x)+f(x) \cdot g'(x) \)
	\item Quotientenregel, \( g(x) \neq 0 \), \( \left(\cfrac{f(x)}{g(x)}\right)'=\cfrac{f'(x)\cdot g(x)-f(x)\cdot g'(x)}{g(x)^2} \) \\
          speziell: \( \left(\cfrac{1}{g(x)}\right)'=-\cfrac{g'(x)}{g(x)^2} \)
\end{enumerate}

\begin{proof}
  \begin{enumerate}
    \item trivial
    \item trivial
    \item Herleitung:
      \begin{align*}
        (f\cdot g)'(x) &= \lim_{h \rightarrow 0}\frac{(f \cdot g)(x+h)-(f \cdot g)(x)}{h}\\
        &= \lim_{h \rightarrow 0}\frac{f(x+h) g(x+h) - f(x)  g(x)} h \\
        &= \lim_{h \rightarrow 0}\frac{f(x+h) g(x+h) \color{blue}- f(x) g(x+h)\color{black} - f(x)  g(x) \color{blue}+ f(x)  g(x+h)\color{black}} h\\
        &= \lim_{h \rightarrow 0}\frac{(f(x+h) - f(x))\cdot g(x+h) + (g(x+h) - g(x)) \cdot f(x) } h\\
        &= \lim_{h \rightarrow 0}\frac{f(x+h) - f(x)} h \cdot g(x+h) + \frac{g(x+h) - g(x)} h \cdot f(x)\\
        &= f'(x) \cdot g(x) + g'(x) \cdot f(x)
      \end{align*}
    \item \label{proof:quotregel} Herleitung:
      \begin{align*}
        \left(\frac f g\right)'(x) &= \left(f(x) \cdot \left( \frac 1 {g(x)} \right) \right)' \\
        &= \frac{f'(x)}{g(x)}  + f(x) \cdot \left( \frac 1 {g(x)} \right)'
          \intertext{Nebenrechnung: Kehrwertregel}
        \left( \frac 1 {f(x)} \right)' &= \lim_{h \rightarrow 0}\frac{\frac 1 {f(x+h)} - \frac 1 {f(x)}}{h}\\
        &= \lim_{h \rightarrow 0}\frac{f(x) - f(x+h)}{h \cdot f(x+h) \cdot f(x)}\\
        &= -\lim_{h \rightarrow 0}\frac 1 {f(x+h) \cdot f(x)} \cdot \frac{f(x+h)-f(x)} h \\
        &= -\frac{f'(x)}{f(x)^2}
          \intertext{damit folgt:}
        \left(\frac f g\right)'(x) &= \frac{f'(x)}{g(x)}  - f(x) \cdot \frac{g'(x)}{g(x)^2} \\
        &= \frac{f'(x)\cdot g(x) - f(x) \cdot g'(x)}{g(x)^2}
      \end{align*}
  \end{enumerate}
\end{proof}

% subsubsection rechenregeln (end)
% subsection rechenregeln_für_funktionen (end)