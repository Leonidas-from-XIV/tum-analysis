\newpage
\lecture{2009-11-24}
\section{Differenzierbarkeit}
\subsection{Ableitungsbegriff}

\begin{description}
	\item[Bisher:]Stetige Funktion, die den Grenzprozess im Definitionsbereich auf den Wertebereich überträgt.
	\item[Frage:]Ist es \emph{lokal} möglich eine Funktion \( f(x) \) genauer zu beschreiben? 
\end{description}

\noindent Wie sieht eine Funktion \( p(x) \) bzw. eine Gerade aus, die \( f(x) \) in \( x_0 \) möglichst gut beschreibt?
\[
p(x)=ax+b 
\]
\begin{figure}[h]
	\begin{center}
\begin{tikzpicture}[scale=0.7, x=2cm, y=2cm]
	\draw[blue,domain=-2:2] plot ({\x},{(\x)^2}) node[above]{};
	\draw[red,domain=-2:2] plot ({\x},{3}) node[above]{\( p_1 \)};
	\draw[red,domain=-2:0.5] plot ({\x},{(-\x)}) node[above]{\( p_2 \)};  
\end{tikzpicture}
\end{center}
\caption{Die Geraden \( p_1(x) \) bzw. \( p_2(x) \) aus der Interpolation sind keine guten Repräsentanten}
\end{figure}

\todo{JP:Welche Interpolation? Mein Mitschrieb gib nicht mehr her.}

\noindent Die Konstruktion der Geraden \( p(x) \), die die Funktion \( f(x) \) in der Nähe von \( x_0 \) gut beschreibt:

\[
	f:D\rightarrow\mathbb{R} \qquad D\subset\mathbb{R} \qquad x_0 \in D
\]

\begin{description}
	\item[1. Forderung] \( p(x_0)=f(x_0) \) 
	\begin{align*}
			p(x) &= ax+b \\
			f(x_0) &= ax_0+b \\
			\implies b &= f(x_0)-ax_0 \\
			\implies p(x) &= a(x-x_0)+f(x_0)
	\end{align*}
	\item[2. Forderung] \( d(x)=f(x)-p(x) \) soll in der Nähe von \( x_0 \) möglichst klein werden. Damit gilt \( d(x) = f(x) -f(x_0)-a(x-x_0) \)
	\item[1. Versuch] \( f(x) \) stetig  
	\[\lim_{x\rightarrow x_0} d(x)=\lim_{x\rightarrow x_0}(\underbrace{f(x)-f(x_0)}_{\rightarrow 0}-a\underbrace{(x-x_0)}_{\rightarrow 0})=0\]Dies ist nichts Neues, da die Information von \( a \) durch den Faktor\( (x-x_0) \) verdeckt wird.
	\item[2. Versuch] Forderung an Parameter \( a \in \mathbb{R} \)
	\[
		\lim_{x\rightarrow x_0}\frac{f(x)-f(x_0)-a(x-x_0)}{x-x_0}\stackrel{!}{=}0
	\]
	Die Division durch \( x-x_0 \) neutralisiert den Faktor \( a \).
\end{description}

\begin{definition}[Lineare Approximierbarkeit]
	\( f:D \rightarrow \mathbb{R} \) heißt linear approximierbar (Gerade) in \( x_0 \in D \), falls ein \( a \in \mathbb{R} \) existiert mit \( \frac{f(x)-f(x_0)-a(x-x_0)}{x-x_0}=0 \)
\end{definition}
\noindent Bemerkungen

\begin{itemize}
	\item Die Herleitung ist auch auf \( \mathbb{R}^n \) übertragbar.
	\item \( f \) heißt auch \emph{differenzierbar} in \( x_0 \)
\end{itemize}
\begin{example} 
\[
f(x)=x^2 \qquad x_0=1 \implies f(x_0)=1
\]
\begin{align*}
	\frac{f(x)-f(x_0)-a(x-x_0)}{x-x_0} &= \frac{x^2-1-a(x-1)}{x-1} \\ &= \frac{(x-1)(x+1)-a(x-1)}{x-1} \\&= x+1-a \qquad \\ x \rightarrow 1 \implies a&=2 \text{ damit Grenzwert gleich }0\text{ ist.}
\end{align*}
\end{example}
	

\begin{definition}[1. Ableitung]
	\( f:D\rightarrow \mathbb{R} \) heißt differenzierbar in \( x_0 \in D \) falls obiger Grenzwert existiert.
\end{definition}
Bemerkung: Die durch \( f \) und \( x_0 \) festgelegte Zahl a heißt 1. Ableitung von f in \( x_0 \).

\begin{align*}
	f'(x_0) oder fpunkt(x_0) \\
	f'(x_0) = \text{limes} \frac{f(x)-f(x_0)}{x-x_0} \text{Differentioalquotient}
\end{align*}

% ZUm Bild: "p(x) as Tangente in f(x_0) liegt gut!"

Bemerkung: Graph von \( p(x) \) läßt sich deuten als Tangente an \( f \) im Punkt \( x_0 \).

1. einfache Differentiationsgleichung \( f(x)=x^n \) bilde

\begin{align*}
	\frac{f(x)-f(x_0)}{x-x_0}=\frac{x^n-x_0^n}{x-x_0} \\
	= \sum_{k=0}^{n-1}x_0^kx^{n-1-k} x \rightarrow x_0 \sum_{k=0}^{n-1}x_0^kx_0^{n-1-k} = \sum_{k=0}^{n-1}x_0^kx_0^{n-1} = nx^{n-1} \\
	f(x)=x^n \quad \text{allg.} \forall x_0 \in \mathbb{R} \quad f'(x)=nx^{n-1} 
\end{align*}

Die Idee von Leibnitz: Approximiere die Steigung der Tangente in \( x_0 \) durch Sekanten.

% Bild von Tangente an f(x) in x_0

Grenzprozeß \( x \rightarrow x_0 \) Sekantensteigung = Tangentensteigung
Differentialquotient Differenzenquotient
\[
	\frac{f(x)-f(x_0)}{x-x_0} \rightarrow \frac{\Delta f}{\Delta x}
\]

Bezeichnung:
\[
	f'(x_0) \qquad \frac{df}{dx}(x_0) \qquad \left.\frac{df}{dx}\right|_{x=x_0}
\]

setze \( x = x_0; \frac{f(x)-f(x_0)}{x-x_0} = \frac{f(x_0+h)-f(x_0)}{h}\)

\subsection{Rechenregeln für Funktionen}
\label{sec:rechenregeln_f_funktionen}

\begin{theorem}[1. Aussage]
	Jede in \( x_0 \in D \) differenzierbare Funktion \( f \) ist stetig in \( x_0\).
	\( \text{limes x->x0} (f(x)-f(x0)-f'(x_0)(x-x_0)) = 0 \)
		% x-x0 geht gegen null, f'(x_0) elem R und fest
\end{theorem}
\begin{align*}
	f(x)=|x| \qquad in x_0=0 \qquad \text{Spitze}
	\text{formal } \frac{f(x_0+h)-f(x_0)}{h} \implies \lim{h \rightarrow 0}\frac{|h|}{h} 
	% Am Schluss folgt dann noch die Fallunterscheidung.
	% Bild der Betragsfunktion
\end{align*}
\begin{theorem}[2. Aussage]
	Nicht jede stetige Funktion ist differenzierbar.

\end{theorem}

\subsubsection*{Einfache differenzierbare Funktionen}
\begin{itemize}
	\item \( f(x)=c \text{Konstante} \implies f'(x)=0 \)
	\item \( f(x)=ax+b \implies f'(x)=a \)
	\item \begin{align*}
			f(x)&=x^2 \implies f'(x)=2x \\
			f(x)&=x^n \implies f'(x)=nx^{n-1}
		\end{align*}
	\item \( f(x)=\sqrt{x} \qquad x>0 \implies f'(x)=\frac{1}{2\sqrt{x}} \)
\end{itemize}
zu IV:
\[
x_0>0 \qquad \frac{\sqrt{x}-\sqrt{x_0}}{x-x_0} =\frac{1}{\sqrt{x}+\sqrt{x_0}}\lim\limits_{x \rightarrow x_0}=\frac{1}{2\sqrt{x_0}}
\]
% Graph der Wurzelfunktion mit der Bemerkung das x_0=0 nicht diffbar ist

\subsubsection*{Rechenregeln} % (fold)
\label{sub:rechenregeln}

\( f,g:D\rightarrow\mathbb{R} \) für alle \( x \in D \) differenzierbar-

\begin{itemize}
	\item \( f \stackrel{+}{-} g \) differenzierbar mit \( (f(x)\stackrel{+}{-}g(x))'=f'(x)\stackrel{+}{-}g'(x) \)
	\item \( c \cdot f \) differenzierbar, \( c \ in \mathbb{R} \qquad (c \cdot f(x))'=c\cdot f'(x) \)
	\item Produktregel \( (f(x)g(x))' = f'(x)g(x)+f(x)+g'(x) \)
	\item Quotientenregel \( (g(x) \neq 0) \qquad (\frac{f(x)}{g(x)})'=\frac{f'(x)g(x)-f(x)g'(x)}{g(x)^2} \text{speziell:} (\frac{1}{g(x)})'=-\frac{g'(x)}{g(x)^2}\)
\end{itemize}

Zu (iii) \( \frac{(fg)(x_0+h)-(fg)(x_0)}{h}=\frac{1}{h} \)
% subsubsection rechenregeln (end)
% subsection rechenregeln_für_funktionen (end)