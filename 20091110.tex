\chapter{Funktionen, Stetigkeit, Differenzierbarkeit}

\lecture{2009-11-10}

\section{Funktionen, Polynome}
\subsection{Grundbegriffe zu Funktionen}
	Reelwertige Funktion $f:D \rightarrow \mathbb{R}, D \subseteq \mathbb{R}$

\begin{example}
	\begin{itemize}
			\item{
			Lineare Funktion $f(x)=ax+b$ \newline
		    Gerade durch $(0,b)$ und $\left(-\frac{b}{a},0\right)$
		
			Erinnerung:
			\begin{center}
				\begin{tikzpicture}[line cap=round,line join=round,>=triangle 45,x=2.5cm,y=2.5cm]
				\draw[->,color=black] (-0.5,0) -- (2,0);
				\foreach \x in {,1}
				\draw[shift={(\x,0)},color=black] (0pt,2pt) -- (0pt,-2pt) node[below] {\footnotesize $\x$};
				\draw[->,color=black] (0,-0.5) -- (0,1.2);
				\foreach \y in {}
				\draw[shift={(0,\y)},color=black] (2pt,0pt) -- (-2pt,0pt) node[left] {\footnotesize $\y$};
				\clip(-0.8,-0.8) rectangle (2.8,1.6);
				\draw [domain=-0.5:1] plot(\x,{(--0.25--0.8*\x)/1});
				
				\draw[color=black] (-0.4, 0.15) node {$(\frac{b}{a},0)$};
				\fill [color=black] (-0.3125,0) circle (1.5pt);
				
				\draw[color=black] (0.25,0.20) node {$(0,b)$};<
				\fill [color=black] (0,0.25) circle (1.5pt);

				\draw[color=black] (2,0.08) node [anchor=north] {x};
				\draw[color=black] (0,1) node [anchor=south west] {y};
				\end{tikzpicture}
			\end{center}
			}
	\item{
	Quadratische Funktion (Parabel)
	\begin{align*}
		f(x) &= ax^2 +bx+c \\
		&= a\left(x^2+\frac{b}{a}x+\frac{b^2}{4a^2}\right)+c-\frac{b^2}{4a} &\\
		&= a\left(x+\frac{b}{2a}\right)^2+c - \frac{b^2}{4a} &\\
		&= \underbrace{(x-x_0)^2}_{\parbox{16mm}{\centering\tiny Scheitelpunkt in $x$-Richtung verschoben}}
		+\underbrace{y_0}_{\parbox{16mm}{\centering\tiny Scheitelpunkt in $y$-Richtung verschoben}}
		& x_0 = -\frac{b}{2a} \hspace{10mm} y_0=c-\frac{b^2}{4a}
	\end{align*}
	}
	\end{itemize}
\end{example}

\begin{definition}[Eigenschaften von Funktionen]
\begin{itemize}
	
	\item Symmetrie
	\[
	\begin{matrix}[l]
		&\text{gerade:}		&f(-x) = f(x)	&\text{Beispiele:}	&f(x) = x^2; 	&f(x) = \cos(x) \\
		&\text{ungerade:}	&f(-x) = -f(x) &             &f(x) = x^{2n+1};	&f(x) = \sin(x)
	\end{matrix}
	\]
	
	\item Monotonie \newline \(f:D\rightarrow \mathbb{R}\) heißt
	\begin{itemize}
		\item \emph{monoton wachsend}, wenn \(x_1<x_2 \implies f(x_1) \leq f(x_2)\)
		\item \emph{streng monoton wachsend}, 
			  wenn \(x_1<x_2 \implies f(x_1) < f(x_2)\)
	\end{itemize}		
\end{itemize}

%JP: Sehr hässliche Lösung, aber subfloats und Listen mögen sich nunmal nicht.
\begin{figure}[h]
	\advance\leftskip+2cm
	\subfloat[Monoton]
	{
	\begin{tikzpicture}[scale=1, x=2cm, y=2cm]
		\draw[blue,domain=0:1] plot ({\x},{\x*\x*\x}) node[above] 
		{};
		\draw[blue,domain=-1:0] plot ({\x},{0}) node[above] 
	    {};
		\draw[blue,domain=-2:-1] plot ({\x},{(\x+1)^3}) node[above] 
		{};
	\end{tikzpicture}
}
\subfloat[Streng monoton]
{
	\begin{tikzpicture}[scale=1, x=2cm, y=2cm]
		\draw[blue,domain=-2:0.5] plot ({\x},{exp(\x)}) node[above] 
		{};
	\end{tikzpicture}
}
\caption{Beispiele für monoton wachsende Funktionen}
\end{figure}
%\todo{Listeneinzug auch auf die Bilder anwenden}

\end{definition}

\begin{definition}[Operationen]
\( f,g: D \rightarrow \mathbb{R} \)
\begin{itemize}
	\item \((f \stackrel[\ast]{+}{-} g)(x) = f(x)  \stackrel[\ast]{+}{-} g(x) \)
	\item \(a \in \mathbb{R}, (af)(x)=af(x)\)
	\item \(\displaystyle g \neq 0, \left(\frac{f}{g}\right)(x)=\frac{f(x)}{g(x)}\)
	\item 	Komposition (Hintereinanderausführung)
	\begin{align*} &f:I\rightarrow\mathbb{R};  g:D\rightarrow\mathbb{R}; g(D) \subseteq I; h:D\rightarrow\mathbb{R} \\ &h = f \circ g  \text{ ist definiert durch } h(x) = f( g(x) )
	\end{align*}
\end{itemize}

\end{definition}

\begin{note} \(f \circ g \) im Allgemeinen \( \neq g \circ f \)
	\begin{align*}
		f(x) &= \cos(x) g(x) &= 1-x^2, D &= I=\mathbb{R} \\
		(f \circ g)(x) &= f(g(x)) &= f(1-x^2) &= \cos(1-x^2) \\
		(g \circ f)(x) &= g(f(x)) &= g(\cos(x)) &= 1-\cos^2x = \sin^2x 
	\end{align*}	
\end{note}


\subsection{Polynome und rationale Funktionen} % (fold)
\label{sub:polyFunk}

Eine Funktion \( p:\mathbb{R} \rightarrow \mathbb{R} \) heißt Polynom vom Grade $n$, falls $p$ die Darstellung besitzt:
\begin{align*}
	&p(x)=a_0+a_1x+\ldots+a_n x^n = \sum_{k=0}^{n} a_kx^k\\
	&\text{Monome }x^k, k=0,\ldots,n; \\
	&\text{Polynomkoeffizienten }a_0,\ldots,a_n \text{ mit } a_n \neq 0
\end{align*}

\subsubsection*{Operationen} % (fold)
\label{ssub:polyOp}

\[
  \sum_{k=0}^{n}  a_kx^k  \pm
  \sum_{k=0}^{n}  b_kx^k  =
  \sum_{k=0}^{n}  (a_k \pm b_k)x^k 
\]
Umordnung möglich, da \( n \) endlich.
% subsubsection operationen (end)

\subsubsection*{Verbindung zur Linearen Algebra} % (fold)
\label{ssub:polyLinAlg}
\( x^0,x^1,\ldots,x^n \) Basis der Polynome vom Grade \( \leq n \) \newline
\( \operatorname{span} \{x^0 ,\ldots,x^n\}=\Pi_n \) Vektorraum im Reellen der Polynome vom Grade \( \leq n \text{ über } \Pi \)

% subsubsection Verbindung zur Linearen Algebra (end)

\subsubsection*{Evaluieren von Polynomen} % (fold)
\label{ssub:evaluieren_von_polynomen}

\begin{equation*}
	p(x)=a_0+a_1x+\ldots+a_nx^n
\end{equation*}
naiv:
\begin{itemize}
	\item \( n \) Multiplikationen \( a_kx^k \)
	\item \( (n-1) \) Potenzen als Multiplikation \( x^k=x(x^{k-1}) \)
	\item \( n \) Additionen
\end{itemize}
%
besser: \emph{Hornerschema} (\( \approx \) 1800)
\[
	p(x)=(\ldots((a_nx+a_{n-1})x+a_{n-2})x\ldots)x+a_0
\]


\begin{lstlisting}[caption=Codierung des Hornerschemas] 
p := a[n];
for k := n - 1 to 0 do
	p := p * x + a[k];
print(x,p);
\end{lstlisting}
Bei dieser Codierung des Hornerschemas wird nur ein Speicherplatz für \( p \) belegt.

% subsubsection evaluieren_von_polynomen (end)

\subsubsection*{Polynom-Nullstellen} % (fold)
\label{ssub:polynom_nullstellen}

\begin{definition}
				Als Nullstelle einer Funktion \( f:D\rightarrow\mathbb{R} \) wird jede Lösung \( \widehat{x} \in D \) der Gleichung \( f(\widehat{x})=0 \) bezeichnet.
\end{definition}
\noindent Speziell die Polynome \( b \in \mathbb{R} \) haben genau dann eine Nullstelle, falls der Linearfaktor \( (x-b) \) abspaltbar ist, d. h.

\[
	p(x)=(x-b) \cdot r(x) \qquad \operatorname{grad}(r)=n-1
\]
$\ell$-fache Nullstellen:
\[
	p(x)=(x-b)^\ell \cdot q(x) \qquad \operatorname{grad}(q)=n-\ell
\]
\begin{theorem}[Fundamentalsatz der Algebra]
	$p$ zerfällt über \( \mathbb{C} \) in $n$ Linearfaktoren (mit Vielfachheiten gezählt)
\end{theorem}

\begin{note}
\begin{align*}
	n &= 2 \rightarrow \text{quadratische Gleichung} \\
	n &= 3 \rightarrow \text{Cardano} \\
	n &= 4 \rightarrow \text{noch analytisch lösbar} \\
	n &> 4 \rightarrow \text{nicht mehr analytisch, nur noch iterativ lösbar}
\end{align*}
\end{note}
% subsubsection polynom_nullstellen (end)

\subsubsection*{Wachstum der Polynome} % (fold)
\label{ssub:wachstum_der_polynome}

\begin{align*}
		&p(x)=a_nx^k+\ldots+a_1x+a_0=a_nx^n\left(1+\frac{a_n-1}{a_n\cdot x}+\ldots+\frac{a_1}{a_n\cdot x^{n-1}}+\frac{a_0}{a_n\cdot x^n}\right), \\
		&\text{wobei } a_n\neq0 \text{ sowie } x\neq0 
\end{align*}
Asymptotisches Verhalten: \( x \longrightarrow \pm \infty:p(x)\longrightarrow a_nx^n \) \newline "`Wie verhält sich eine Funktion, wenn x über alle Maßen wächst?"'

\begin{definition}[Rationale Funktion]
	\[
			r(x)=\frac{p(x)}{g(x)} \qquad p,q \text{ sind Polynome}
	\]
\end{definition}
\noindent Die Nullstelle des Nenners (\( g(x)=0 \)) nennt man Polstelle. Die Division von Polynomen kann man mit dem Euklidischen Algorithmus durchführen (siehe auch \cite{bauerFaktPoly} und \cite{knuthPolynoms}).
% subsubsection wachstum_der_polynome (end)

\subsubsection*{Polynominterpolation} % (fold)
\label{ssub:polynominterpolation}

\begin{description}
	\item[1. Fragestellung:] Es gibt \( 2n(n+1) \) Messdaten, durch die ein Polynom konstruiert werden soll.
	\begin{figure}[h]
	\begin{tikzpicture}[scale=0.1, x=9cm, y=0.03cm]
		\draw[blue,domain=-6:6] plot ({\x},36*{\x}-12*{\x}^2+3*{\x}^3) node[above] 
		{};
		\fill [color=black] (0,0) circle (30pt);
		\fill [color=black] (1,27) circle (30pt);
		\fill [color=black] (-1,-51) circle (30pt);
		\fill [color=black] (2,48) circle (30pt);
		\fill [color=black] (-2,-144) circle (30pt);
	\end{tikzpicture}
	\caption{Konstruktion eines Polynoms durch Stützpunkte/Messdaten}
	\end{figure}
	
	\item[2. Fragestellung:] Möglichst gutes Approximieren einer unbekannten Funktion \( f(x) \). Von \( f \) sind Messwerte bekannt.
	
	\item[3. Fragestellung:] "`Schöne Bilder"', Buchstaben, etc. zeichnen. (\TeX)
\end{description}

\begin{theorem}[Gut gestelltes Problem]
	\begin{itemize}
		\item eine Lösung existiert
		\item die Lösung ist eindeutig
		\item kleine Änderungen der Daten bedingen kleine Änderungen der Lösung
	\end{itemize}
	
\end{theorem}
% subsubsection polynominterpolation (end)
% subsection Polynome und rationale Funktionen (end)

%%% LARS: REVIEWSTATE