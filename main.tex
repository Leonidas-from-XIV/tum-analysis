\documentclass[a4paper,10pt]{book}
\usepackage[utf8]{inputenc}
\usepackage{amsmath}
\usepackage{amsfonts}
\usepackage{amssymb}

%opening
\title{Analysis für Informatiker}
\author{Michael Kerscher}

\begin{document}

\maketitle


\chapter{Grundlagen: Zahlbegriff}
\section{Zahldarstellung}
\subsection{Natürliche Zahlen zu reelle Zahlen}
diskret: $1,2,3,...$
$\mathbb{N} = {1,2,3,...}$ Menge der natürlichen Zahlen
%\begin{definition}
 Menge ist Zusammenfassung bestimmter wohlunterschiedener Objekte unserer Anschauung oder unserers Denkens
%\end{definition}

Axiomensystem nach Peano
\begin{enumerate}
 \item $1 \in \mathbb{N}$ (Anfang)
 \item $n \in \mathbb{N} \Rightarrow (n+1) \in \mathbb{N}$ (Nachfolger)
 \item $n \neq m \Rightarrow (n+1) \neq (m+1)$
 \item $n \in \mathbb{N} \Rightarrow (n+1) \neq 1$
 \item $A \in \mathbb{N}: 1 \in A \land (\forall n: n \in A \Rightarrow (n+1) \in A) \Rightarrow A = \mathbb{N}$ (Vollständigkeitsaxiom, alle natürlichen Zahlen werden erfasst)
\end{enumerate}

\begin{enumerate}
 \item Erweiterung zu $\mathbb{Z} = {...,-2,-1,0,1,2,...}$ ganze Zahlen. In $\mathbb{Z}$ Operationen $+,-$
 \item Erweiterung zu $\mathbb{Q}$ rationale Zahlen durch $*,/$ $q=\frac{m}{n}; m \in \mathbb{Z}, n \in \mathbb{N}$
$\mathbb{Q} = {x | x=\frac{m}{n}, m \in \mathbb{Z}, n \in \mathbb{N}}$
Q ist dicht, d.h. zwischen $q_1, q_2$ liegt ein $\tilde{q}$
 \item Erweiterung durch $\sqrt{}$ bzw. Quadrieren $x^2=a$
$a=2 \Rightarrow x = \sqrt{a} \notin \mathbb{Q}, \sqrt{2}=1,4142...$ irrational
Beweis: (indirekt)
Aus $\sqrt{2}$ wäre Element in Q. $\sqrt{2}=\frac{p}{q}$ gekürzt ($p \in \mathbb{Z}, g \in \mathbb{N}$
$2g^2=p^2 \Rightarrow p^2$ gerade $\Rightarrow p=2\hat{p}$ und $2g^2=4\hat{p}^2$
$\Rightarrow g^2 = 2\hat{p}^2 \Rightarrow g=2\hat{g}$ 
$\Rightarrow$ Widerspruch zur gekürzten Form: $\sqrt{2}=\frac{p}{q}=\frac{2\hat{p}}{2\hat{g}}$

Aussage: $\sqrt{2}$ ist keine rationale Zahl.

Beweistechnik war indirekt, z.z. Aussage A $\Rightarrow$ Aussage B
indirekt: nicht Aussage B $\Rightarrow$ nicht Aussage A

Beispiel: direkte Beweistechnik
$p \in \mathbb{N}$ gerade $\Leftrightarrow p \Leftarrow 2 \hat{p} \Leftrightarrow p^2 = 4\hat{p}^2$ gerade
$p \in \mathbb{N}$ ungerade $\Leftrightarrow p = 2 \hat{p}+1 \Leftrightarrow p^2 = (2\hat{p}+1)^2=4\hat{p}^2+4\hat{p}+1$ ungerade
$\Rightarrow$ reelle Zahlen, formaler Weg siehe Buch F. Bornemann

 \item reelle Zahlen
  neue Kandidaten im Vergleich zu $\mathbb{Q}$
\begin{itemize}
 \item $\sqrt{}$-Bildung
 \item $c = 0,\overline{b_1...b_k}$ periodische Zahl
$c = \frac{b_1...b_k}{\underbrace{g...g}_{k-mal}} \in \mathbb{Q} \Rightarrow 10^k c -c = b_1 \ldots b_k$ periodischer Bruch

neue periodische Zahlen, die nicht als Bruch darstellbar sind, wären z.b. $c=0.101001000100001...$
 \item $\infty$-Summen, $e, \pi,\ldots$
\end{itemize}

\end{enumerate}

\subsection{Maschinenzahlen $\mathbb{M}$}
INTEGER (Assoziation: $\mathbb{Z}$, REAL (Assoziation: $\mathbb{R})$-Zahlen sind 2 Zypen unterschiedlicher Codierung

4 Byte für INTEGER, 4,8,10 Byte für REAL
zu Integer: 31 Bits für Mantisse, größte Zahl $\pm \underbrace{1\ldots1}_{31-mal}$

entspricht Zahldarstellung: $2^0+2^1+\ldots+2^{30}=2^{31}-1$
$\leadsto$ 10-er System: $2^{31}-1 \hat{=} x$ $2{10}\approx 10^3$
Zahlbereich: -2Mrd. bis 2 Mrd

zu REAL-4: Mantisse 23 Bits, Exponent 7 Bits

$\pm 0.\underbrace{\_\_\_\_\_}_{wieviele Stellen}E\pm\underbrace{\_\_\_\_}_{wieviele Stellen}$

Darstellung ist \emph{normalisiert} d.h. nach Dezimalpunkt keine 0-en
IEEE 754 ($\hat{=}$ VDE
\begin{itemize}
 \item Exponentenspielraum: $1\cdot 2^0+1\cdot2^1+\ldots+1\cdot2^6 = 2^7 -1 = 127$
  Exponent $10^{-127}$ bis $10{127}$
 \item Länge der Maschine im 10-er System
$1\cdot 2^0+1\cdot2^1+\ldots+1\cdot2^22=2^{23} \hat{=} 10^x$
$2^{23}\hat{=}10^x, x=23 \log_10 2 = $
$m \in \mathbb{M}, m=\pm 0.\_\_\_\_\_\_\_E\pm \_\_\_\_\_\_\_$ Lücke zwischen $-10^{-127}$ und $10^{_127}$ Faktor $10^7$ groß
\end{itemize}

\subsubsection{Rundungsfehler}

Wesentlich mitbestimmt von F.L.Bauer, Samelson, Zenger aus der Informatik und R.Bukisch, Chr. Reinsch aus der Mathematik und Wilkinson.

Hilfsmittel: Abbildung von den reellen Zahlen (Alltag) auf den Rechner

Abb: rd. round, Rundung. rd: $\mathbb{R}\rightarrow\mathbb{M}$

rd: $\{x|x \in [\frac{m_{i-1}+m_i}{2},\frac{m_{i}+m_{i+1}}{2}[\} \rightarrow m_i$

Bemerkung: Intervall-Arithmetik hat sich trotz Hardwareunterstützung \emph{nicht} bewährt: Intervallängen zu pessimistisch.

Definition: Abbildung
Vor(?): A,B Mengen
$f: A \rightarrow B, x \mapsto f(x)$
Die Abbildung ist eine Vorschrift, die jedem $x \in A$ ein element $x=f(x) \in B$ zuordnet

Charakterisierung von Abbildungen
\begin{itemize}
 \item gehören zu verschiedenen Argumenten verschiedene Funktionswerte, heißt $f$ \emph{injektiv},
$x_1,x_2 \in A, x_1 \neq x_2 \Rightarrow f(x_1) \neq f(x_2)$
 \item Wertebereich $C \subseteq A$
$f(C) = {f(x)|x\in C}$
 \item $f$ \emph{surjektiv}, falls $f(A)=B$
 \item Gilt $f$ injektiv und surjektiv, so heißt $f$ \emph{bijektiv}

$f: A \mapsto B$ ist genau dann bijektiv, falls zu jedem $y \in B$ genau ein $x \in A $ existiert mit $y=f(x)$ in diesem Fall existiert eine Umkehrabbildung $f^{-1}: B \mapsto A$
\end{itemize}

\subsection{Ungleichungen,  Betrag Kalkül-Teil}
Definition: Unter einer Ungleichung für reelle Zahlen $x,y$ verstehen wir einen Größenvergleich
\begin{center}
% use packages: array
\begin{tabular}{lll}
$x<y$ & "kleiner $<$" & kleiner $\leq$ \\ 
$x>y$ & "größer $>$" & größer $\geq$
\end{tabular}
\end{center}

Abschätzung: $x<y$ heißt Größe von $x$ durch Größe von $y$ abschätzen. Regelwerk für Abschätzungen (Anordnungsaxiome)
$\leadsto$ Zahlengerade.
\begin{enumerate}
 \item $x \leq y, a\leq b \Rightarrow x+a \leq y+b$
 \item $x<y, 0 \leq a \Rightarrow ax \leq ay$
 
$x<y, 0 < a \Rightarrow ax < ay$
 \item $0<x\leq y \Rightarrow 0 < \frac{1}{y} \leq \frac{1}{x}$
\end{enumerate}
Typische Aufgabe:
lege $a$ fest mit Eigenschaft
\begin{align*}
-3a-2 &\leq 5 &\leq -3a+4 \\
-3a\leq 7 & & 1 \leq -3a \\
-\frac{1}{3} &\leq a &\leq -\frac{7}{3}
\end{align*}

Definition: $S \subset \mathbb{R}$ heißt nach oben beschränkt, falls eine Zahl $b$ existiert mit $S\subseteq ]-\infty,b]$ "b obere Schranke von S"

Definition: Ist $S \subseteq \mathbb{R}$ nach oben beschränkt, so heißt die kleinste obere Schranke von $S$ das \emph{Supremum} $s:=\sup S$

analog: \emph{Infimum}: "größte unter Schranke" $u := \inf S$

Bemerkung: kleinste obere Schranke muss nicht element von $S$ sein, das \emph{Maximum} schon

\begin{itemize}
 \item $\sup \{x \in \mathbb{Q} | x^2 < 2 \} = \sqrt{2} \notin \mathbb{Q}$
 \item $\sup \{[a,b]\} = b \in [a,b]$
 \item $\inf \{ 1+\frac{1}{n}|n\in \mathbb{N}\} = 1$
\end{itemize}

Vollständigkeitsaxiom für $\mathbb{R}$
Jede nach oben beschränkte Menge reeller Zahlen besitzt ein \emph{Supremum}.$\mathbb{R}$ überabzählbar, $\mathbb{Q}$ abzählbar (siehe Bornemann)

Definition: Betrag $|a|, a \in \mathbb{R}$
$|a|=a \textrm{falls} a \geq 0 \land -a \textrm{falls} a<0$

Rechenregeln(Beträge):
\begin{itemize}
 \item $-|a|\leq a \leq |a|$
 \item $-|a| = |a|$
 \item $|ab| = |a||b|$
 \item $|\frac{a}{b}| = \frac{|a|}{|b|}, b\neq 0$
\end{itemize}

Anwendung: Dreiecksungleichung
z.z.: $|a+b| \leq |a|+|b|$. Dazu
\begin{align*}
-|a|&\leq a &\leq |a| \\
-|b|&\leq b &\leq |b| \\
\Rightarrow -(|a|+|b|)&\leq a+b &\leq |a|+|b| \\
-(|a|+|b|)&\leq |a+b| &\leq |a|+|b|
\end{align*}
Abstandsmessung $|x-a| \leq \epsilon$

Rechenbeispiel:
$x\in \mathbb{R}$ gesucht mit $\frac{3}{x-9} \leq \frac{2}{x+2}$. Nenner $x\neq9 \land x\neq -2$

(Zahlenstrahl mit Markierung von -2 nach links und Markierung von 9 nach rechts)

\begin{align*}
M_1 &= \{x \in \mathbb{R} | x < -2\} \\
M_2 &= \{x \in \mathbb{R} | -2 < x < 9\} \\
M_3 &= \{x \in \mathbb{R} | x > 9\}
\end{align*}

\begin{itemize}
 \item Diskussion von $M_1: x < -2$
\begin{align*}
|x-9|>0 &\Rightarrow \frac{3}{|x-9|} > 0 \\
x<-2 &\Rightarrow x+2<0 \Rightarrow \frac{2}{x+2}<0
\end{align*}
ganz $M_1$ zulässig
 \item Diskussion von $M_2: -2<x<9$
\begin{align*}
\Rightarrow &x+2 > 0 \\
&x-9<0 \textrm{d.h.} |x-9|=9-x \\
\textrm{zu prüfen:}
\frac{3}{9-x} > \frac{2}{x+2} \textrm{Betrag weg} \\
3x+6 > 18-2x \\
x > \frac{12}{5} = 2 \frac{2}{5}
\textrm{erlaubt:} \frac{12}{5}<x<9
\end{align*}
 \item Diskussion von $M_3: x>9$
\begin{align*}
 \Rightarrow \underbrace{|x-9|}_{>0} = x-9 \\
\frac{3}{x-9} > \frac{2}{x+2}
\Rightarrow x>-24
\end{align*}
$M_3$ zulässig

Ergebnis: $\{x|x<-2, \frac{12}{5}<x<9, 9<x\}$
\end{itemize}

Anwendung: Geometrie
\begin{align*}
M = \{(x,y)| |x|+|y| \leq 1 \} x,y \in \mathbb{R} \\
|x|+|y| \leq 1 \Rightarrow &x \geq 0, y \geq 0: x+y \leq 1 \Rightarrow y \leq 1-x \\
	&x \geq 0, y < 0: x-y \leq 1 \Rightarrow y \geq -1+x \\
	&x < 0, y \geq 0: -x+y \leq 1 \Rightarrow y \leq 1+x \\
	&x < 0, y < 0: -x-y \leq 1 \Rightarrow y \geq -1-x \\
\end{align*}
(Zeichnung einer "Raute" mit gefülltem Inhalt als Lösung)

Kreis um Ursprung mit Fläche
Radius $\{(x,y)| x^2+y^2\leq r^2 \}$

Kreislinie $\{(x,y)| x^2+y^2 = r^2 \}$
\end{document}
