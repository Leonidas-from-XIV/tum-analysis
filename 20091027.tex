\lecture{2009-10-27}

\section{Vollständige Induktion}

bisherige Beweistechniken:

\begin{enumerate}
 \item direkter Beweis: $A \implies B$
 \item indirekter Beweis: $\neg B \implies \neg A$
\end{enumerate}
jetzt: vollständige Induktion

\subsection{Schema}

\begin{enumerate}
 \item Induktionsbeginn (bzw. -anfang): zeige, dass Aussage $A(n)$ für ein festes $n_0 \in \mathbb{N}$ gilt
 \item Induktionsschluss: $A(n)$ zu $A(n+1)$
\end{enumerate}

\begin{example}
 \begin{itemize}
  \item $1+2+\ldots+n = \frac n 2 (n+1)$
    \begin{enumerate}
    \item Induktionsanfang: $n_0 = 1$, $1 = \frac 1 2 (1+1) = 1$
    \item Induktionsschluss: \begin{equation*}\underbrace{1+2+\ldots+n}_{A_n} + (n+1) = \frac 1 2 (n+1) + (n+1) = \frac{n+1}2 (n+2)\end{equation*}
    \end{enumerate}
  \item $1^2+2^2+\ldots+n^2 > \frac{n^3}3$
    \begin{align*}
      n_0 = 1:\text{\hspace{1cm}} &1 > \frac 1 3 \text{\hspace{1cm}} \\
      n \rightarrow n + 1 :\text{\hspace{1cm}} &\underbrace{1^2+2^2+\ldots+n^2}_{>\frac{n^3}3}+(n+1)^2 > \frac{n^3}3+(n+1)^2 \\
      &= \frac{n^3+3n^2+6n+3}3 > \frac{n^3+3n^2+3n+1}3 = \frac{(n+1)^3}3
    \end{align*}
 \end{itemize}
\end{example}

Eng verwandt mit der vollständigen Induktion ist die Rekursion:
\begin{itemize}
 \item lege $A_0$ fest
 \item setze $A_k$ als bekannt voraus für $k \leq n$
 \item definiere $A_n$ aus $A_k$
\end{itemize}

\begin{example}[Standard]
  \begin{itemize}
    \item Potenz: $a \in \mathbb{R}, n \in \mathbb{N}: a^0 = 1, a^{n+1} = a\cdot a^n$
    \item Fakultät: $n \in \mathbb{N}_0:$
    \begin{equation*} n! = \begin{cases} 1 & n = 0 \\ n \cdot (n-1)! & n \geq 1 \end{cases} \end{equation*}
  \end{itemize}
\end{example}

\begin{definition}[Summensymbol]
  \begin{equation*} a_j \in \mathbb{R}: a_0 + a_1 + \ldots + a_n =: \sum_{j=0}^{n} \left( a_j \right) \end{equation*}
\end{definition}

\subsection{Potenzen, Wurzel}

\begin{theorem}[Satz zur $m$-ten Potenz]\flush
  $x,y \in \mathbb{R}; n,m \in \mathbb{N}_0$
  \begin{enumerate}
   \item $x^nx^m = x^{n+m}$
   \item $(x^n)^m = x^{nm}$
   \item $(xy)^n = x^ny^n$
   \item $y \neq 0 \implies \left(\frac x y\right)^n = \frac{x^n}{y^n}$
   \item $0 < x < y \implies 0 < x^n < y^n$
   \item $n \geq 2, 0<x<1 \implies x^n<x<1$
   \item $n \geq 2, x > 1 \implies x^n>x>1$
   \item $x > 1, m > n \implies x^m > x^n$
  \end{enumerate}
  (\induction)
\end{theorem}
%
\noindent Hilfsmittel für Abschätzungen: Linearisierungen, Bernoulli-Ungleichung

\begin{proposition}
  für alle $x > -1, x \neq 0$ und $n \in \mathbb{N}, n \geq 2$ gilt
  \begin{equation*} \underbrace{(1+x)^n}_\text{nichtlin. Term} > \underbrace{1+nx}_\text{lin. Term} \end{equation*}
  \induction
\end{proposition}
%
Wunsch: in der Nähe von 1, d. h. $1+x$ für kleine $x$, soll der lineare Term den nichtlinearen Term ersetzen (Abschätzungstechnik)

\begin{theorem}[Elementare Summenformel]\flush
  $q \in \mathbb{R}, n \in \mathbb{N}$
  \begin{equation*} \sum_{k=0}^n \left( q^k\right) = \begin{cases}\frac{1-q^{n+1}}{1-q} & q \neq 1 \\ n+1 & q = 1\end{cases} \end{equation*}
  \induction
\end{theorem}

\begin{definition}[$n$-te Wurzel]
 $x\in \mathbb{R}$
 \begin{align*}
  x \geq 0,\; n \text{ gerade} &\implies \exists_{y \in \mathbb{R}} \left( y \geq 0 \implies y^n = x \right) \\
  x \text{ bel.},\; n \text{ ungerade} &\implies \exists_{y \in \mathbb{R}} \left( y^n = x \right) \\
 \end{align*}
Bezeichnung: $y = \sqrt[n]{x} = x^\frac 1 n$\\
Problem: $\sqrt{-1}$ sprengt $\mathbb{R}$, Definition der imaginären Einheit $\imag$\\$\leadsto$ komplexe Zahlen $\mathbb{C}$ (später)
\end{definition}

\subsubsection*{Rechenregeln für Wurzeln}

$x,y \in \mathbb{R}$ passend zu $n,m \in\mathbb{N}$
\begin{enumerate}
 \item $\sqrt[n]{xy} = \sqrt[n]x \sqrt[n]y$
 \item $\sqrt[n]{\sqrt[m]x} = \sqrt[nm]x$
 \item $\sqrt[n]{x^m} = (\sqrt[n]x)^m$
 \item $x<y \Rightarrow \sqrt[n]x < \sqrt[n]y$
 \item $0<x<1, m<n \Rightarrow \sqrt[m]{x} < \sqrt[n]{x}$
 \item $x>1, m<n \Rightarrow \sqrt[m]x > \sqrt[n]x$
 \item $\sqrt[n]{x^n} = \begin{cases} x & n \text{ ungerade} \\ |x| & n \text{ gerade} \end{cases}$
\end{enumerate}
\induction\\
gegen Ende der Vorlesung: $x > 0, \alpha \in \mathbb{R} \Rightarrow x^\alpha := \euler^{\alpha \operatorname{ln}(x)}$

\subsection{Binomischer Lehrsatz}

Zusammenhang zwischen Addition von Zahlen und Potenzbildung: $(a+b)^n$

\subsubsection*{Spezialfälle}

\begin{itemize}
 \item $(a+b)^2 = a^2+2ab+b^2$
 \item $a^2-b^2 = (a+b)(a-b)$
 \item $(a+b)^3=a^3+3a^2b+3ab^2+b^3$
\end{itemize}
%
Aufbau der Koeffizienten: "`Pascalsches Dreieck"'

\begin{definition}[Binomialkoeffizient ("`$n$ über $k$"')]
 \[ n,k \in \mathbb{N}_0: {n \choose k} := \frac{n!}{k!(n-k)!} \]
 (Bruch, aber ganzzahlig)
\end{definition}

\begin{proposition}[rekursive Berechnung] $n,k \in \mathbb{N}, k < n$
  \[ {n+1 \choose k} = {n \choose k-1} + {n \choose k} \]
\end{proposition}

\noindent Beweis aus Definition/Pascalsches Dreieck

\begin{proposition}[Binomischer Lehrsatz]
 \[ x,y \in \mathbb{R}, n \in \mathbb{N} \implies (x+y)^n =  \sum_{k=0}^n \left( {n \choose k} x^{n-k}y^k \right) \]
\end{proposition}

\section{Komplexe Zahlen $\mathbb{C}$}

\subsubsection*{Gründe}

Mathe $\sqrt{-1}$, Nachrichtentechnik, \TeX-\MF\  Version 1 (Donald Knuth)\\
unbehagliche Situation $x^2+1=0$, in $\mathbb{R}$ nicht lösbar

\subsubsection*{intuitiver Zugang}

neues Symbol $\imag = \sqrt{-1}$, imaginäre Einheit $\imag$

\begin{definition}[komplexe Zahlen]
  \begin{equation*} a, b \in \mathbb{R}: z = \underbrace{a}_\text{Realteil} + \imag \underbrace{b}_\text{Imaginärteil} \end{equation*}
  \begin{equation*} \mathbb{C} = \left\{ z : z = a+\imag b; a,b \in \mathbb{R}, \imag = \sqrt{-1} \right\} \end{equation*}
\end{definition}
%
\noindent Was ist neu in $\mathbb{C}$: Standardoperationen $\pm$ in $\mathbb{C}$ wie in $\mathbb{R}^2$; Multiplikation von $z_1$ mit $z_2$ ist anders definiert als in $\mathbb{R}^2$