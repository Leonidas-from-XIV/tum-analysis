% PREAMBLE

\documentclass[12pt]{scrreprt}

\usepackage[a4paper]{geometry}
\usepackage{graphicx,color}
\usepackage[numbers,sort&compress]{natbib}
\usepackage[utf8]{inputenc}
\usepackage[ngerman]{babel}
\usepackage{amsmath,amssymb}
\usepackage{amsthm,mathdots,gauss}

\usepackage{pxfonts}

\usepackage{hyperref,float,bbding,stmaryrd}

% amsthm conf
\newtheorem{definition}{Definition}
\newtheorem{theorem}{Satz}
\newtheorem{lemma}{Lemma}
\newtheorem{example}{Beispiel}
\newtheorem{note}{Anmerkung}
\newtheorem{proposition}{Behauptung}


% macros
\newcommand{\lecture}[1]{


  \noindent
  \fbox{
      \begin{minipage}{\textwidth}\centering
        \textbf{Vorlesung vom #1}
      \end{minipage}
  }
}

% Style
\renewcommand{\theenumi}{\arabic{enumi}}
\renewcommand{\labelenumi}{(\theenumi)}
\renewcommand{\theenumii}{\alph{enumii}}
\renewcommand{\labelenumii}{(\theenumii)}
\allowdisplaybreaks

% Meta
\title{Analysis für Informatiker}
\author{Lars Hupel, Michael Kerscher}
\date{\today}

% Lecturer shortcuts
\newcommand{\induction}{Beweis per vollständiger Induktion \Checkmark}
\newcommand{\imag}{\operatorname{i}}
\newcommand{\euler}{\operatorname{e}}