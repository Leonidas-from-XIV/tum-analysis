% PREAMBLE

\documentclass[12pt]{scrreprt}

\usepackage[a4paper]{geometry}
\usepackage{graphicx,color}
\usepackage[numbers,sort&compress]{natbib}
\usepackage[utf8]{inputenc}
\usepackage[ngerman]{babel}
\usepackage{amsmath,amssymb}
\usepackage{amsthm,mathdots,gauss}
\usepackage{capt-of}
\usepackage{todonotes,ifthen,wasysym,mflogo}

%\usepackage{pxfonts}

\usepackage{hyperref,float}

% tikz, geogebra colors
\usepackage{pgf,tikz}
\usetikzlibrary{arrows}
% geogebra farbcodes unschoen, aber zu aufwendig um die haendisch auszubessern -.-
\definecolor{qqqqff}{rgb}{0,0,1}
\definecolor{xdxdff}{rgb}{0.49,0.49,1}
\definecolor{uququq}{rgb}{0.25,0.25,0.25}
\definecolor{evtftf}{rgb}{0.9,0.25,0.25}
\definecolor{qqwuqq}{rgb}{0,0.39,0}
\definecolor{qqffqc}{rgb}{0,1,0.05}
\definecolor{ffffqq}{rgb}{1,1,0}
\definecolor{ffqqff}{rgb}{1,0,1}


% macros

\newcommand{\newtheoremenv}[3]{
  \newtheorem{#1_thm}{#2}
  \newenvironment{#1}{\begin{#1_thm}\normalfont}{\begin{flushright}#3\end{flushright}\end{#1_thm}}
}

\newcommand{\newtheoremenvstar}[4][default]{
  \newtheorem*{#2_thm}{#3}
  \newenvironment{#2}[1][]{\begin{#2_thm}[##1]\newtheoremselectfont{#1}}{\newtheoremstop{#4}\end{#2_thm}}
}

\newcommand{\newtheoremenvoldqed}{}

\newcommand{\newtheoremstop}[1]{
  \ifthenelse{\equal{#1}{}}{}{
    \renewcommand{\newtheoremenvoldqed}{\qedsymbol}
    \renewcommand{\qedsymbol}{#1}
    \qed
    \renewcommand{\qedsymbol}{\newtheoremenvoldqed}
  }
}

\newcommand{\newtheoremselectfont}[1]{
  \ifthenelse{\equal{#1}{default}}{}{}
  \ifthenelse{\equal{#1}{normal}}{\normalfont}{}
}

\newcommand{\flush}{\begin{flushleft}\end{flushleft}\noindent}

\newcommand{\lecture}[1]{


  \noindent
  \fbox{
      \begin{minipage}{\textwidth}\centering
        \textbf{Vorlesung vom #1}
      \end{minipage}
  }
}

% amsthm conf
\newtheorem{theorem}{Satz}
\newtheorem{lemma}{Lemma}
\newtheorem{proposition}{Behauptung}


% \newtheorem{definition}{Definition}

% \newtheorem{example}{Beispiel}
% \newtheorem{note}{Anmerkung}

\newtheoremenvstar[normal]{definition}{Definition}{$\LHD$}
\newtheoremenvstar[normal]{example}{Beispiel}{$\lhd$}
\newtheoremenvstar[normal]{note}{Anmerkung}{$\lhd$}


% Style
\renewcommand{\theenumi}{\arabic{enumi}}
\renewcommand{\labelenumi}{(\theenumi)}
\renewcommand{\theenumii}{\alph{enumii}}
\renewcommand{\labelenumii}{(\theenumii)}
\allowdisplaybreaks

% Meta
\title{Analysis für Informatiker}
\author{Lars Hupel\\ Michael Kerscher\\ Markus Grimm\\ Andreas Heider}
\date{\today}

% Lecturer shortcuts
\newcommand{\induction}{Beweis per vollständiger Induktion}
\newcommand{\imag}{\operatorname{i}}
\newcommand{\euler}{\operatorname{e}}
\newcommand{\equals}{\mathrel{\widehat{=}}}

\newcommand{\integer}{\texttt{INTEGER}}
\newcommand{\real}{\texttt{REAL}}

\newcommand{\annotation}[1]{\emph{#1, Anm. d. Verfasser}}

\renewcommand{\Re}{\ensuremath{\operatorname{Re}}}
\renewcommand{\Im}{\ensuremath{\operatorname{Im}}}