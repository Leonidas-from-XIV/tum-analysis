\lecture{2009-12-08}

\subsection{Regeln von L'Hospital}

unbestimmte Ausdrücke der Form $\frac 0 0$, $\frac \infty \infty$, $0 \cdot \infty$, $\ldots$ $\approx$ 1680 (Bernoulli)
\begin{example}\[ \liminfty[x]{\frac{\sin x} x} \]\end{example}

\noindent Hilfsmittel dazu: 2. Mittelwertsatz der Differentialrechnung
\begin{theorem}
  $f,g: \left[a,b\right] \to \mathbb R$ stetig und differenzierbar mit $g'(x) \neq 0$
  \[ g(a) \neq g(b) \implies \exists x_0: \frac{f(b)-f(a)}{g(b)-g(a)}=\frac{f'(x_0)}{g'(x_0)}  \]
\end{theorem}
\begin{note}
  nach 1. Mittelwertsatz
  \begin{equation*} \cfrac{\frac{f(b)-f(a)}{b-a}}{\frac{g(b)-g(a)}{b-a}} = \frac{f'(x_1)}{g'(x_2)} \end{equation*}
  mit $a < x_1, x_2 < b$, aber nicht $x_1 = x_2$
\end{note}
\begin{proof}
  Hilfsfunktion: $h(x) = f(x) - \frac{f(b)-f(a)}{g(b)-g(a)} \cdot (g(x)-g(a))$ stetig, differenzierbar
  \[
    \left.\begin{array}{r}h(a) = f(a)\\h(b)=f(a)\end{array}\right\} \leadsto \text{ Satz von Rolle: } \exists \textcolor{blue}{x_0}: h'(x_0) = 0
  \]
  \[ h'(x) = f'(x) - \frac{f(b)-f(a)}{g(b)-g(a)} \cdot g'(x) \]
  \[ \textcolor{blue}{x_0}: h'(x_0) = 0 = \frac{f(b)-f(a)}{g(b)-g(a)} \cdot g'(x_0) \]
\end{proof}

\begin{theorem}[Regeln von L'Hospital] Aus $f, g: \left]a,b\right[ \to \mathbb R$ differenzierbar mit $g'(x) \neq 0$, außerdem
  \begin{equation*} f(x), g(x) \xrightarrow[x \rightarrow a]{} 0 \text{ oder } f(x), g(x) \xrightarrow[x \rightarrow a]{} \infty \end{equation*}
  folgt:
  
  Falls der Grenzwert $\lim_{x \rightarrow a} \frac{f'(x)}{g'(x)}$ existiert, dann existiert auch der Grenzwert $\lim_{x \rightarrow a} \frac{f(x)}{g(x)}$, wobei
  \[ \lim_{x \rightarrow a} \frac{f(x)}{g(x)} = \lim_{x \rightarrow a} \frac{f'(x)}{g'(x)} \]
\end{theorem}
\begin{note}Nicht zu verwechseln mit der Quotientenregel! Zähler und Nenner werden getrennt voneinander differenziert.\end{note}
\begin{example}
  \begin{itemize}
    \item $\displaystyle\lim_{x \rightarrow 0} \underbrace{\left(\frac{\sin x} x\right)}_{\text{Form }\frac 0 0} =
      \lim_{x \rightarrow 0} \left(\frac{\cos x} 1\right) = 1$
    \item nützliche Umformungen
      \begin{footnotesize}\begin{align*}
        f(x) \cdot g(x) \text{ hat Form } 0 \cdot \infty &\implies f(x) \cdot g(x) = \frac{f(x)}{\frac 1 {g(x)}} \text{ hat Form } \frac 0 0 \\
        f(x) - g(x) \text{ hat Form } \infty - \infty &\implies f(x) \cdot g(x) \cdot \left( \frac 1 {f(x)} - \frac 1 {g(x)} \right) \text{ hat Form } \frac 0 0 \text { oder } \frac\infty\infty \\
      \end{align*}\end{footnotesize}
    \item
      \begin{align*}
        \lim_{x \rightarrow 0} \underbrace{\left( \frac 1 x - \frac 1 {\sin x} \right)}_{\text{Form } \infty - \infty}
        &= \lim_{x \rightarrow 0} \underbrace{\left( \frac{\sin x - x}{x \cdot \sin x} \right)}_{\text{Form } \frac 0 0} \\
        &= \lim_{x \rightarrow 0} \underbrace{\left( \frac{\cos x - 1}{x \cdot \cos x + \sin x} \right)}_{\text{Form } \frac 0 0} \\
        &= \lim_{x \rightarrow 0} \left( \frac{-\sin x}{2 \cdot \cos x - x\cdot \sin x} \right) \\
        &= \frac 0 2 = 0
      \end{align*}
  \end{itemize}
\end{example}

\subsubsection*{Nachtrag: Differentiation der $\euler$-Funktion $\euler^x$}

Einführung $\euler^x$ als Folgengrenzwert:
\begin{equation*}
  \euler^x = \liminfty{\left( 1+\frac x n\right)^n}
\end{equation*}
Definition $g(x) = \liminfty{\left( 1+\frac x n\right)^n}$, stetig differenzierbar
\begin{align*}
  \implies g'(x) &= \frac n n \cdot \left( 1+\frac x n\right)^{n-1} = \left( 1+\frac x n\right)^{n-1} \\
  &= \left( 1+\frac x n\right)^{-1}\cdot \left( 1+\frac x n\right)^{n-1} \\
  &= \left( 1+\frac x n\right)^{-1}\cdot g(x) \\
  &\xrightarrow[n \rightarrow \infty]{x: \text{ fest}} g(x)
\end{align*}

\begin{note}
  $\euler$-Funktion ist so gebaut, dass Ableitung $=$ Funktion $\leadsto$ gewöhnliche Differentialgleichung:
  \begin{align*}
    \dot y = y &\implies y(t) = c\cdot \euler^t \\
    \lambda \in \mathbb R\quad \dot y = \lambda y &\implies y(t) = c \cdot \euler^{\lambda t}
  \end{align*}
  (wegen $\frac\diff{\diff t} \euler^{h(t)} = h'(t)\cdot\euler^{h(t)}, h(t) = \lambda t$)

\end{note}

\subsection{Kurvendiskussion $y = f(x)$}

\begin{enumerate}
  \item Definitionsbereich, Wertebereich
  \item Symmetrien (gerade, ungerade)
  \item Stetigkeit, Polstellen
  \item Nullstellen (siehe \todo{Referenz auf 7.8 bzw. 2.3.8})
  \item $f'$ berechnen, $f' = 0$ Nullstellen
  \item Extremwerte: innere Extrema, Randextrema, Monotoniebereiche
  \item $f''$ berechnen, $f'' = 0$ Nullstellen
  \item Wendepunkte, konvexe/konkave Bereiche
  \item asymptotisches Verhalten $x \rightarrow \pm\infty$
  \item Graph
\end{enumerate}

\subsection{Gewöhnliche Differentialgleichungen -- 2. Teil}