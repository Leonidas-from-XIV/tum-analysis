\newpage
\lecture{2009-11-03}

\subsubsection*{Deutung für das Produkt zweier komplexer Zahlen}
\begin{align*}
	z &= |z|(\cos \varphi + \imag \sin \varphi) \\
	w &= |w|(\cos \psi + \imag \sin \psi) \\
	\Rightarrow zw &= |z||w|\left((\cos \varphi \cos \psi - \sin \varphi \sin \psi) + \imag (\cos \varphi \sin \psi + \sin\varphi \cos \psi)\right) \\
	\intertext{Aus den Additionstheoremen für trigonometrische Funktionen folgt:}
	&= |z||w|\left(\cos(\varphi + \psi) + \imag \sin(\varphi + \psi)\right)
\end{align*}
Produkt $z \cdot w$: Produkt der Beträge und Addition der Winkel
\begin{center}
	\begin{tikzpicture}[line cap=round,line join=round,>=triangle 45,x=1.0cm,y=1.0cm]
		\draw[->,color=black] (-0.5,0) -- (3.5,0);
		\foreach \x in {,1,2,3}
			\draw[shift={(\x,0)},color=black] (0pt,2pt) -- (0pt,-2pt);
		\draw[color=black] (3.98,0.08) node [anchor=south west] { Re};
		\draw[->,color=black] (0,-0.5) -- (0,4.5);
		\foreach \y in {,1,2,3,4}
			\draw[shift={(0,\y)},color=black] (2pt,0pt) -- (-2pt,0pt);
		\draw[color=black] (0.1,4.06) node [anchor=west] { Im};
		\fill [color=xdxdff] (0,0) circle (1.5pt);

		\draw [shift={(0,0)},color=qqwuqq,fill=qqwuqq,fill opacity=0.1] (0,0) -- (0:2.4) arc (0:15:2.4) -- cycle;
		\draw [->] (0,0) -- (2.32,0.62);
		\fill [color=qqqqff] (2.32,0.62) circle (1.5pt);
		\draw[color=qqqqff] (2.5,0.6) node {$z$};
		\draw[color=qqwuqq] (2.2,0.25) node {$\varphi$};

		\draw [shift={(0,0)},color=qqwuqq,fill=qqwuqq,fill opacity=0.1] (0,0) -- (0:2) arc (0:35:2) -- cycle;
		\draw [->] (0,0) -- (1.64,1.15);
		\fill [color=evtftf] (1.64,1.15) circle (1.5pt);
		\draw[color=evtftf] (1.8,1.3) node {$w$};
		\draw[color=qqwuqq] (1.6,0.75) node {$\psi$};
		
		\draw [shift={(0,0)},color=qqwuqq,fill=qqwuqq,fill opacity=0.1] (0,0) -- (0:1.6) arc (0:50:1.6) -- cycle;
		\draw [->] (0,0) -- (3.09,3.68);
		\fill [color=ffqqff] (3.09,3.68) circle (1.5pt);
		\draw[color=ffqqff] (4,3.7) node {$z + w$};
		\draw[color=qqwuqq] (1,0.35) node {$\varphi + \psi$};
	\end{tikzpicture}
\end{center}

\subsubsection*{Formel von Moivre (Potenzbildung)}
\begin{align*}
	z &= |z|(\cos \varphi + \imag \sin \varphi) \\
	z^2 &= |z|^2(\cos 2\varphi + \imag \sin 2\varphi) \\
	\vdots \\
	z^n &= |z|^n(\cos n\varphi + \imag \sin n\varphi) \\
	\text{für } |z| = 1 \Rightarrow z^n &= (\cos \varphi + \imag \sin \varphi)^n = \cos n\varphi + \imag \sin n\varphi \text{ (Formel von Moivre)}
\end{align*}
%
Rückwärts gelesen als "`Wurzel ziehen"'
\begin{align*}
	\left(\cos\frac{\psi}{n} + \imag \sin\frac{\psi}{n}\right)^n &= \cos \psi + \imag \sin \psi \\
	\intertext{speziell: $\psi = 2\pi$}
	\left(\cos\frac{2\pi}{n} + \imag \sin\frac{2\pi}{n}\right)^n &= \underbrace{\cos 2\pi}_{=1} + \underbrace{\imag \sin 2\pi}_{=0} = 1 \\
\end{align*}
(obiges ist eine Interpretation für $\sqrt[n]{1}$)

\begin{definition}[n-te Einheitswurzel $\omega_k$]
	\begin{align*}
		\omega_k &:= \cos\frac{2\pi k}{n} + \imag \sin\frac{2\pi k}{n} \\
		\text{mit } k &= 0, 1, 2, \ldots, n - 1 \\
		\Rightarrow \omega_k^n &= 1
	\end{align*}
	(mit $\cos \left(2\pi k\right) = 1 \text{ und } \sin \left(2\pi k\right) = 0$)
\end{definition}

\begin{example}
	\begin{equation*}
		\omega^6 = 1 \Rightarrow \omega = \sqrt[6]{1}
	\end{equation*}
	\begin{center}
		\begin{tikzpicture}[line cap=round,line join=round,>=triangle 45,x=2.0cm,y=2.0cm]
			\draw[->,color=black] (-1.25,0) -- (1.25,0);
			\foreach \x in {-1,1}
				\draw[shift={(\x,0)},color=black] (0pt,2pt) -- (0pt,-2pt);
			\draw[->,color=black] (0,-1.25) -- (0,1.25);
			\foreach \y in {-1,1}
				\draw[shift={(0,\y)},color=black] (2pt,0pt) -- (-2pt,0pt);
			\draw [shift={(0,0)},color=qqwuqq,fill=qqwuqq,fill opacity=0.1] (0,0) circle (1);
			
			\fill [color=xdxdff] (0.5,0.87) circle (1.5pt);
			\draw [color=xdxdff] (0,0) -- (0.5,0.87);
			\draw [color=xdxdff] (0.7,0.87) node {$\omega_1$};
			
			\fill [color=xdxdff] (-0.5,0.87) circle (1.5pt);
			\draw [color=xdxdff] (0,0) -- (-0.5,0.87);
			\draw [color=xdxdff] (-0.7,0.87) node {$\omega_2$};
			
			\fill [color=xdxdff] (-1,0) circle (1.5pt);
			\draw [color=xdxdff] (0,0) -- (-1,0);
			\draw [color=xdxdff] (-1.2,0.1) node {$\omega_3$};
			
			\fill [color=xdxdff] (-0.5,-0.87) circle (1.5pt);
			\draw [color=xdxdff] (0,0) -- (-0.5,-0.87);
			\draw [color=xdxdff] (-0.7,-0.87) node {$\omega_4$};
			
			\fill [color=xdxdff] (0.5,-0.87) circle (1.5pt);
			\draw [color=xdxdff] (0,0) -- (0.5,-0.87);
			\draw [color=xdxdff] (0.7,-0.87) node {$\omega_5$};
			
			\fill [color=xdxdff] (1,0) circle (1.5pt);
			\draw [color=xdxdff] (0,0) -- (1,0);
			\draw [color=xdxdff] (1.2,0.1) node {$\omega_6$};
		\end{tikzpicture}
	\end{center}
	$n = 6$ d.h. teile Einheitskreis (mit Unfang $2\pi$) in 6 Teile. 1 Teil $\equals \frac{2\pi}{6} = \frac\pi 3 (60^\circ)$ (hier: $k = 1 \ldots n$ (statt $0 \ldots n - 1$))
\end{example}
%
\noindent Ohne Beschränkung auf Länge 1:
\begin{gather*}
	a = |a|(\cos \alpha + \imag \sin \alpha) \in \mathbb{C} \\
	\Rightarrow z_n = \sqrt[n]{|a|}\left(\cos \frac{\alpha + 2\pi k}{n} + \imag \sin \frac{\alpha + 2\pi k}{n}\right) \text{ mit } k = 0\dotsc n - 1 \text{ $\ldots$"`$n$-Wurzeln"'}
\end{gather*}
\begin{note}
	Ist das reelle $\sqrt{}$-ziehen eingebettet?
	\begin{align*}
		a \in \mathbb{R} &\Rightarrow \alpha = 0 \Rightarrow \\
		&\Rightarrow \left\{\begin{aligned}
			\cos 0 &= 1 \text{, } & \sin 0 &= 0 && (k &= 0) \\
			\cos \pi &= -1 \text{, } & \sin \pi &= 0 && (k &= 1)
		\end{aligned}\right. \\
		\text{2 Lösungen: } x_1 &= \sqrt{|a|} \cdot 1 \\
		x_2 &= \sqrt{|a|} \cdot (-1)
	\end{align*}
\end{note}
\begin{note}[Euler-Formel]
	\begin{equation*}\euler^{\imag \varphi} =\cos \varphi + \imag \sin \varphi\end{equation*}(Basis für die Fast-Fourier-Transformation)
\end{note}
\begin{note}Interessante Gleichung: $\euler^{2\pi\imag} = 1$\end{note}

\section{Elementares aus $\mathbb{R}^2$ bzw. $\mathbb{R}^n$}

\subsection{Kartesisches Koordinatensystem}
\begin{definition}[Graph einer Funktion]
	\begin{align*}
		f &: I \rightarrow \mathbb{R} \text{ mit }I \subseteq \mathbb{R} \\
		G &: \{(x,y) | x \in I, y = f(x)\} \\
		& \text{($I$: Definitionsbereich, $f(I)$: Wertebereich)}
	\end{align*}
\end{definition}

\subsubsection*{Kurven}

\begin{definition}[Kurve in $\mathbb{R}$, erweiterter Funktionsbegriff]\flush
      \begin{gather*}
              C := \{(x, y) \in \mathbb{R}^2 \;|\; F(x, y) = 0\} \text{ (implizite Darstllung)} \\
              \text{Achtung: keine Form } y = f(x) \text{ erforderlich}
      \end{gather*}
\end{definition}
\begin{example}[Kreislinie]
        \[ x^2 + y^2 = r^2; \;\; F(x,y) = x^2 + y^2 - r^2 = 0 \]
\end{example}
\begin{note}[Parameterdarstellung]\flush
  \begin{itemize}
    \item
          \TeX-\MF: Kurve als Parameterdarstellung
          \begin{equation*}
                  \text{Parameter }t \in I : C(t) = \left(\begin{aligned}x(t) \\ y(t)\end{aligned}\right)
          \end{equation*}
    \item falls $t = x$: siehe oben, entspricht Funktion
    \item andere Kreisdarstellung:
      \begin{equation*}
        0 \leq t <2\pi: \left.\begin{aligned}x(t) &= r \cos t \\ y(t) &= r \sin t\end{aligned}\right\}
      \end{equation*}
    \item Wahl des Parameters $t$: nahezu "`Lottospiel"' (d. h. beliebig). Beste Wahl: $t = s$ (Bogenlänge; ist im Allgemeinen ein implizites Problem.)
  \end{itemize}
  \todo{hier fehlt noch etwas Struktur}
\end{note}

\begin{repetition}[Trigonometrische Funktionen]
\begin{center}
	\begin{tikzpicture}[line cap=round,line join=round,>=triangle 45,x=2.0cm,y=2.0cm]
		\draw[color=black] (0,0) -- (3,0);
		\draw[color=black] (0,0) -- (2.01,1.4);
		\draw[color=black] (3,0) -- (2.01,1.4);
		
		\draw [color=black] (-0.2,0) node {$A$};
		\draw [color=black] (3.2,0) node {$B$};
		\draw [color=black] (2.01,1.6) node {$C$};
		\draw [color=black] (1.5,-0.2) node {$c$};
		\draw [color=black] (0.8,0.8) node {$b$};
		\draw [color=black] (2.7,0.8) node {$a$};
		
		\draw [color=black] (1,0) arc (0:35:1);
		\draw [color=black] (0.55,0.15) node {$\alpha$};
	\end{tikzpicture}
\end{center}
Im rechtwinkligen Dreieck gilt:
\begin{itemize}
	\item $\sin \alpha = \frac a c$
	\item $\cos \alpha = \frac b c$
	\item $a^2 + b^2 = c^2$
	\item falls $c = 1 \Rightarrow \text{Bogenmaß}$
\end{itemize}

\begin{center}
	\begin{tabular}[h]{|c||c|c|c|c|c|}
		\hline
		$\alpha$ & $0$ & $\frac \pi 6$ & $\frac \pi 4$ & $\frac \pi 3$ & $\frac \pi 2$ \\ \hline
		$\cos \alpha$ & $1$ & $\frac 1 2 \sqrt{3}$ & $\frac 1 2 \sqrt{2}$ & $\frac 1 2$ & $0$ \\ \hline
		$\sin \alpha$ & $0$ & $\frac 1 2$ & $\frac 1 2 \sqrt{2}$ & $\frac 1 2 \sqrt{3}$ & $1$ \\ \hline
	\end{tabular}
	\captionof{table}{Typische Werte}
\end{center}

\subsubsection*{Graphen}
\annotation{die Graphen von Sinus und Kosinus sollten bekannt sein}

\subsubsection*{Eigenschaften}
\begin{itemize}
	\item $\sin x$, $\cos x$ sind $2\pi$-periodisch, d.h. $\sin(x + 2\pi k) = \sin x$; $\cos(x + 2\pi k) = \cos x$ mit $k \in \mathbb{Z}$
	\item $\cos x$ ist eine gerade Funktion: $\cos(-\alpha) = \cos\alpha$
	\item $\sin x$ ist eine ungerade Funktion: $\sin(-\alpha) = -\sin\alpha$
\end{itemize}

\end{repetition}

\subsection{Vektorrechnung im $\mathbb{R}^2$ (Winkel, Längen)}
\begin{definition}[Vektor]
	\[\vec{x} = \begin{pmatrix}x_1 \\ x_2\end{pmatrix}\]
\end{definition}
\noindent Die folgenden Darstellungen sind äquivalent (weil eine Bijektion existiert):
\begin{itemize}
	\item Zahlenpaar im $\mathbb{R}^2$
	\item Punkt im $\mathbb{R}^2$
	\item Vektor im $\mathbb{R}^2$
	\item Pfeil im $\mathbb{R}^2$
\end{itemize}
%
Standardvektorraum ist der $\mathbb{R}^2$ bzw. $\mathbb{R}^n$ über dem Zahlenkörper $\mathbb{R}$.

\subsubsection*{Rechnen}
\begin{itemize}
	\item Addition zweier Vektoren $\vec{x} + \vec{y} = \begin{pmatrix}x_1 \\ x_2\end{pmatrix} + \begin{pmatrix}y_1 \\ y_2\end{pmatrix} = \begin{pmatrix}x_1 + y_1 \\ x_2 + y_2\end{pmatrix}$
	\item skalare Multiplikation ($\lambda \in \mathbb{R}$): $\lambda \vec{x} = \lambda \begin{pmatrix}x_1 \\ x_2\end{pmatrix} = \begin{pmatrix}\lambda x_1 \\ \lambda x_2\end{pmatrix}$
\end{itemize}
\begin{note}
	Drehen von Vektoren $\Rightarrow$ Matrizeneinführung $Q = \begin{pmatrix}\cos \alpha & \sin \alpha \\ -\sin \alpha & \cos \alpha\end{pmatrix}$
\end{note}
