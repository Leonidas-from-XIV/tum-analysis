\lecture{2010-02-02}

\begin{theorem}[Eigenschaften des Riemann-Integrals]
  Seien $f$ und $g$ integrierbar über $[a,b]$.
  \begin{enumerate}
    \item Additivität \[ \int\limits_a^b f(x)\,\diff x = \int\limits_a^c f(x)\,\diff x + \int\limits_c^b f(x)\, \diff x \]
    \item Verschiebung \[ \int\limits_a^b f(x)\,\diff x = \int\limits_{a+c}^{b+c} f(x-c)\,\diff x \]
    \item Streckung  \[ \int\limits_a^b f(x)\,\diff x = \frac 1 k \int\limits_{ka}^{kb} f\left(\frac x k\right)\,\diff x \]
    \item Orientierung \[ \int\limits_a^b f(x)\,\diff x = - \int\limits_b^a f(x)\,\diff x \]
    \item \emph{lineares Funktional}\todo{umstruk\-turieren}\ $\int: C^0[a,b] \to \mathbb{R}$
      \[ \int\limits_a^b (f+g)(x)\,\diff x = \int\limits_a^b f(x)\,\diff x + \int\limits_a^b g(x)\,\diff x \]
      \[ \lambda \in \mathbb{R}: \quad \int\limits_a^b \lambda f(x)\,\diff x  = \lambda \int\limits_a^b f(x)\,\diff x \]
    \item $(f \cdot g)$ (Produkt) ist integrierbar
    \item $\left| f \right|$ ist integrierbar mit
      \[ \left| \int\limits_a^b f(x)\,\diff x\right| = \int\limits_a^b \left| f(x) \right| \,\diff x \]
    \item $f(x) \geq 0$ auf $[a,b]$ $\implies$ $\ds \int\limits_a^b f(x)\,\diff x \geq 0$
    \item $f(x) \geq g(x)$ auf $[a,b]$ $\implies$ $\ds \int\limits_a^b f(x)\,\diff x \geq \int\limits_a^b g(x)\,\diff x$
    \item $m \leq f(x) \leq M$ auf $[a,b]$ $\implies$ $\ds m(b-a) \leq \int\limits_a^b f(x)\,\diff x \leq M(b-a)$
  \end{enumerate}
\end{theorem}

\subsection{Stammfunktion, Hauptsatz}

Ziel: Zusammenhang zwischen Differential- und Integralrechnung\\
Hinweis: $\int\limits_0^b x^n\,\diff x = \frac{b^{n+1}}{n+1} \leadsto$ Ableitung $x^n$

\begin{definition}[Stammfunktion $F$]
  Sei $f: [a,b] \to \mathbb{R}$ integrierbar (damit beschränkt). Dann ist die Stammfunktion
  \[ F(x) = \int\limits_a^x f(t)\,\diff t \]
\end{definition}

\begin{theorem}[Hauptsatz der Differential- und Integralrechnung]
  \begin{enumerate}
    \item $F$ ist stetig
    \item $f$ stetig in $x_0$ $\implies$ $F$ differenzierbar in $x_0$ mit $F'(x_0) = f(x_0)$
    \item $\ds \int\limits_a^b f(t)\,\diff t = F(x)\big|_a^b = F(b) - F(a)$
  \end{enumerate}
\end{theorem}

\begin{proof}
  \begin{enumerate}
    \item $f$ integrierbar $\implies$ $f$ beschränkt, d. h. es existiert ein $L \in \mathbb R$ mit $\left|f(x)\right| \leq L$ in $a\leq x \leq b$
      \begin{align*}
        \left| F(x) - F(y) \right| &= \left| \int\limits_a^y f(t)\,\diff t - \int\limits_a^x f(t)\,\diff t \right|\\
        &= \left| \int\limits_x^y f(t)\,\diff t \right |\\
        &\leq \int\limits_x^y \left| f(t) \right| \,\diff t\\
        &\leq \int\limits_x^y L\,\diff t\\
        &= L\cdot (y-x)
      \end{align*}
      Sei nun $\delta = y-x$. Wähle $\varepsilon = \frac \delta L$, dann folgt mittels Satz \ref{thm:delta_epsilon} (Seite \pageref{thm:delta_epsilon}) die Behauptung, da:
      \[ \left| y - x \right| < \delta \implies \left| F(y) - F(x) \right| < \varepsilon \]
      Ergebnis: $F$ ist stetig
      \todo[inline]{der vorletzte Schritt war versehen mit "`da $y \geq x$ bzw. $y \leq x$"' -- prüfen}
    \item zu zeigen: $F'(x) = f(x)$\\
      Differentialquotient: $\ds F'(x)= \lim_{x \to x_0}\frac{F(x)-F(x_0)}{x-x_0}$\\
      bekannt: $f(x)$ stetig
      \[ \forall \varepsilon > 0 \exists \delta > 0 \forall x: \left| x - x_0 \right| < \delta \implies \left| f(x) - f(x_0) \right| < \varepsilon \]
      \begin{align*}
        \left| \frac{F(x)-F(x_0)}{x-x_0} - f(x_0) \right|
        &= \left| \frac{1}{x-x_0} \int\limits_{x_0}^x f(t)\,\diff t - \textcolor{blue}{f(x_0)} \right|\\
        &= \left| \frac{1}{x-x_0} \int\limits_{x_0}^x f(t)\,\diff t - \textcolor{blue}{\frac 1 {x-x_0} \int\limits_{x_0}^x f(x_0)\,\diff t} \right| \\
        &= \Biggl| \frac{1}{x-x_0} \int\limits_{x_0}^x \underbrace{f(t) - f(x_0)}_{< \varepsilon} \,\diff t \Biggr| \quad\text{mit }x>x_0\\
        &\leq \frac{1}{x-x_0} \cdot \varepsilon (x-x_0) = \varepsilon
      \end{align*}
  \end{enumerate}
\end{proof}

\begin{note}
  \begin{itemize}
    \item Integrale können jetzt ohne Ober-/Untersummen beschrieben werden
    \item Integrale können als "`Umkehrung"' der Differentiation aufgefasst werden $\leadsto$ Suche nach Stammfunktion
  \end{itemize}
\end{note}

\subsection{Separable Differentialgleichungen}

DGL vom Typ $y'(x) = f(x)g(y)$ -- $x$, $y$ treten separiert auf

\subsubsection*{Separationsansatz}

\[ y'(x) = \frac{\diff y}{\diff x} = f(x) g(y) \implies \frac{\diff y}{g(y)} = f(x)\diff x\]
\[ \int\limits_{y_0}^y \frac{\diff s}{g(s)} = \int\limits_{x_0}^{x} f(t)\,\diff t \quad\text{AW } y(x_0) = y_0 \]

\begin{note}
  Die Lösung $y$ wird eventuell nur implizit oder als $x(y)$ angegeben.
\end{note}

\begin{example}[Modell des Bevölkerungswachstums der Erde]
  \begin{itemize}
    \item $y(t)$ Anzahl der Menschen auf der Erde zur Zeit $t$
    \item jährlicher relativer Bevölkerungszuwachs werde durch $c(t,y)$ modelliert
    \item einfache DGL $\dot y(t) = c(t,y)y(t)$
  \end{itemize}
  Aufgabe: lege $c(t,y)$ fest
  \begin{itemize}
    \item 1969 ("`Club of Rome"'): $c(t,y) = c = 0,02$ (konstant)\\
      aufrüttelnde Prognose, aber zu grob
    \item maximale Anzahl an Menschen, die unter würdigen Bedingungen leben können: $N \approx 18 \text{ Mrd.}$\\
      $c(t,y) = \alpha (N-y)^k$ mit $k = 0, 1, 2$

