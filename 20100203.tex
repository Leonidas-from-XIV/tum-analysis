\lecture{2010-02-03}
% Fortführung der example-Umgebung

\minisec{Umskalierung}
keine absoluten Zahlen, sondern Vielfache des Anfangswerts

\begin{align*}
	y(t) &= y_0 \cdot \alpha (t) \\
	N &= \beta \cdot y_0 \qquad\text{(mit $\beta \approx 5$)} \\
	\implies \dot u &= c_0 \left( \frac{\beta - u}{\beta - 1} \right)^k u \\
	\text{denn}\quad \dot y &= c y = \alpha (N - y)^k \cdot y \\
	&= c_0 \cdot \frac{(N - y)^k}{(N - y_0)^k} \cdot y \\
	&= c_0 \cdot \frac{(\beta y_0 - y_0 u)^k}{(\beta y_0 - y_0)^k} \cdot y_0 u \\
	y_0 \dot u &= c_0 \cdot \frac{ (\beta - u)^k }{ ( \beta - 1 )^k } \cdot y_0 u \\
	\dot u &= c_0 \cdot \frac{ (\beta - u)^k }{ ( \beta - 1 )^k } \cdot u
\end{align*}
\minisec{Diskussion der DGL}
$\dot u$ ist separabel
\begin{align*}
\dot u(t) &= c_0 \left( \frac{ \beta - u }{\beta - 1} \right)^k u \\
u(0) &= 1 \qquad\text{(Anfangswert von 1969)} \\
c_0 &= 0,02 \\
\beta &= 5
\end{align*}

\begin{description}
  \item[1. Spiel: $ k = 0 $]
\begin{equation*}
	\dot u = 0,02 \cdot u \implies u(t) = e^{0,02 \cdot t}
\end{equation*}
zu "`primitiv"'
\begin{note}lineare Modelle sind für Prognosen untauglich\end{note}

  \item[2. Spiel: $ k = 1 $]
\begin{equation*}
	\dot u = c_0 \cdot \frac{\beta - u}{\beta - 1} \cdot u
\end{equation*}
nicht linear $\leadsto$ Kandidat für Prognose

\begin{align*}
(\beta - 1) \cdot\frac{\diff u}{(\beta - u) u} &= c_0 \diff t \\
(\beta - 1) \int\limits_1^u \underbrace{\frac{\diff s}{(\beta - s) s}}_{= \frac 1 \beta \left( \frac 1 {\beta - s} + \frac 1 s \right)} &= \int\limits_0^t c_0 \diff x \\
\frac{\beta - 1}{\beta} \left( \int\limits_1^u \frac{\diff s}{\beta - s} + \int\limits_1^u \frac{\diff s}{s}\right) &= c_0 \int\limits_0^t \diff x \\
\frac{\beta - 1}{\beta} \cdot \left( \ln(\beta - s) \big\vert_1^u + \ln s\big\vert_1^u \right) &= c_0 \left( x\big\vert_0^t \right) \\
\frac{\beta - 1}{\beta} \cdot \left( \ln(\beta-n) - \ln(\beta -1) + \ln u - \ln 1 \right) &= c_0 t \\
\frac{\beta - 1}{\beta} \cdot \ln \left( \frac{u (\beta -u) }{\beta - 1} \right) &= c_0 t
\end{align*}

implizite Lösung der DGL

  \item[3. Spiel: $ k = 2 $] (Selbststudium)

\end{description}

% Wird vor der Klausur nichts mehr mit den Plots. --lars
% \missingfigure{Plots}
\end{example}


\subsection{Integrationstechniken}
\emph{Ziel:} Zusammenstellung von Techniken um eine Stammfunktion zu berechnen \\
Hilfsmittel: Umformung von Differentiationsregeln

\begin{enumerate}
	\item Integranden der Form $\frac{f'}{f}$ \\
	\[
		f: [a, b] \quad\text{stetig differenzierbar, } f \neq 0
	\]
	\[
		\int \frac{f'}{f} \,\diff x = \ln \left|f(x)\right| + c
	\]
	
	\begin{example}
		\begin{align*}
			\int \tan x \,\diff x &= \int \frac{\sin x}{\cos x} \,\diff x \\
			&= - \ln \left|\cos x\right| + c \\
			&= \ln \left|\frac{1}{\cos x}\right| + c \\
			\text{wobei } (\cos x)' &= - \sin x
		\end{align*}
	\end{example}
	
	\item partielle Integration (aus Produktregel)
	\begin{align*}
		\frac{\diff}{\diff x} (f g) &= f' g + f g' \\
		\leadsto \int f'(x) g(x) \,\diff x &= f(x) g(x) - \int f(x) g'(x) \,\diff x
	\end{align*}
	
	\begin{example}
		\begin{align*}
			\int \underbrace{x}_g \underbrace{\sin x}_{f'} \,\diff x &= \underbrace{x}_g \underbrace{(- \cos x)}_f - \int \underbrace{1}_{g'} \underbrace{(- \cos x)}_f \,\diff x \\
			&= -x \cos x + \int \cos x \,\diff x \\
			&= -x \cos x + \sin x + c
		\end{align*}
	\end{example}
	
	\item Substitutionsregel
		\begin{align*}
			g: [a, b] \to \mathbb{R} \text{ stetig differenzierbar} \\
			f: g(I) \to \mathbb{R} \text{ stetig differenzierbar}
		\end{align*}
		
		Stammfunktion: $F(y) = \int f(y) \,\diff y$
		
		\begin{equation*}
			\left[ \int f(g(x)) g'(x) \,\diff x = F(g(x)) \right]
		\end{equation*}
		
		handlicher:
		\begin{equation*}
			\int f(x) \,\diff x = \int f(g(u)) g'(u) \,\diff u
		\end{equation*}	
		
		Merkregel:
		\begin{equation*}
			\int f(g(x)) g'(x) \,\diff x = \int f(g) \frac{\diff g}{\diff x} \,\diff x = \int f(g) \,\diff g
		\end{equation*}
		
		\begin{example}
			Streckung $\int\limits_a^b f(x) \,\diff x$ mit neuer Variablen $t = k x$
			\begin{align*}
				x = a &\implies t = ka \\
				x = b &\implies t = kb
			\end{align*}
			\begin{equation*}
				\int\limits_a^b f(x) \,\diff x = \int\limits_{ka}^{kb} f(t) \frac{\diff t}{k} = \frac{1}{k} \int\limits_{ka}^{kb} f(t) \,\diff t
			\end{equation*}
		\end{example}
		
	\item Spezialfälle mit Integranden $ \cfrac{1}{1\pm x^2}$, $\cfrac{1}{\sqrt{1-x^2}}$ usw. \\
		führen zu Lösungen mit Arcus-, Area-Funktionen
		
	\item rationale Nenner: Partialbruchzerlegung \\
		in diesen Fällen: Formelsammlungen
\end{enumerate}
		
\subsection{Restglieder: Taylorformel}
\label{sec:integr_taylor}
\emph{Hilfaussage:} Mittelwertsatz der Integralrechnung
\begin{equation*}
	f, g \text{ stetig auf } [a, b], g(x) \geq 0
\end{equation*}
$\implies$ es existiert mindestens eine Zwischenstelle $\xi$
\begin{equation*}
	\int\limits_a^b f(x) g(x) \,\diff x = f(\xi) \int\limits_a^b g(x) \,\diff x
\end{equation*}
%
\emph{Beweisidee:} $f$ stetig $\implies$ es existieren:
\[ m, M: m \leq f \leq M \]
\[ g \geq 0: mg \leq fg \leq Mg \]
\[ m \int g \,\diff x \leq f \int g \,\diff x \leq M \int g \,\diff x \]
es existiert $c = f(\xi)$:
\[ \int\limits_a^b fg \,\diff x = c \int_a^b g \,\diff x \quad\text{(Zwischenwertsatz)} \]
	
\begin{align*}
	f&: [a, b] \to \mathbb{R} \quad\text{$(n+1)$-fach stetig differenzierbar} \\
	x_0&: \text{Entwicklungspunkt}
\end{align*}

Behauptung:
\begin{equation*}
	f(x) = \underbrace{\sum_{\nu = 0}^n \frac{f^{(\nu)} (x_0)}{\nu !} (x - x_0)^{\nu}}_{T_n(x)} + \underbrace{\frac{1}{n!} \int\limits_{x_0}^x (x - t)^n f^{(n+1)} (t) \,\diff t}_{R_{n+1}(x)}
\end{equation*}

\begin{proof}
  \begin{itemize}
    \item $n=0$:

\begin{equation*}
	f(x) = f(x_0) + \int\limits_{x_0}^x 1 f'(t) \,\diff t = f(x_0) + f(t) \big|_{x_0}^x = f(x)
\end{equation*}

    \item $n-1 \leadsto n$:
\begin{align*}
	R_n(x) &= \frac{1}{(n-1)!} \int\limits_{x_0}^x \underbrace{(x-t)^{n-1}}_{u'} \underbrace{f^{(n)}(t)}_{v} \,\diff t\\
	&= \frac{1}{(n-1)!} \left( \underbrace{-\frac{(x-t)^n}{n}}_u \underbrace{f^{(n)}(t)}_v\Big|_{x_0}^x  - \int\limits_{x_0}^x \underbrace{-\frac{(x-t)^n}{n}}_u \underbrace{f^{(n+1)}(t)}_{v'} \,\diff t \right) \\
	&= - \frac{(x-t)^n}{n!} f^{(n)} (t)\Big|_{x_0}^x + \underbrace{\frac{1}{n!} \int_x (x-t)^n f^{(n+1)} (t) \,\diff t}_{R_{n+1}}\\
	&= \frac{(x-x_0)^n}{n!} f^{(n)} (x_0) + R_{n+1}(x)\\
	f(x) &= T_{n-1}(x) + \frac{(x-x_0)^n}{n!} f^{(n)} (x_0) + R_{n+1}(x) \\
        &= T_n(x) + R_{n+1}(x)
\end{align*}

  \end{itemize}

\end{proof}

\begin{note}
	Restglied nach Lagrange: MWS der Integralrechnung
	\begin{align*}
		R_{n+1} (x) &= \frac{1}{n!} \int\limits_{x_0}^x \underbrace{(x-t)^n}_{\geq 0} f^{(n+1)} (t) \,\diff t \\
		&= \frac{f^{(n+1)}(\xi)}{n!} - \int_{x_0}^x (x-t)^n \,\diff t \\
		&= f^{(n+1)} (\xi) (-1) \frac{(x-t)^n+1}{(n+1)!}\Bigg|_{x_0}^x \\
		&= f^{(n+1)} (\xi) \frac{(x-x_0)^{n+1}}{(n+1)!}
	\end{align*}
\end{note}






















