\chapter{Grundlagen: Zahlbegriff}

\lecture{2009-10-20}


\section{Zahldarstellung}
\subsection{Natürliche Zahlen zu reelle Zahlen}
diskret: $1,2,3,...$\\
$\mathbb{N} = \{1,2,3,...\}$ Menge der natürlichen Zahlen
\begin{definition}
 Menge ist Zusammenfassung bestimmter wohlunterschiedener Objekte unserer Anschauung oder unserers Denkens
\end{definition}

\subsubsection*{Axiomensystem nach Peano}
\begin{enumerate}
 \item $1 \in \mathbb{N}$ (Anfang)
 \item $n \in \mathbb{N} \Rightarrow (n+1) \in \mathbb{N}$ (Nachfolger)
 \item $n \neq m \Rightarrow (n+1) \neq (m+1)$
 \item $n \in \mathbb{N} \Rightarrow (n+1) \neq 1$
 \item $A \in \mathbb{N}: 1 \in A \land (\forall n: n \in A \Rightarrow (n+1) \in A) \Rightarrow A = \mathbb{N}$ (Vollständigkeitsaxiom, alle natürlichen Zahlen werden erfasst)
\end{enumerate}

\subsubsection*{Erweiterungen}

\begin{enumerate}
 \item ...zu $\mathbb{Z} = \{\ldots,-2,-1,0,1,2,\ldots\}$ ganze Zahlen. In $\mathbb{Z}$ Operationen $+,-$
 \item ...zu $\mathbb{Q}$ rationale Zahlen durch $*,/$ $q=\frac{m}{n}; m \in \mathbb{Z}, n \in \mathbb{N}$
  \begin{equation*}\mathbb{Q} = \left\{x | x=\frac{m}{n}, m \in \mathbb{Z}, n \in \mathbb{N} \right\}\end{equation*}
  $\mathbb{Q}$ ist dicht, d.h. zwischen $q_1, q_2$ liegt ein $\tilde{q}$
 \item ...durch $\sqrt{}$ (Wurzelziehen) bzw. Quadrieren $x^2=a$
 \begin{equation*}a=2 \Rightarrow x = \sqrt{a} \notin \mathbb{Q}, \sqrt{2}=1,4142\ldots\text{ irrational}\end{equation*}

\begin{proof}[Beweis (indirekt)]
Aus $\sqrt{2}$ wäre Element in Q. $\sqrt{2}=\frac{p}{q}$ gekürzt ($p \in \mathbb{Z}, g \in \mathbb{N}$
$2g^2=p^2 \Rightarrow p^2$ gerade $\Rightarrow p=2\hat{p}$ und $2g^2=4\hat{p}^2$
$\Rightarrow g^2 = 2\hat{p}^2 \Rightarrow g=2\hat{g}$ 
$\Rightarrow$ Widerspruch zur gekürzten Form: $\sqrt{2}=\frac{p}{q}=\frac{2\hat{p}}{2\hat{g}}$

Aussage: $\sqrt{2}$ ist keine rationale Zahl.
\end{proof}

Beweistechnik war indirekt, z. z. Aussage $A$ $\Rightarrow$ Aussage $B$\\
indirekt: "`nicht"' Aussage $B$ $\Rightarrow$ "`nicht"' Aussage $A$


\begin{proof}[Beispiel: direkte Beweistechnik]\todo{Beispiel prüfen}

$p \in \mathbb{N}$ gerade $\Leftrightarrow p \Leftarrow 2 \hat{p} \Leftrightarrow p^2 = 4\hat{p}^2$ gerade\\
$p \in \mathbb{N}$ ungerade $\Leftrightarrow p = 2 \hat{p}+1 \Leftrightarrow p^2 = (2\hat{p}+1)^2=4\hat{p}^2+4\hat{p}+1$ ungerade\\
$\Rightarrow$ reelle Zahlen, formaler Weg siehe \cite[S. 9ff]{bornemann}
\end{proof}

 \item reelle Zahlen -- neue Kandidaten im Vergleich zu $\mathbb{Q}$
\begin{itemize}
 \item $\sqrt{}$-Bildung
 \item $c = 0,\overline{b_1 \;\ldots\; b_k}$ periodische Zahl\\
$c = \cfrac{b_1 \;\ldots\; b_k}{\underbrace{g \;\ldots\; g}_{k\text{-mal}}} \in \mathbb{Q} \Rightarrow 10^k c -c = b_1 \ldots b_k$ periodischer Bruch\\
(eine neue periodische Zahl, die nicht als Bruch darstellbar ist, wäre z. B. $c=0,101001000100001\ldots$)
 \item $\infty$-Summen, $\euler, \pi,\ldots$
\end{itemize}

\end{enumerate}

\subsection{Maschinenzahlen $\mathbb{M}$}
\integer-Zahlen (Assoziation: $\mathbb{Z}$), \real-Zahlen (Assoziation: $\mathbb{R})$ sind 2 Typen unterschiedlicher Codierung

4 Byte für \integer; 4, 8, 10 Byte für \real

\subsubsection*{\integer}

31 Bits für Mantisse, größte Zahl $\pm \underbrace{1\ldots1}_{31\text{ Mal}}$ (im Zweiersystem)

entspricht Zahldarstellung: $2^0+2^1+\ldots+2^{30}=2^{31}-1$ \\
$\leadsto$ 10er-System: $2^{31}-1 \equals x$, $2^{10}\approx 10^3$\\
Zahlbereich: -2 Mrd. bis 2 Mrd

\subsubsection*{\real-4}

Mantisse 23 Bits, Exponent 7 Bits

$\pm 0.\underbrace{\text{\ttfamily\_\_\_\_\_}\ldots\text{\ttfamily\_\_\_\_\_}}_{\text{Mantisse}}e\pm\underbrace{\text{\ttfamily\_\_\_}\ldots\text{\ttfamily\_\_\_}}_{\text{Exponent}}$

Darstellung ist \emph{normalisiert}, d. h. nach Dezimalpunkt keine Nullen (IEEE 754 $\hat{=}$ VDE)

\begin{itemize}
 \item Exponentenspielraum: $1\cdot 2^0+1\cdot2^1+\ldots+1\cdot2^6 = 2^7 -1 = 127$
  Exponent $10^{-127}$ bis $10^{127}$
 \item Länge der Maschine im 10er-System:\\
$1\cdot 2^0+1\cdot2^1+\ldots+1\cdot2^22=2^{23} \equals 10^x$\\
$2^{23}\equals 10^x, x=23 \log_{10}(2) \approx 6,923 $ \todo{soll das hier stehen?}\\
$m \in \mathbb{M}, m=\pm 0.\text{\ttfamily\_\_\_\_\_}\ldots\text{\ttfamily\_\_\_\_\_} e\pm \text{\ttfamily\_\_\_}\ldots\text{\ttfamily\_\_\_}$\\Lücke zwischen $-10^{-127}$ und $10^{-127}$ Faktor $10^7$ groß
\end{itemize}

\subsubsection{Rundungsfehler}

Wesentlich mitbestimmt von F. L. Bauer, Samelson, Zenger aus der Informatik und R. Bukisch und Chr. Reinsch aus der Mathematik und Wilkinson.\\
Hilfsmittel: Abbildung von den reellen Zahlen (Alltag) auf den Rechner

\begin{definition}[Abbildung \emph{rd} bzw. \emph{round} (Rundung)]
rd: $\mathbb{R}\rightarrow\mathbb{M}$\\
rd: $\left\{x|x \in [\frac{m_{i-1}+m_i}{2},\frac{m_{i}+m_{i+1}}{2}[\right\} \mapsto m_i$
\end{definition}

\begin{note}
Intervall-Arithmetik hat sich trotz Hardwareunterstützung nicht bewährt: Intervallängen zu pessimistisch.
\end{note}

\begin{definition}[Abbildung]
Vorraussetzung: $A$, $B$ Mengen\todo{prüfen}
\begin{equation}f: A \rightarrow B, x \mapsto f(x)\end{equation}
Eine Abbildung ist eine Vorschrift, die jedem $x \in A$ ein Element $x=f(x) \in B$ zuordnet.
\end{definition}

\subsubsection*{Charakterisierung von Abbildungen}
\begin{itemize}
 \item gehören zu verschiedenen Argumenten verschiedene Funktionswerte, heißt $f$ \emph{injektiv:}
\[\forall x_1,x_2 \in A: x_1 \neq x_2 \Rightarrow f(x_1) \neq f(x_2)\]
 \item Wertebereich $C \subseteq A$:
\[ f(C) = \{f(x)|x\in C\} \]
 \item $f$ \emph{surjektiv}, falls $f(A)=B$
 \item $f$ injektiv und surjektiv $\Leftrightarrow$ $f$ \emph{bijektiv}

$f: A \mapsto B$ ist genau dann bijektiv, falls zu jedem $y \in B$ genau ein $x \in A $ existiert mit $y=f(x)$. In diesem Fall existiert eine Umkehrabbildung $f^{-1}: B \mapsto A$.
\end{itemize}
