\lecture{2010-01-27}

\section{Bestimmtes Integral}
\label{sub:bestInt}

\begin{itemize}
	\item anschaulich: Arbeit, die verrichtet wird \( \hat{=} \) Fläche unter dem Kraftverlauf
	\item allg. Maßtheorie – Integralbegriffe: 
	\begin{itemize}
		\item deterministisch: Riemann-, Lebesgue-Integral
		\item stochastisch: It\={o}-,Stratonovich Kalkül
	\end{itemize}
\end{itemize}

\subsubsection{Vorraussetzung}

Gegeben ist eine beschränkte (notwendige Voraussetzung) Funktion \( f:[a,b]\rightarrow \mathbb{R} \). D.h. es existiert \(  m, M \in \mathbb{R} \quad m \leq f(x) \leq M \)

\begin{note}
	Wäre \( f \) stetig (oder differenzierbar), dann wäre f sowieso beschränkt, da f auf einem abgeschlossenem Interval \( [a,b] \) definiert ist.
\end{note}

\begin{center}
    \begin{tikzpicture}[]
		\draw[->,semithick] (-0.5,0) -- (11,0) node[below]{x};
		\draw[->,semithick] (0,-0.5) -- (0,3.5) node[left]{y};
		
		\draw (-2pt,1) -- (2pt,1) node[left] {\(m\)};
		\draw (-2pt,2.7182) -- (2pt,2.7182) node[left] {\(M\)};

		\draw[domain=1:3] plot [id=log_ln, samples=100] function { log(x) + 1 };
		\draw [dash pattern=on 1pt off 1pt] (1,0)-- (1,1);
		\draw [dash pattern=on 1pt off 1pt] (3,0)-- (3,2.0986);
		\draw (1,2pt) -- (1,-2pt) node[below] {\(\underbrace{a}_{=x_0}\)};
		
		\draw[domain=3:5] plot [id=efunk, samples=100] function { exp((x-3)/2) };
		\draw [dash pattern=on 1pt off 1pt] (4,0)-- (4,1.6487);
		\draw [dash pattern=on 1pt off 1pt] (5,0)-- (5,2.7182);
		\draw (3,2pt) -- (3,-2pt) node[below] {\(x_1\)};
		\draw (4,2pt) -- (4,-2pt) node[below] {\(x_2\)};
		\draw (5,2pt) -- (5,-2pt) node[below] {\(x_3\)};
		
		\draw[domain=7:8] plot [id=const, samples=100] function { exp(1) };
		\draw [dash pattern=on 1pt off 1pt] (7,0)-- (7,2.7182);
		\draw [dash pattern=on 1pt off 1pt] (8,0)-- (8,2.7182);
		\draw (7,2pt) -- (7,-2pt) node[below] {\(x_4\)};
		\draw (8,2pt) -- (8,-2pt) node[below] {\(x_5\)};
		
		\draw[domain=8:10] plot [id=einsdurch, samples=100] function { exp(1)/(x-7) };
		\draw [dash pattern=on 1pt off 1pt] (10,0)-- (10,0.9060);
		\draw (10,2pt) -- (10,-2pt) node[below] {\(\underbrace{b}_{=x_b}\)};
		
	\end{tikzpicture}
\end{center}

Erlaubtes \( f(x) \):
\begin{itemize}
	\item \( f \) darf Sprünge und Lücken haben – allerdings nicht zu viele (Signalverarbeitung)
	\item pathologisches \( f \) : \( f(x) = x \quad x \in \mathbb{Q} \quad x \in \mathbb{R}\backslash\mathbb{Q} \)
\end{itemize}

\subsubsection{Intervallunterteilung  (Zerlegung $ Z $ ) }

\begin{align*}
	h &= \frac{1}{n} (b-a) \\
	x_k &= a+k \cdot h \quad k=0,\ldots,n \\
	a &= x_0<x_1<x_2<\ldots<x_n=b 
\end{align*}
Die Stützstellen sind dabei Äquidistant (aus Bequemlichkeit). Betrachtet man nun \(f\) in den Teilintervallen \( [x_{k-1},x_k] \), so gilt aufgrund der Beschränktheit der Funktion: \( m_k \leq f \leq M_k \) auf \( [x_{k-1},x_k] \)
\begin{align*}
	m_k &= \inf f(x) \\
	M_k &= \sup f(x)
\end{align*}

\begin{note}
	Ist \( f \) stetig, gilt:
	\begin{align*}
		x_{n-1} \leq x \leq x_k \\
		m_k &= \min f(x) \\
		M_k &= \max f(x)
	\end{align*}	
\end{note}

\begin{definition}[Untersumme von f zur Zerlegung Z]
	\[
		U(f, Z)= \sum_{k=0}^nm_k(x_k-x_{k-1}) \qquad \text{einbeschriebenes Rechteck}
	\]
\end{definition}


\begin{definition}[Obersumme von f zur Zerlegung Z]
	\[
		O(f, Z)= \sum_{k=0}^nM_k(x_k-x_{k-1})  \qquad \text{umbeschriebenes Rechteck}
	\]
\end{definition}

\begin{itemize}
	\item Es gilt: \( U(f,Z) \leq O(f,Z) \), denn
	\[ O(f,Z)- U(f,Z)=\sum_{k=1}^n(\underbrace{M_k-m_k}_{\geq0})(\underbrace{x_n-x_{k-1}}_{>0}) \geq 0 \]
	\item Verfeinerung der Zerlegung liefert die feinere Unterteilung \( Z^\star \). 
	\[ U(f,Z) \leq U(f, Z^\star) \leq O(f,Z^\star) \leq O(f,Z) \]
	\item Grenzprozess \( n \rightarrow \infty \): Zerlegungsfolgen \( (Z_s)_{s\in\mathbb{N}} \implies \) 
	\begin{align*}
		U &= \sup U(f,Z_j) \qquad \text{größte einbeschriebene Rechtecksfläche}\\
		O &= \sup O(f,Z_j) \qquad \text{größte umbeschriebene Rechtecksfläche}
	\end{align*}
\end{itemize}

\begin{definition}[Integrierbarkeit]
	Sei \( f[a,b] \rightarrow \mathbb{R} \) beschränkt, so gilt
	\begin{align*}
		U &= \sup U(f,Z_j)\\
		O &= \inf O(f,Z_j)\\
		U &= O \implies \text{f ist integrabel}
	\end{align*}
	Die Zahl \( U = O \) heißt Integral von f über \( [a,b] \)
\end{definition}

\begin{note}
	\( f \geq 0 \): Integral \( \hat{=} \) Flächeninhalt
\end{note}

\subsubsection{Bezeichnung}
\begin{align*}
	f&: \text{Integrand} \\
	x&: \text{Integrationsvariable}\\
	a,b&: \text{untere/obere Integrationsgrenze}\\	
	\int_a^b \,\mathrm{d}x &:\text{Integrationssymbol}
\end{align*}

\begin{example}[f(x)=x in \lbrack0,b\rbrack]
	\[
		\int_0^b f(x) \,\mathrm{d}x = \int_0^b x \,\mathrm{d}x = I_f = b \cdot b \cdot \frac{1}{2} = \frac{1}{2} b^2
	\]
\end{example}
\todo[inline]{Plot der funktion f(x)=x fehlt.}


\subsection{Riemann-Formalismus}
Testfall:
\begin{align*}
	Z_n &= {0,\frac{b}{n},\frac{2b}{n},\ldots,\frac{(n-1)b}{n}} \\
	\quad h &= \frac{b}{n} \quad x_k=h \cdot k \quad k=0,\ldots,n \qquad \text{(Äquidistante Stützstellen)} 
\end{align*}

\begin{align*}
	U(f,Z_n) 
	&= \sum_{k=1}^n m_k(x_k-x_{k-1}) \\
	&= \sum_{k=1}^n x_{k-1}(x_k-x_{k-1}) \\
	&= \sum_{k=1}^n (k-1)h \cdot h = \left( \frac{b}{n} \right)^2 \sum_{k=1}^n (k-1) \\
	&= \frac{b^2}{n^2} \frac{n}{2}(n-1) = \frac{b^2}{2}(1-\frac{1}{n}) \\
	\\
	O(f,Z_n)
	&=\sum_{k=1}^n M_k(x_k-x_{k-1}) = \sum_{k=1}^n x_k(\underbrace{x_k-x_{k-1}}_{h}) = \sum_{k=1}^n k \cdot h \cdot h \\
	&= \left(\frac{b^2}{n^2}\right)\sum_{k=1}^n k = \frac{b^2}{n^2} \cdot \frac{n}{2}(n+1) = \frac{b^2}{2}(1+\frac{1}{n}) \\
	&\implies \underbrace{ \frac{b^2}{2}(1-\frac{1}{n}) }_{U(f,Z_n)}\leq I_f \leq \underbrace{\frac{b^2}{2}(1+\frac{1}{n})}_{O(f,Z_n)}
\end{align*}

Grenzfall \( n \rightarrow \infty : I_f = \frac{1}{2}b^2 \implies f(x) = x \) ist ein Integral.

\begin{note}
	\begin{itemize}
		\item gleiche Technik für \( \int_0^b x^p\,\mathrm{d}x= \frac{b^{p+1}}{p+1}, p \geq 1 \).
		\newline Achtung: Differentiation \( \frac{x^{p+1}}{p+1} \rightarrow \underbrace{x^p}_{=\text{Integrand}} \)
		\item monotone Funktion \( \rightarrow \) integrabel
		\item stetige (meist reicht auch differenzierbare) Funktion \( \rightarrow \) integrabel.
	\end{itemize}	
\end{note} 

\subsection{Numerische Quadratur}
Approximiere den Wert des Integrals. Stammfunktion-Suche über symbolische Verarbeitung (MAPLE, Bronstein)

\begin{itemize}
	\item Brutal: \( y(t) = \int_0^t f(x) \,\mathrm{d}x \implies \text{Differientialgleichung} \quad \dot{y} = f(t) \) \newline \( \rightarrow \) ODE Software MATLAB/MATHEMATICA
	\item Besser: Qudraturformeln
\end{itemize}

\subsubsection{Verfahren 1: Rechteckregel $[a,b]$}
\begin{align*}
	x_k &= a+ k \frac{b-a}{n} \qquad h= \frac{b-a}{n} \qquad \text{(Äquidistante Stützstellen)} \\
	t_k &= x_{k-1}+\frac{1}{2}(x_k-x_{k-1})= \frac{1}{2}(x_k+x_{k-1})
\end{align*}

\todo[inline]{Grafik zur Qudratur fehlt.}

\[ \int_a^b f(x) \,\mathrm{d}x \dot{=} \underbrace{\frac{b-a}{n} \sum_{k=1}^n f(t_k)}_{\text{Auf Rechner auswerten!}} \]

\subsubsection{Praxis}

Auswertungsreihe: \( n=8, 16, 32, 64, \ldots \) \newline
Vergleiche \( I_8,I_{16}, I_{32}, \ldots\) auf stehende führende Dezimale!

\begin{align*}
	I_8 &= \underline{1},0578 \\
	I_{16} &= \underline{1,2}569 \\
	I_{32} &= \underline{1,24}87 \\
	I_{64} &= \underline{1,249}3 \\
\end{align*}





