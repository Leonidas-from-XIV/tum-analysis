\newpage
\lecture{2009-11-04}

% re im vorher für geschweifte klammer

\subsubsection*{Anwendung: Darstellung einer Geraden}

Idee: Charakterisiere die Gerade durch ``Aufpunkt'' und ``Richtungsvektor''

\begin{equation*}
	\vec{x} = \vec{a} + \lambda \vec{b}
\end{equation*}

\begin{center}
	\definecolor{ttttff}{rgb}{0.2,0.2,1}
	\definecolor{qqqqff}{rgb}{0,0,1}
	\definecolor{ffffff}{rgb}{1,1,1}
	\definecolor{xdxdff}{rgb}{0.49,0.49,1}
	\begin{tikzpicture}[line cap=round,line join=round,>=triangle 45,x=2.0cm,y=2.0cm]
	\draw[->,color=black,dash pattern=on 3pt off 3pt] (0,0) -- (2.5,0);
	\draw[color=black] (0pt,-10pt) node[right] {\footnotesize $0$};
	\clip(-2,-0.5) rectangle (3,2.2);
	\draw [->] (0,0) -- (-1.5,1.58);
	\draw [->] (0,0) -- (1.4,1.2);
	\draw [color=ttttff,domain=-2:3] plot(\x,{(-4.01--1.58*\x)/-1.5});
	\fill [color=black] (0,0) circle (1.5pt);
	\draw[color=black] (-0.74,0.92) node [anchor=west] {$\vec{b}$};
	\draw[color=black] (0.7,0.82) node [anchor=south] {$\vec{a}$};
	\draw[color=black] (2,1) node {$\lambda\vec{b}$};
	\draw[color=black] (1.5,0) node [anchor=south] {$\vec{x}$};
	\end{tikzpicture}
\end{center}
%
Erinnerung:
\begin{center}
	\begin{tikzpicture}[line cap=round,line join=round,>=triangle 45,x=2.5cm,y=2.5cm]
	\draw[->,color=black] (-0.5,0) -- (2,0);
	\foreach \x in {,1}
	\draw[shift={(\x,0)},color=black] (0pt,2pt) -- (0pt,-2pt) node[below] {\footnotesize $\x$};
	\draw[->,color=black] (0,-0.5) -- (0,1.2);
	\foreach \y in {}
	\draw[shift={(0,\y)},color=black] (2pt,0pt) -- (-2pt,0pt) node[left] {\footnotesize $\y$};
	\clip(-0.8,-0.8) rectangle (2.8,1.6);
	\draw [domain=-0.5:2] plot(\x,{(--0.5--0.28*\x)/1});
	\fill [color=black] (0,0.5) circle (1.5pt);
	\draw[color=black] (0.16,0.78) node {$k$};
	\fill [color=black] (1,0.78) circle (1.5pt);
	\draw[color=black] (1,0.85) node [anchor=north west] {$y = mx + k$};

	\draw[color=black] (2.1,0.08) node [anchor=north] {x};
	\draw[color=black] (0.05,1) node [anchor=south west] {y};
	\end{tikzpicture}
\end{center}
%
Zusammenhang:
%
\begin{itemize}
	\item $k \equals \vec{a}$
	\item $m \equals \vec{b}$
	\item $x \equals \lambda$
\end{itemize}
%
Geradengleichung in Koordinaten:
\begin{equation*}
  \begin{matrix}
    \ds\binom{x}{y} = \binom{x}{mx+k} = & \ds\binom{0}{k} & + & \ds x      & \cdot & \ds\binom{1}{m} \\
                                        &    \downarrow   &   & \downarrow &       & \downarrow \\
                                        &    \vec{a}      &   & \lambda    &       & \vec{b}
  \end{matrix}
\end{equation*}

%%% DONE --lars
% 
% 
% \begin{align*} % TODO: Richtig ausrichten
% 	\binom{x}{y} = \binom{x}{mx+k} = &\binom{0}{k} + x& \binom{1}{m} \\
% 	\downarrow& \downarrow& \downarrow \\
% 	\vec{a}& \lambda& \vec{b}
% \end{align*}


% Pfeile $\leadsto \vec{a}$

\noindent Drehen eines Vektors $\longrightarrow$ Matrizenbeschreibung \\
allgemein: Matrizen entsprechen linearen Abbildungen

\begin{align*}
&V, W: \text{ Vektorraum} \\
f: &V \mapsto W  \text{ lineare Abbildung bijektiv zu Matrizen} \\
&y = Ax \\
f: &v \mapsto w \\
&x \mapsto y
\end{align*}\todo{besseres Layout}

$V, W$ als Vektorraum, $\mathbb{R}^{n} \mapsto$ A: (n x n)-Matrix \\
A staucht, dehnt, rotiert Vektor x.

\begin{align*}
A \mapsto 	\begin{cases} 
			(0 / 0) \text{ Diagonalgestalt} \\
			(o.../0) \text{ Jordannormalform}
		\end{cases}
\end{align*}\todo{Diago\-nalen verschö\-nern}

\subsubsection*{Drehung}

\missingfigure{Bild}

Gegeben:
\begin{itemize}
	\item $x, y$, d. h. $P$
	\item $\alpha$
\end{itemize}
%
Wie sehen $u$, $v$ aus?
%
\begin{align*}
x &= \overline{OS} - \overline{RS} \\
\cos \alpha &= \frac{\overline{OS}}{\alpha} \\
\sin \alpha &= \frac{\overline{TQ}}{v} = \frac{\overline{RS}}{v}
\end{align*}
%
\begin{align*}
\text{Ergebnis: } x &= u \cos \alpha - v \sin \alpha \\
\text{analog: } &u \sin \alpha + v \cos \alpha
\end{align*}

\subsubsection*{Matrix-Vektor-Notation}

\begin{definition}
	$ R_{\alpha} =
	\begin{pmatrix}
		\cos \alpha & -\sin \alpha \\
		\sin \alpha & \cos \alpha
 	\end{pmatrix}$
\end{definition}

\begin{equation*}
	\begin{pmatrix}
		x \\
		y
 	\end{pmatrix} = R_{\alpha} 
 	\begin{pmatrix}
		u \\
		v
 	\end{pmatrix}
 	\leadsto
 		\begin{pmatrix}
		u \\
		v
 	\end{pmatrix} = R_{\alpha}^{T}
 	\begin{pmatrix}
		x \\
		y
 	\end{pmatrix}
\end{equation*}
%
Weiterhin gilt:
\begin{equation*}
	R_{\alpha}^{-1} = R_{-\alpha} = R_{\alpha}^{T}
\end{equation*}

\begin{note}
Drehung liefert Anlass für Matrix-Vektor-Produkt und Matrix-Matrix-Produkt
\end{note}

\noindent Zwei Drehungen um $\angle \alpha$ und $\angle \beta$
\begin{equation*}
	R_{\beta} R_{\alpha} = R_{\beta + \alpha}
\end{equation*}
%
Drehungen im $\mathbb{R}^{n}$ in der Ebene $(i, j)$

\missingfigure{Großes Bild.}

\subsubsection*{Anwendung in $\mathbb{R}^{2}$: Längenmessung/Abstand}

\begin{equation*}
	\vec{x} = 
 	\begin{pmatrix}
		x_{1} \\
		x_{2}
 	\end{pmatrix}
\end{equation*}

\begin{definition}Länge von $\vec{x}$ als $|x| = \sqrt{x_1^2 + x_2^2}$
\end{definition}

\missingfigure{kleines Bild}

\begin{note}
Abstand zweier Vektoren entspricht der Länge des Differenzvektors $\vec{x} - \vec{y}$
\end{note}

\begin{equation*}
	|\vec{x} - \vec{y}| = \left|
 	\begin{array}{cc}
		(x_1 - y_1)\\
		(x_2 - y_2)
 	\end{array}\right| = 
 	\sqrt{(x_1 - y_1)^2 + (x_2 - y_2)^2}
\end{equation*}

\subsubsection*{Anwendung: Rechte Winkel}

\begin{theorem}
\begin{equation*}
	\vec{x} \perp \vec{y} \Leftrightarrow |\vec{x}|^2 + |\vec{y}|^2 = |\vec{x} - \vec{y}|^2
\end{equation*}
wegen
\begin{align*}
	|\vec{x}|^2 + |\vec{y}|^2 = (x_1 - y_1)^2 + (x_2 - y_2)^2 &= x_1^2 + x_2^2 + y_1^2 + y_2^2 - 2(x_1 y_1 + x_2 y_2) \\
	\Rightarrow x_1 y_1 + x_2 y_2 &= 0 \\
	S = x^T y &= 0
\end{align*}
\end{theorem}

%%% LARS REVIEWSTATE

\subsubsection*{Formale Beschreibung}
\begin{definition}[Skalarprodukt]\flush
$ x, y \in \mathbb{R}^2 $ \\
Notation: $ <, >; (, ); x^T y$ o. ä. \\
$<, >$: $\mathbb{R}^n \times \mathbb{R}^n \mapsto \mathbb{R}$ pos. def. symm. Bilinearform
\end{definition}

% TODO: mehr Platz
d.h. \begin{itemize}
	\item $ <x, y> = <y, x>$
	\item $ <\lambda x_1 + \lambda x_2, y> = \lambda <x_1, y> + \lambda <x_2, y>$
	\item $ <x, x> \geq 0 = 0 \text{falls} x = 0$
\end{itemize}

Das Skalarprodukt induziert eine Länge (Norm)

\begin{equation*}
	|| x || = \sqrt{<x, x>}
\end{equation*}

Wenn $V$ ein reeller Vektorraum mit Skalarprodukt ist dann bezeichnen wir ihn als \emph{euklidischen Vektorraum}

\subsection{Ungleichung von Cauchy-Schwarz}
Gegeben: $x, y \in \mathbb{R}^n$\\
In kompakter Notation:
\begin{equation*}
	<x, y> \leq || x || || y ||
\end{equation*}
Ausgeschrieben:
\begin{equation*}
	\sum_{k=1}^n x_k y_k \leq \sqrt{\sum_{k=1}^n x_k^2} \sqrt{\sum_{k=1}^n y_k^2} 
\end{equation*} 

\begin{align*}
	n = 1:& x_1 y_1 \leq x_1 y_1 \\
	n = 2:& x_1 y_1 + x_2 y_2 \leq sqrt{x_1^2 + x_2^2} sqrt{y_1^2 + y_2^2} \\
	\Rightarrow& (x_1 y_1 + x_2 y_2)^2 \leq (x_1^2 + x_2^2) (y_1^2 + y_2^2) \\
	& 2 x_1 y_1 x_2 y_2 \leq x_1^2 y_2^2 - x_2^2 y_1^2 \\
	\Rightarrow& (x_1 y_2 -  x_2 y_1)^2 \geq 0
\end{align*}
allg. Beweis: s. Bornemann

\begin{definition}
	Norm eines Vektors $ x \in \mathbb{R}^n $: \\
	$ ||.||: \mathbb{R}^n \mapsto \mathbb{R} $ \\
	Eigenschaften:
	\begin{itemize}
		\item $ || x || = 0 \Leftrightarrow x = 0 $
		\item $ || \lambda x || = | \lambda | || x || $
		\item $ || x + y || \leq || x || + || y || $
	\end{itemize}
\end{definition}

Bei uns im $ \mathbb{R}^n $ ist $ || x ||_2 $ die euklidische Norm.

$(1)$ und $(2)$ sind trivial, nur Dreiecksungleichung $(3)$ ist zu zeigen.

\begin{align*}
	|| x + y ||^2 =& || x ||^2 + 2<x, y> + || y ||^2 \\
	\leq& || x ||^2 + 2 || x || || y || + || y ||^2 \\
	=& (|| x || + || y ||)^2
\end{align*}

\begin{note}
Zwei weitere Normen
	\begin{itemize}
		\item Maximumnorm $ || x ||_\infty = max \{ |x_k|, k=1..n\} $
		\item $l^1$-Norm  $ || x ||_1 = \sum_{k=1}^n |x_k|$
	\end{itemize}
\end{note}
