\newpage
\lecture{2009-11-04}

\subsubsection*{Anwendung: Darstellung einer Geraden}

Idee: Charakterisiere die Gerade durch ``Aufpunkt'' und ``Richtungsvektor''

\begin{equation*}
	\vec{x} = \vec{a} + \lambda \vec{b}
\end{equation*}

\begin{center}
	\begin{tikzpicture}[line cap=round,line join=round,>=triangle 45,x=2.0cm,y=2.0cm]
	\draw[->,color=black,dash pattern=on 3pt off 3pt] (0,0) -- (2.5,0);
	\draw[color=black] (0pt,-10pt) node[right] {\footnotesize $0$};
	\clip(-2,-0.5) rectangle (3,2.2);
	\draw [->] (0,0) -- (-1.5,1.58);
	\draw [->] (0,0) -- (1.4,1.2);
	\draw [color=ttttff,domain=-2:3] plot(\x,{(-4.01--1.58*\x)/-1.5});
	\fill [color=black] (0,0) circle (1.5pt);
	\draw[color=black] (-0.74,0.92) node [anchor=west] {$\vec{b}$};
	\draw[color=black] (0.7,0.82) node [anchor=south] {$\vec{a}$};
	\draw[color=black] (2,1) node {$\lambda\vec{b}$};
	\draw[color=black] (1.5,0) node [anchor=south] {$\vec{x}$};
	\end{tikzpicture}
\end{center}
%
Erinnerung:
\begin{center}
	\begin{tikzpicture}[line cap=round,line join=round,>=triangle 45,x=2.5cm,y=2.5cm]
	\draw[->,color=black] (-0.5,0) -- (2,0);
	\foreach \x in {,1}
	\draw[shift={(\x,0)},color=black] (0pt,2pt) -- (0pt,-2pt) node[below] {\footnotesize $\x$};
	\draw[->,color=black] (0,-0.5) -- (0,1.2);
	\foreach \y in {}
	\draw[shift={(0,\y)},color=black] (2pt,0pt) -- (-2pt,0pt) node[left] {\footnotesize $\y$};
	\clip(-0.8,-0.8) rectangle (2.8,1.6);
	\draw [domain=-0.5:2] plot(\x,{(--0.5--0.28*\x)/1});
	\fill [color=black] (0,0.5) circle (1.5pt);
	\draw[color=black] (0.16,0.78) node {$k$};
	\fill [color=black] (1,0.78) circle (1.5pt);
	\draw[color=black] (1,0.85) node [anchor=north west] {$y = mx + k$};

	\draw[color=black] (2.1,0.08) node [anchor=north] {x};
	\draw[color=black] (0.05,1) node [anchor=south west] {y};
	\end{tikzpicture}
\end{center}
%
Zusammenhang:
%
\begin{itemize}
	\item $k \equals \vec{a}$
	\item $m \equals \vec{b}$
	\item $x \equals \lambda$
\end{itemize}
%
Geradengleichung in Koordinaten:
\begin{equation*}
  \begin{matrix}
    \ds\binom{x}{y} = \binom{x}{mx+k} = & \ds\binom{0}{k} & + & \ds x      & \cdot & \ds\binom{1}{m} \\
                                        &    \downarrow   &   & \downarrow &       & \downarrow \\
                                        &    \vec{a}      &   & \lambda    &       & \vec{b}
  \end{matrix}
\end{equation*}

\noindent Drehen eines Vektors $\longrightarrow$ Matrizenbeschreibung \\
allgemein: Matrizen entsprechen linearen Abbildungen

\begin{align*}
	&V, W: \text{ Vektorraum} \\
	&\begin{matrix}
		f: & V & \mapsto  & W & \text{ lineare Abbildung bijektiv zu Matrizen} \\
		f: & v & \mapsto  & w & y = Ax \\
		   & \downarrow & & \downarrow \\ 
		   & x &\mapsto   & y
	\end{matrix}
\end{align*}
$V, W$ als Vektorraum, $\mathbb{R}^{n} \mapsto$ A: (n x n)-Matrix \\
A staucht, dehnt, rotiert Vektor x.

\begin{align*}
A \mapsto 	\begin{cases} 
			\begin{pmatrix} d_1 & & 0 \\  & \ddots & \\ 0 & & d_n \end{pmatrix} \text{ Diagonalgestalt} \\
			\begin{pmatrix}
			J_1 &          & 0   \\
			    & \ddots &     \\ 
			  0 &          & J_k \end{pmatrix} \text{ mit } J_i= \begin{pmatrix} \lambda_i & 1 &  &  & 0 \\  & \lambda_i & 1 &  &  \\ && \ddots{} & \ddots{}\\ &&& \lambda_i & 1 \\ 0 &  & &  & \lambda_i \end{pmatrix}  \text{ Jordannormalform}
		\end{cases}
\end{align*}

\subsubsection*{Drehung}

\begin{center}
\begin{tikzpicture}[line cap=round,line join=round,>=triangle 45,x=1.5cm,y=1.5cm]
\draw[->,color=black] (-2,0) -- (4,0);
\draw[->,color=black] (0,-1) -- (0,4);
\draw[color=black] (0pt,-10pt) node[right] {\footnotesize $0$};
\clip(-2,-1) rectangle (4,4);
\draw [shift={(1.01,3.36)},color=qqwuqq,fill=qqwuqq,fill opacity=0.1] (0,0) -- (-90:0.6) arc (-90:-40.31:0.6) -- cycle;
\draw [shift={(0,0)},color=qqwuqq,fill=qqwuqq,fill opacity=0.1] (0,0) -- (0:0.6) arc (0:47.74:0.6) -- cycle;
\draw [->] (0,0) -- (-2.9,2.46);
\draw [->] (0,0) -- (2.98,3.28);
\draw [dash pattern=on 3pt off 3pt] (2.17,0) -- (2.17,2.36);
\draw [dash pattern=on 3pt off 3pt,domain=-1.16:1.16] plot(\x,{(--6.69--3.28*\x)/2.98});
\draw [dash pattern=on 3pt off 3pt,domain=1.16:2.17] plot(\x,{(-12.24--2.46*\x)/-2.9});
\draw [dash pattern=on 3pt off 3pt] (1.01,0) -- (1.01,3.36);
\draw [dash pattern=on 3pt off 3pt,domain=0:1.16] plot(\x,{(--56.61-0*\x)/16.84});
\fill [color=qqqqff] (-2.9,2.46) circle (1.5pt);
\draw[color=qqqqff] (-2.74,2.74) node {$A$};
\draw[color=black] (-1.22,1.34) node {$\vec{v}$};
\draw[color=black] (1.5,1.86) node {$\vec{u}$};
\fill [color=black] (2.17,2.38) circle (1.5pt);
\draw[color=black] (2.4,2.36) node {$Q$};
\draw[color=black] (2.06,6.16) node {$a_1$};
\draw[color=black] (3.16,6.16) node {$b$};
\draw[color=black] (-1.88,6.16) node {$c$};
\fill [color=uququq] (1.01,3.36) circle (1.5pt);
\draw[color=uququq] (1.16,3.64) node {$P$};
\draw[color=black] (1.24,6.16) node {$d$};
\fill [color=xdxdff] (3.56,6.16) circle (1.5pt);
\draw[color=xdxdff] (3.7,6.22) node {$F$};
\fill [color=xdxdff] (8.1,-2.65) circle (1.5pt);
\draw[color=xdxdff] (8.26,-2.36) node {$G$};
\fill [color=xdxdff] (1.01,-3.88) circle (1.5pt);
\draw[color=xdxdff] (1.2,-3.6) node {$H$};
\draw[color=qqwuqq] (1.16,3) node {$\alpha$};
\draw[color=black] (-4.14,3.22) node {$e$};
\draw[color=qqwuqq] (0.4,0.19) node {$\alpha$};
\fill [color=uququq] (0,3.36) circle (1.5pt);
\draw[color=uququq] (0.16,3.64) node {$y$};
\fill [color=uququq] (1.01,0) circle (1.5pt);
\draw[color=uququq] (1.18,-0.3) node {$x, R$};
\fill [color=uququq] (2.17,0) circle (1.5pt);
\draw[color=uququq] (2.3,-0.3) node {$S$};
\end{tikzpicture}
\end{center}

Gegeben:
\begin{itemize}
	\item $x, y$, d. h. $P$
	\item $\alpha$
\end{itemize}
%
Wie sehen $u$, $v$ aus?
%
\begin{align*}
x &= \overline{OS} - \overline{RS} \\
\cos \alpha &= \frac{\overline{OS}}{\alpha} \\
\sin \alpha &= \frac{\overline{TQ}}{v} = \frac{\overline{RS}}{v}
\end{align*}
%
\begin{align*}
\text{Ergebnis: } x &= u \cos \alpha - v \sin \alpha \\
\text{analog: } &u \sin \alpha + v \cos \alpha
\end{align*}

\subsubsection*{Matrix-Vektor-Notation}

\begin{definition}
	$ R_{\alpha} =
	\begin{pmatrix}
		\cos \alpha & -\sin \alpha \\
		\sin \alpha & \cos \alpha
 	\end{pmatrix}$
\end{definition}

\begin{equation*}
	\begin{pmatrix}
		x \\
		y
 	\end{pmatrix} = R_{\alpha} 
 	\begin{pmatrix}
		u \\
		v
 	\end{pmatrix}
 	\leadsto
 		\begin{pmatrix}
		u \\
		v
 	\end{pmatrix} = R_{\alpha}^{T}
 	\begin{pmatrix}
		x \\
		y
 	\end{pmatrix}
\end{equation*}
%
Weiterhin gilt:
\begin{equation*}
	R_{\alpha}^{-1} = R_{-\alpha} = R_{\alpha}^{T}
\end{equation*}

\begin{note}
Drehung liefert Anlass für Matrix-Vektor-Produkt und Matrix-Matrix-Produkt
\end{note}

\noindent Zwei Drehungen um $\angle \alpha$ und $\angle \beta$
\begin{equation*}
	R_{\beta} R_{\alpha} = R_{\beta + \alpha}
\end{equation*}
%
Drehungen im $\mathbb{R}^{n}$ in der Ebene $(i, j)$

\begin{equation*}
	R_{\alpha} =
	\bordermatrix{
		&		&	i 			&	& j	\cr
		&	\begin{smallmatrix} 1 & & \\ &\ddots & & \\ & & 1 \end{smallmatrix} & & & & 0 \cr
	i	&		&	\cos \alpha	&	&	-\sin \alpha \cr
		&		&	\vdots			& \begin{smallmatrix} 1 & & \\ &\ddots & & \\ & & 1 \end{smallmatrix} & \vdots \cr
	j	&		&	\sin \alpha	&	&	-\cos \alpha \cr
		&	0	&					&	&	& \begin{smallmatrix} 1 & & \\ &\ddots & & \\ & & 1 \end{smallmatrix}
	}
\end{equation*}

\subsubsection*{Anwendung in $\mathbb{R}^{2}$: Längenmessung/Abstand}

\begin{equation*}
	\vec{x} = 
 	\begin{pmatrix}
		x_{1} \\
		x_{2}
 	\end{pmatrix}
\end{equation*}

\begin{definition}Länge von $\vec{x}$ als $|x| = \sqrt{x_1^2 + x_2^2}$
\end{definition}

\begin{center}
\begin{tikzpicture}[line cap=round,line join=round,>=triangle 45,x=1.5cm,y=1.5cm]
\draw[->,color=black] (-0.3,0) -- (2.5,0);
\draw[->,color=black] (0,-0.3) -- (0,1.7);
\draw[color=black] (0pt,-10pt) node[right] {\footnotesize $0$};
\clip(-0.5,-0.5) rectangle (2.5,1.7);
\draw [->] (0,0) -- (2,1.36);
\draw [dash pattern=on 3pt off 3pt] (2,1.36)-- (2,0);
\draw [dash pattern=on 3pt off 3pt] (2,1.36)-- (0,1.36);
\draw[color=black] (1.04,0.9) node {$\vec{x}$};
\draw[color=black] (2,0) node [anchor=north] {$x_1$};
\draw[color=black] (0,1.36) node [anchor=east] {$x_2$};

\end{tikzpicture}
\end{center}

\begin{note}
Abstand zweier Vektoren entspricht der Länge des Differenzvektors $\vec{x} - \vec{y}$
\end{note}

\begin{equation*}
	|\vec{x} - \vec{y}| = \left|
 	\begin{array}{cc}
		(x_1 - y_1)\\
		(x_2 - y_2)
 	\end{array}\right| = 
 	\sqrt{(x_1 - y_1)^2 + (x_2 - y_2)^2}
\end{equation*}

\subsubsection*{Anwendung: Rechte Winkel}

\begin{theorem}
\begin{equation*}
	\vec{x} \perp \vec{y} \Leftrightarrow |\vec{x}|^2 + |\vec{y}|^2 = |\vec{x} - \vec{y}|^2
\end{equation*}
wegen
\begin{align*}
	|\vec{x}|^2 + |\vec{y}|^2 = (x_1 - y_1)^2 + (x_2 - y_2)^2 &= x_1^2 + x_2^2 + y_1^2 + y_2^2 - 2(x_1 y_1 + x_2 y_2) \\
	\Rightarrow x_1 y_1 + x_2 y_2 &= 0 \\
	S = x^T y &= 0
\end{align*}
\end{theorem}

\subsubsection*{Formale Beschreibung}
\begin{definition}[Skalarprodukt]\flush
	\begin{itemize}
		\item $ x, y \in \mathbb{R}^2 $
		\item Notation: ``$ \left<, \right>$''; ``$(, )$''; ``$x^T y$'' o. ä.
		\item $\left<, \right>$: $\mathbb{R}^n \times \mathbb{R}^n \mapsto \mathbb{R}$ pos. def. symm. Bilinearform
	\end{itemize}
\end{definition}

\noindent d. h. \begin{itemize}
	\item $ \left<x, y\right> = \left<y, x\right>$
	\item $ \left<\lambda x_1 + \lambda x_2, y\right> = \lambda \left<x_1, y\right> + \lambda \left<x_2, y\right>$
	\item $ \left<x, x\right> \geq 0 = 0 \text{ falls } x = 0$
\end{itemize}
%
Das Skalarprodukt induziert eine Länge (Norm):
\begin{equation*}
	\left\| x \right\| = \sqrt{\left<x, x\right>}
\end{equation*}
%
\begin{definition}[euklidischer Vektorraum]
  Ein reeller Vektorraum mit Skalarprodukt heißt \emph{euklidischer Vektorraum}.
\end{definition}

\subsection{Ungleichung von Cauchy-Schwarz}
gegeben: $x, y \in \mathbb{R}^n$\\
in kompakter Notation:
\begin{equation*}
	\left<x, y\right> \leq \left\| x \right\| \left\| y \right\|
\end{equation*}
ausgeschrieben:
\begin{equation*}
	\sum_{k=1}^n x_k y_k \leq \sqrt{\sum_{k=1}^n x_k^2} \sqrt{\sum_{k=1}^n y_k^2} 
\end{equation*} 

\begin{align*}
\intertext{$n=1$:}
	x_1 y_1 &\leq x_1 y_1
\intertext{$n=2$:}
	x_1 y_1 + x_2 y_2 &\leq \sqrt{x_1^2 + x_2^2} \sqrt{y_1^2 + y_2^2} \\
	(x_1 y_1 + x_2 y_2)^2 &\leq (x_1^2 + x_2^2) (y_1^2 + y_2^2) \\
	2 x_1 y_1 x_2 y_2 &\leq x_1^2 y_2^2 - x_2^2 y_1^2 \\
	(x_1 y_2 -  x_2 y_1)^2 &\geq 0
\end{align*}
allgemeiner Beweis: siehe Bornemann \todo{Literatur\-angabe per bib}

\begin{definition}[Norm eines Vektors]\flush
	$ x \in \mathbb{R}^n $: \\
	\begin{equation*} \left\| \;\cdot\; \right\|: \mathbb{R}^n \mapsto \mathbb{R} \end{equation*}
	Eigenschaften:
	\begin{itemize}
		\item $ \left\| x \right\| = 0 \Leftrightarrow x = 0 $
		\item $ \left\| \lambda x \right\| = | \lambda | \cdot \left\| x \right\| $
		\item $ \left\| x + y \right\| \leq \left\| x \right\| + \left\| y \right\| $
	\end{itemize}
\end{definition}
%
\begin{note}
  Im $ \mathbb{R}^n $ ist $ \left\| x \right\|_2 $ die euklidische Norm.\\
  Die ersten beiden Eigenschaften sind trivial, nur die Dreiecksungleichung ist zu zeigen:
    \begin{align*}
	\left\| x + y \right\|^2 &= \left\| x \right\|^2 + 2\cdot \left<x, y\right> + \left\| y \right\|^2 \\
	&\leq \left\| x \right\|^2 + 2 \cdot \left\| x \right\| \cdot \left\| y \right\| + \left\| y \right\|^2 \\
	&= (\left\| x \right\| + \left\| y \right\|)^2
    \end{align*}
\end{note}

\begin{note}[zwei weitere Normen]
	\begin{itemize}
		\item Maximumnorm $ \left\| x \right\|_\infty = \max \left( \left\{ \left|x_k\right|\!; k=1 \ldots n\right\} \right)$
		\item $\ell^1$-Norm  $ \left\| x \right\|_1 = \sum_{k=1}^n \left|x_k\right|$
	\end{itemize}
\end{note}
