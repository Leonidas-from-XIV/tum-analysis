\lecture{2010-01-20 Teil 2}
% Führe das Beispiel aus der letzten Vorlesung natlos fort.
\begin{equation*}
D=\mathbb{R}\setminus \{0\}
\end{equation*}
$f(0)$ wird stetig ergänzt.
\begin{proposition}
  $\forall \nu$ gilt: $f^{(\nu)}(0) = 0$
\end{proposition}

\begin{proof}[per Induktion]\flush
   \begin{align*}
      \begin{array}{ll}
         \nu = 1: & f'(x) = \euler^{-\frac{1}{x^2}}\frac{\mathrm d}{\mathrm dx}(-\frac{1}{x^2}) = \frac{2}{x^3}f(x) \\
         & f'(0) \text{stetig ergänzen zu 0} \\
         \nu = n+1: & \text{Annahme: } f^{(\nu)}(x) = q_\nu(x)\cdot f(x) \\
         & \nu=1\ldots n \quad q_\nu(x) \quad \text{rat. Funktion.}\\
         & f^{(n+1)}(x) = (\frac{\mathrm d}{\mathrm dx} q_n(x))\cdot f(x) + \frac{2}{x^3} q_n(x) f(x)= q_{n+1}(x)f(x)
      \end{array}
   \end{align*}
\end{proof}
Ergebnis: $f^{(\nu)}(x) = q_\nu(x)\cdot f(x)$, $q_\nu$ rationale Funktion.
Verhalten in $x_0 = 0$:
\begin{align*}
\begin{array}{rlc}
   \lim_{x\rightarrow \infty} \frac{\euler^x}{x^n} & = \infty & \forall n \in \mathbb{N}\\
   \lim_{x\rightarrow \infty} \frac{x^n}{\euler^x} & = 0 & \forall n \in \mathbb{N}\\
   \lim_{x\rightarrow 0} \frac{\euler^{-\frac{1}{x^2}}}{x^n} & = 0 & \leadsto f^{(n)}(0) = 0
   \end{array}
\end{align*}
$\leadsto$ Taylorreihe $f \equiv 0 \neq \euler^{-\frac{1}{x^2}}$\\
Offen: Konvergenz Taylorreihe zu $f(x)$?\\
Taylorformel entählt Restglied $R_\nu(x)$. Bildet die Folge der Restglieder $(R_\nu(x))_{\nu \in \mathbb{N}}$ eine Nullfolge, so konvergiert die Taylorreihe (als Grenzwert der Taylorpolynome) gegen $f(x)$.

Bemerkung: für $f(x) = \euler^{-\frac{1}{x^2}}$ bilden $R_\nu(x)$ keine Nullfolge.

\todo[inline]{Anmerkung Landausymbole}
%\begin{note}
%Zu den Landausymbolen:\\
%Wir verstehen unter:
%\begin{itemize}
%
%   \item $R_{n+1}(x) = o(x)$ den Grenzwert für $x\to 0$:
%
%\end{itemize}
% TODO: Fertig machen
%\end{note}
