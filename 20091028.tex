\newpage
\lecture{2009-10-28}


\subsection{Grundoperationen in $\mathbb{C}$}
\paragraph{Symbol} imaginäre Einheit $i = \sqrt{-1} $

\paragraph{Komplexe Zahlenebene} $z \in \mathbb{C}$\\
% Bild zur komplexen Zahlenebene aus geogebra
\begin{minipage}[htbp]{4 cm}
\vspace{0.5 cm}
\begin{tikzpicture}[line cap=round,line join=round,>=triangle 45,x=1.0cm,y=1.0cm]
\draw[->,color=black] (-0.5,0) -- (2.5,0);
\foreach \x in {,1,2}
\draw[shift={(\x,0)},color=black] (0pt,2pt) -- (0pt,-2pt);
\draw[color=black] (1.98,0.08) node [anchor=south west] { Re};
\draw[->,color=black] (0,-1.5) -- (0,2.5);
\foreach \y in {-1,1,2}
\draw[shift={(0,\y)},color=black] (2pt,0pt) -- (-2pt,0pt);
\draw[color=black] (0.1,2.06) node [anchor=west] { Im};
\clip(-0.5,-1.5) rectangle (2.5,2.5);
\draw [dash pattern=on 2pt off 2pt] (0,1)-- (1,1);
\draw [dash pattern=on 2pt off 2pt] (1,0)-- (1,1);
\draw [dash pattern=on 2pt off 2pt] (1,-1)-- (1,0);
\draw [dash pattern=on 2pt off 2pt] (1,-1)-- (0,-1);
\fill [color=xdxdff] (0,1) circle (1.5pt);
\draw[color=xdxdff] (-0.35,1.1) node {$b$};
\fill [color=qqqqff] (1,1) circle (1.5pt);
\draw[color=qqqqff] (1.2,1.28) node {$z$};
\fill [color=xdxdff] (1,0) circle (1.5pt);
\draw[color=xdxdff] (1.3,0.28) node {$a$};
\fill [color=xdxdff] (0,-1) circle (1.5pt);
\draw[color=xdxdff] (-0.35,-0.9) node {$-b$};
\fill [color=qqqqff] (1,-1) circle (1.5pt);
\draw[color=qqqqff] (1.25,-1.1) node {$\overline{z}$};
\end{tikzpicture}
% bild ende
\end{minipage}
  \begin{minipage}[htbp]{6cm}
  
Darstellung: $ z = a + \in b; a,b \in \mathbb{R}$ \\
Realteil von $z$: $Re(z) = a$ \\
Imaginärteil von $z$: $Im(z) = b$ 
\end{minipage}

\begin{definition}
 $\overline{z} = a - \imag b$ Komplex konjugiertes von $z$
\end{definition}

\paragraph{Rechenoperationen}
\begin{enumerate}

\item ``$\pm$'' $z_1 \pm z_2$
\begin{align*}
z_1 & = a_1 + \imag b_1 \\ 
z_2 & = a_2 + \imag b_2 \\
z_1 \pm z_2 & = (a_1 \pm a_2) + \imag \cdot (b_1 \pm b_2)
\end{align*}

\item "$\ast$" (intuitiv)

$z_1 \cdot z_2 = (a_1 + \imag b_1) \cdot (a_2 + \imag b_2) = (a_1 a_2 - b_1 b_2) + \imag \cdot (b_1 a_2 + a_1 b_2) \in \mathbb{C}$

(Term: $\imag b_1 \cdot \imag b_2 = i^2 b_1 b_2 = - b_1 b_2$)

\item Division, $z_2 \neq 0$

$$ \frac{z_1}{z_2} = \frac{a_1 + \imag b_1}{a_2 + \imag b_2} = \frac{(a_1 + \imag b_1) \cdot (a_2 - \imag b_2)}{\underbrace{(a_2 + \imag b_2) \cdot (a_2 - \imag b_2)}_{a_2^2 - \imag^2 b_2^2 = a_2^2 + b_2^2}} = \frac{a_1 a_2 + b_1 b_2}{a_2^2 + b_2^2} + \frac{b_1 a_2 - a_1 b_2}{a_2^2 + b_2^2} \imag $$

\end{enumerate}

\paragraph{Formale Einführung}
\begin{alignat*}{3}
& \mathbb{R}^2 && \overset{bij.}{\longleftrightarrow} && \mathbb{C} \\
& (x, y) && \longleftrightarrow && z = x + \imag y
\end{alignat*}
\begin{enumerate}
\item Addition von 2-Tupeln $\mathbb{R}^2$ in $\mathbb{C}$

$(x_1, y_1) \pm (x_2, y_2) := (x_1 \pm x_2, y_1 \pm y_2)$

\item Multiplikation von 2-Tupeln im $\mathbb{R}^2$ (Dvision analog)

$ (x_1, y_1) \cdot (x_2, y_2) := (x_1 x_2 - y_1 y_2, x_1 y_2 + x_2 y_1)$
\end{enumerate}

\paragraph{Bemerkung}
\begin{enumerate}

\item Die reellen Zahlen $\mathbb{R}$ sind eingebettet in $\mathbb{C}$
\begin{align*}
a \in \mathbb{R} & \mapsto (a, 0) \in \mathbb{C}\\
\imag = \sqrt{-1} & \mapsto (0, 1)  \in \mathbb{C}
\end{align*}

\item Wir haben leider keine Anwendung für $<$ oder $>$ in $\mathbb{C}$\\
$\rightarrow$ Hilfskonstruktion

\end{enumerate}

\paragraph{Anwendungen komplexe Zahlen}
\begin{itemize}
\item D.Knuth: Mathematical Typography Bullitin of the AMS Nr. 1 (1979) S. 337- 372
\item Heute: Metafont / \TeX{}
\item Gegeben: Punkte in der Ebene ($\mathbb{R}^2$ bzw. $\mathbb{C}$)
\item Aufgabe: Kunstruiere zu den Punkten eine schön aussehende Kurve.
\item Idee: Ein Buchstabe aus vielen stückweise aneinandergesetzten Kurven. Kurvenstücke sind kubische Polynome mit komplexen Koeffizienten
\item Formal: Parameter $ t \in [0,1] $ \\
$$ z(t) = a_0 + a_1 t + a_2 t^2 + a_3 t^3 \text{ mit }  a_i \in \mathbb{C}$$
1 Buchstabe aus vielen (10-12) $z(t)$ Kurven. Verschiedene $z(t)$ möglichst \glqq rund\grqq
 aneinander setzen.
\end{itemize}

\paragraph{Achtung}
\begin{itemize}
\item bisher: $f: \underbrace{I}_{\subset \, \mathbb{R}} \longrightarrow \mathbb{R}$ \\
\glqq Funktion\grqq: $x$ aus $I$ bekommt eindeutig ein $y\in \mathbb{R}: y=f(x)$ zugewiesen.\\
$\rightarrow$ für Buchstaben zu eng.
\item neu: Kurve
\begin{align*}
C:I & \longrightarrow \mathbb{R}^n \\
t & \longmapsto \begin{pmatrix}x_1(t) \\ \vdots \\ x_n(t)\end{pmatrix}
\end{align*}
\item neu: Buchstaben
\begin{align*}
C:I & \longrightarrow \mathbb{R}^2 \\
C(t) & = \begin{pmatrix}x(t) \\ y(t)\end{pmatrix} \underset{\mathbb{C}}{\overset{\mathbb{R}}{=}} z(t)
\end{align*}
\end{itemize}



Offen in $\mathbb{C}$ sind  \underline{Anordnungsfragen}.\\
Abhilfe aus der Vektorrechnung: Entfernung vom Ursprung (Länge)

\begin{definition}
Betrag einer komplexen Zahl $z$

$$ |z| = \sqrt{a^2 + b^2} $$
Mit Hilfe von $\overline{z}$:
$$|z|^2 = z \cdot \overline{z} = a^2 + b^2$$

\begin{center} 
\begin{tikzpicture}[line cap=round,line join=round,>=triangle 45,x=1.0cm,y=1.0cm]
\draw[->,color=black] (-0.5,0) -- (2.8,0);
\foreach \x in {,1,2}
\draw[shift={(\x,0)},color=black] (0pt,2pt) -- (0pt,-2pt);
\draw[color=black] (2.28,0.08) node [anchor=south west] { Re};
\draw[->,color=black] (0,-0.5) -- (0,2.5);
\foreach \y in {,1,2}
\draw[shift={(0,\y)},color=black] (2pt,0pt) -- (-2pt,0pt);
\draw[color=black] (0.1,2.06) node [anchor=west] { Im};
\clip(-0.5,-0.5) rectangle (2.8,2.5);
\draw [dash pattern=on 1pt off 1pt] (0,1)-- (1,1);
\draw [dash pattern=on 1pt off 1pt] (1,0)-- (1,1);
\draw (0,0)-- (1,1);
\fill [color=xdxdff] (0,1) circle (1.5pt);
\draw[color=xdxdff] (-0.25,1.1) node {$b$};
\fill [color=qqqqff] (1,1) circle (1.5pt);
\draw[color=qqqqff] (1.72,1.28) node {z = a + ib};
\fill [color=xdxdff] (1,0) circle (1.5pt);
\draw[color=xdxdff] (1,-0.28) node {$a$};
\fill [color=uququq] (0,0) circle (1.5pt);
\end{tikzpicture}
\end{center}
\end{definition}

\paragraph{Rechenregeln für Beträge}
\begin{enumerate}

\item
\begin{align*}
Re(z) & = \frac{1}{2} (z+\overline{z}) \\
Im(z) & = \frac{1}{2\imag} (z - \overline{z})
\end{align*}

\item
\begin{align*}
|z| & \geq 0 \\
|z| & = 0 \text{, falls } z=0 \\
|w\cdot z| & = |w| \cdot |z|
\end{align*}

\item
\begin{align*}
|\overline{z}| & = |z| \\
|z - w| & = |w - z| \\
-|z| & = \begin{cases} Re(z) \\ Im(z) \end{cases} \leq |z|
\end{align*}

\item Dreiecksungleichung
\begin{align*}
|w + z| & \leq |w| + |z| \\
|w - z| & \geq |w| - |z|
\end{align*}
\end{enumerate}
\newpage
\paragraph{Arbeiten mit Beträgen}
\begin{enumerate}
\item Kreisscheibe (inkl. Rand) um $z_0$ mit Radius $r$

\begin{center}
\begin{tikzpicture}[line cap=round,line join=round,>=triangle 45,x=1.0cm,y=1.0cm]
\draw[->,color=black] (-0.5,0) -- (3.5,0);
\foreach \x in {,1,2,3}
\draw[shift={(\x,0)},color=black] (0pt,2pt) -- (0pt,-2pt);
\draw[color=black] (2.98,0.08) node [anchor=south west] { Re};
\draw[->,color=black] (0,-0.5) -- (0,3.5);
\foreach \y in {,1,2,3}
\draw[shift={(0,\y)},color=black] (2pt,0pt) -- (-2pt,0pt);
\draw[color=black] (0.1,3.06) node [anchor=west] { Im};
\clip(-0.5,-0.5) rectangle (3.5,3.5);
\draw [color=evtftf,fill=evtftf,fill opacity=0.1] (2,2) circle (1cm);
\draw [->] (2,2) -- (3,2);
\fill [color=qqqqff] (3,2) circle (1.5pt);
\fill [color=qqqqff] (2,2) circle (1.5pt);
\draw[color=qqqqff] (2.1,2.28) node {$z_0$};
\draw[color=black] (2.4,1.7) node {$r$};
\end{tikzpicture}
\end{center}
$$ K_r(z_0) = \lbrace z \in \mathbb{C} \mid |z-z_0| \leq r \rbrace $$
\item alle $z$ aus $\mathbb{C}$ gesucht, mit $|z+1| = |z-1|$
\begin{center}
\begin{tikzpicture}[line cap=round,line join=round,>=triangle 45,x=1.0cm,y=1.0cm]
\draw[->,color=black] (-1,0) -- (1.5,0);
\foreach \x in {-1,1}
\draw[shift={(\x,0)},color=black] (0pt,2pt) -- (0pt,-2pt);
\draw[color=black] (0.98,0.08) node [anchor=south west] { Re};
\draw[->,color=black] (0,-1) -- (0,1.5);
\foreach \y in {-1,1}
\draw[shift={(0,\y)},color=black] (2pt,0pt) -- (-2pt,0pt);
\draw[color=black] (0.1,1.06) node [anchor=west] { Im};
\clip(-1,-1) rectangle (1.5,1.5);
\draw [line width=2pt,color=qqffqc] (0,-1) -- (0,1.25);
\fill [color=qqqqff] (-1.72,6.3) circle (1.5pt);
\end{tikzpicture}
\end{center}
analytisch:
\begin{align*}
|z+1|^2 & = |z-1|^2 \\
(z+1)(\overline{z}+1) & = (z-1)(\overline{z}-1) \\
2(z+\overline{z}) & = 0 \\
Re(z) & = 0
\end{align*}
\end{enumerate}


\subsection{Polarform, Satz von Moivre}
\begin{center}
\begin{tikzpicture}[line cap=round,line join=round,>=triangle 45,x=1.0cm,y=1.0cm]
\draw[->,color=black] (-1.5,0) -- (2.5,0);
\foreach \x in {-1,1,2}
\draw[shift={(\x,0)},color=black] (0pt,2pt) -- (0pt,-2pt);
\draw[color=black] (1.98,0.08) node [anchor=south west] { Re};
\draw[->,color=black] (0,-0.5) -- (0,2.5);
\foreach \y in {,1,2}
\draw[shift={(0,\y)},color=black] (2pt,0pt) -- (-2pt,0pt);
\draw[color=black] (0.1,2.06) node [anchor=west] { Im};
\clip(-1.5,-0.5) rectangle (2.5,2.5);
\draw [shift={(0,0)},color=qqwuqq,fill=qqwuqq,fill opacity=0.1] (0,0) -- (0:1.4) arc (0:45:1.4) -- cycle;
\draw [->] (0,0) -- (1,1);
\fill [color=xdxdff] (0,0) circle (1.5pt);
\fill [color=qqqqff] (1,1) circle (1.5pt);
\draw[color=qqqqff] (1.14,1.28) node {$z$};
\draw[color=black] (0.4,0.75) node {$|z|$};
\draw[color=evtftf] (0,0)-- (0,1);
\draw[color=evtftf] (0,0)-- (1,0);
\draw[color=black] (-0.85,0.5) node {$|z| \sin \varphi$};
\draw[color=black] (0.7,-0.3) node {$|z| \cos \varphi$};
\fill [color=qqqqff] (-1.72,6.3) circle (1.5pt);
\draw[color=qqqqff] (-1.56,6.22) node {$C$};
\draw[color=qqwuqq] (0.92,0.38) node {$\varphi$};
\end{tikzpicture}
\end{center}
\begin{definition}
Polarform einer komplexen Zahl z
\begin{align*}
z & = |z| \cdot (\cos \varphi + \imag \cdot \sin \varphi) \\
a & = |z| \cdot \cos \varphi \\
b & = |z| \cdot \sin \varphi
\end{align*}
\end{definition}


Messung des Winkels $\varphi$
\begin{itemize}
\item $\varphi $ Gradmaß $0^\circ$ bis $360^\circ$
\item $\varphi$ \underline{Bogenmaß}, das ist die Länge des Kreisbogens mit Radius 1
\end{itemize}
Umrechnung:

\begin{align*}
360^\circ & \equals 2 \pi \\
\varphi \text{ Bogenmaß } & \equals \alpha \text{ Gradmaß} \\
\varphi & = \frac{2 \pi}{360} \cdot \alpha
\end{align*}

