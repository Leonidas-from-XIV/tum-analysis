\lecture{2010-01-20 Teil 1}

\subsection{Potenzreihen $ \sum_{k=0}^\infty \left( a_k x^k \right) $}
\label{sec:potenzreihen}

\begin{enumerate}

   \item Großer Umordnungssatz \\
   Ist die Reihe $ \sum_{k=0}^\infty \left( a_k \right) $ konvergent, so folgt für jede ungeordnete Reihe: $ \sum_{k=0}^\infty \left( a_k \right) $ konvergiert mit 
   \begin{equation*}
   \sum_{k=0}^\infty \left( a_{\sigma_k} \right) = \sum_{k=0}^\infty \left( a_k \right)
   \end{equation*}
   wobei $\sigma_k$ eine Permutation der Indizes $\{0,\ldots, k,\ldots, \infty\}$

   \item Produkte von Reihen \\
   Voraussetzung: Reihen $\sum a_k, \sum b_k$ seien absolut konvergent.\\
   Behauptung ($\sigma, \mu $ Permutationen):
   \begin{equation*}
      \sum_{k=0}^\infty \left( a_{\sigma_k} b_{\mu_k} \right) = \left(\sum_{k=0}^{\infty} a_k \right)\cdot \left(\sum_{k=0}^{\infty} b_k \right)
   \end{equation*}

   \item Cauchy-Produkt zweier Reihen \\
   Voraussetzung: Reihen $\sum a_k, \sum b_k $ seien absolut konvergent.\\
   \begin{align*}
      \left(\sum_{k=0}^\infty  a_k \right)\cdot\left(\sum_{k=0}^\infty  b_k\right) &= \sum_{n=0}^k \cdot \left(\sum_{k=0}^n  a_k b_{n-k} \right) \\
      &= \underbrace{(a_0 b_0)}_{n=0} + \underbrace{(a_0 b_1 + a_1 + b_0)}_{n=1} + \underbrace{(a_0 b_2 + a_1 b_1 + a_2 b_0)}_{n=2} + \ldots
   \end{align*}

\end{enumerate}

\begin{definition}[Potenzreihe]
   Eine Potenzreihe ist eine unendliche Reihe der Form
   \begin{equation*}
   \sum_{k=0}^\infty a_k x^k
   \end{equation*}
\end{definition} 

\subsubsection*{Fragen}

\begin{itemize}
   \item Für welche $x$ ist die Reihe konvergent?
   \item Welchen Wert nimmt die Reihe an?
   \item Auswertung \checkmark (Horner-Schema)
   \item \ref{sub:Taylorpolynome}, \ref{sub:Restglieddarstellung} Funktionswert als Potenzreihe.
\end{itemize}

\subsubsection*{Konvergenz-Frage}

\begin{itemize}

   \item $x = 0$ konvergent, aber uninteressant
   
   \item für welche $x$ haben wir absolute Konvergenz?\\
   Erwartung: \\
   \begin{tabular}{cc}
      in $\mathbb{R}$ & in $\mathbb{C}$ \\
      \begin{minipage}[t]{.5\textwidth}
         \begin{center}
            \begin{tikzpicture}[line join=round,>=triangle 45,x=1.2cm,y=1cm]
              \draw[->,color=black] (-2,0) -- (2,0);
              \foreach \x in {-1,0,1}
              \draw[shift={(\x,0)},color=black] (0,0.3) -- (0,-0.3);
                    
              \draw[ultra thick,color=orange] (-2,0) -- (-1,0);
              \draw[ultra thick,color=blue](-1,0) -- (1,0);
              \draw[ultra thick,color=orange] (1,0) -- (1.9,0);
              \draw[color=orange] (-1.5,0.6) node {Dvgz.};
              \draw[color=blue]   (0,   0.6) node {Kvgz.};
              \draw[color=orange] (1.5, 0.6) node {Dvgz.};
              \draw[<->] (-1,-0.5) -- (1,-0.5);
              \draw (0,-0.5) node[below] {R};
              
             \end{tikzpicture}
          \end{center}
     \end{minipage} &
     \begin{minipage}[t]{.4\textwidth}
      \begin{center}
       \begin{tikzpicture}[line cap=round,line join=round,>=triangle 45,x=1.0cm,y=1.0cm]
         \draw[->,color=black] (-1.5,0) -- (1.5,0);
         \draw[->,color=black] (0,-1.5) -- (0,1.5);
         \draw[color=black] (1.5,0.1) node [anchor=south west] { $\Re$};
         \draw[color=black] (0.1,1.5) node [anchor=west] { $\Im$};
         
         \draw[color=evtftf,fill=evtftf,fill opacity=0.1] (0,0) circle (1cm);
         \draw[->, color=blue] (0,0) -- (0.7,0.7);
         \draw[color=blue] (1,1) node {$R$};
         \draw[color=orange] (1.5,-1) node {Dvgz.};
       \end{tikzpicture}
       Konvergenzkreis mit Radius $R$ (Konvergenzradius)
     \end{center}
     \end{minipage}
  \end{tabular}
   \flush
   Antwort nach Cauchy-Hadamard (Wurzelkriterium \pageref{theorem:Wurzelkriterium}):\\
   \begin{equation*}
   \sqrt[n]{|a_k x^n|} \leadsto R = \frac{1}{\limsup_{n \to \infty} \sqrt[n]{a_n}}
   \end{equation*}
   Ergebnis: Potenzreihe ist konvergent, falls $|x| < R$

\end{itemize}

\begin{example}[Harmonische Reihe $\sum_{k=1}^\infty a_k = \sum_{k=1}^\infty \frac{1}{k}$ ist divergent!]\flush

   Potenzreihe $\sum_{k=1}^\infty a_k x^k = \sum_{k=1}^\infty \frac{x^k}{k}$
   \begin{align*}
      \begin{array}{ccrl}
         |x| & > & 1 & \text{Divergenz, da Minorante divergent.}\\
         x & = & 1 & \text{Divergenz, harmonische Reihe.}\\
         x & = & -1 & \text{Konvergenz, Leibnitz Reihe.}\\
         |x| & < & 1 & \text{Konvergenz } (\leadsto \ln(1+x) = \sum\left((-1)^\nu \cdot \frac{x^\nu}{\nu}\right))
      \end{array}
   \end{align*}
   \begin{center}
   \begin{tikzpicture}[line join=round,>=triangle 45,x=2cm,y=1cm]
     \draw[->,color=black] (-2,0) -- (2,0); % Linie+Pfeil des Zahlenstrahls
     \foreach \x in {-1,0,1}
     \draw[shift={(\x,0)},color=black] (0,0.3) -- (0,-0.3) node[below] {\footnotesize $\x$}; % Zahlen
           
     \draw [ultra thick,color=orange] (-2,0) -- (-1,0); % Markierung links
     \draw [[-,ultra thick,color=blue](-1,0) -- (1,0); % Markierung mitte
     \draw [[-,ultra thick,color=orange] (1,0) -- (1.9,0); % Markierung rechts
     \draw[color=orange] (-1.5,0.6) node {Dvgz.};
     \draw[color=blue]   (0,   0.6) node {Kvgz.};
     \draw[color=orange] (1.5, 0.6) node {Dvgz.};
     
    \end{tikzpicture}
  \end{center}
   
   
\end{example}

\subsection{Potenzreihen spezieller Funktionen}


Funktionsgleichung $f(x+y)=f(x) \cdot f(y)$ (*) \\
Suche Funktion $f$ mit Eigenschaft (*)\\
Normierung: $\underbrace{f(0) = 1, f'(0) = 1}_{\euler^x}$\\
Idee: Suche Lösung von (*) in Form einer Potenzreihe:\\
$f(x) = \sum_{k=0}^{\infty} a_k x^k \leadsto $ suche $a_k$, bestimme $R$ (Konvergenzradius)\\
Umsetzung:
\begin{align*}
   f(x+y) = \sum_{n=0}^\infty a_n \underbrace{(x+y)^n}_{\text{Binom.}} &= \underbrace{\left( \sum_{n=0}^\infty a_n x^n \right) \cdot \left( \sum_{n=0}^\infty a_n y^n \right)}_{\text{Cauchy-Produkt}} = f(x) \cdot f(y) \\
   \sum_{n=0}^\infty \left( a_n \cdot \sum_{k=0}^\infty {n \choose k} x^k y^{n-k}\right) &= \sum_{n=0}^\infty \sum_{k=0}^\infty a_k a_{n-k} x^k y^{n-k}
\end{align*}

Da die Potenzen $x^k, y^{n-k}$ passen (auf beiden Seiten), können die Koeffizienten abgeglichen werden:\\
per Induktion: $a_n = \frac{1}{n!}$\\
Mit Normierung $f(x) = \euler^x$\\
\begin{proof}[Induktionsbeweis]
   Normierung: $a_0 = \frac{1}{0!}$\, $a_1=\frac{1}{1!}$\\
   Induktionsschritt: $n-1 \to n$
   \begin{align*} % Sehr grausamer und unuebersichtlicher Code
      a_n \cdot \sum_{k=0}^n {n \choose k} x^k y^{n-k} &= \sum_{k=0}^n a_k a_{n-k} x^k y^{n-k}\\
      a_n \left(\underbrace{\color{red}{n \choose 0} y^n}_{k=0} + \underbrace{\color{blue}{n \choose n} x^n}_{k=n} \right) + a_n \sum_{k=1}^{n-1} {n \choose k} x^k y^{n-k} &= \underbrace{\color{red}a_n y^n}_{k=0} + \sum_{k=1}^{n-1}\underbrace{a_k a_{n-k}}_{\substack{Ind.Ann:\\ a_k = \frac{1}{k!}}} x^k y^{n-k} + \underbrace{\color{blue}a_n x^n}_{k=n}\\
      a_n \sum_{k=1}^{n-1} {n \choose k} x^k y^{n-k} &= \sum_{k=1}^{n-1} \frac{1}{k!} \frac{1}{(n-k)!} x^k y^{n-k}
   \end{align*}
   Forderung:
   \begin{align*}
      \forall x,y: \sum_{k-1}^{n-1} \left(\underbrace{a_n {n \choose k} - \frac{1}{k!} \frac{1}{(n-k)!}}_{\Rightarrow 0 \text{(Forderung)}} \right)x^k y^{n-k} &= 0 \\
      a_n {n \choose k} - \frac{1}{k!} \frac{1}{(n-k)!} &= 0\\
      \frac{a_n n!}{k! (n-k)!} - \frac{1}{k!} \frac{1}{(n-k)!} &= 0\\
      \Rightarrow a_n &= \frac{1}{n!}
   \end{align*}
\end{proof}
