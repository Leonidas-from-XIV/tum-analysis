\lecture{2010-01-20 Teil 1}

\emph{(\textannotation{Laut Dozent machte es der Übungsbetrieb erforderlich, die späteren Abschnitte \ref{sub:Taylorpolynome} und \ref{sub:Restglieddarstellung} vorzuziehen. Im Skript folgen wir allerdings der ursprünglichen Gliederung des Dozenten, so dass die chronologische Reihenfolge nicht übereinstimmt})}

\subsection{Potenzreihen $ \sum_{k=0}^\infty \left( a_k x^k \right) $}
\label{sec:potenzreihen}

\begin{theorem}[Großer Umordnungssatz]
   Ist die Reihe $ \sum_{k=0}^\infty \left( a_k \right) $ absolut konvergent, so ist auch jede umgeordnete Reihe $ \sum_{k=0}^\infty \left( a_{\sigma_k} \right) $ absolut konvergent mit 
   \begin{equation*}
    \sum_{k=0}^\infty \left( a_{\sigma_k} \right) = \sum_{k=0}^\infty \left( a_k \right),
   \end{equation*}
   wobei $\sigma_k$ eine beliebige Permutation der Indizes $\{0,\ldots, k,\ldots, \infty\}$ ist.
\end{theorem}
\begin{theorem}[Produkte von Reihen]
   Seien die Reihen $\sum a_k, \sum b_k$ absolut konvergent und $\sigma, \mu $ Permutationen. Dann gilt:
   \begin{equation*}
      \sum_{k=0}^\infty \left( a_{\sigma_k} b_{\mu_k} \right) = \left(\sum_{k=0}^{\infty} a_k \right)\cdot \left(\sum_{k=0}^{\infty} b_k \right)
   \end{equation*}
\end{theorem}
\begin{theorem}[Cauchy-Produkt zweier Reihen]
   Seien die Reihen $\sum a_k, \sum b_k $ absolut konvergent. Dann gilt:
   \begin{align*}
      \left(\sum_{k=0}^\infty  a_k \right)\cdot\left(\sum_{k=0}^\infty  b_k\right) &= \sum_{n=0}^\infty \left(\sum_{k=0}^n  a_k b_{n-k} \right) \\
      &= \underbrace{(a_0 b_0)}_{n=0} + \underbrace{(a_0 b_1 + a_1 + b_0)}_{n=1} + \underbrace{(a_0 b_2 + a_1 b_1 + a_2 b_0)}_{n=2} + \ldots
   \end{align*}
\end{theorem}

\begin{definition}[Potenzreihe]
   Eine Potenzreihe ist eine unendliche Reihe der Form
   \begin{equation*}
   \sum_{k=0}^\infty a_k x^k.
   \end{equation*}
\end{definition} 

\subsubsection*{Fragen}

\begin{itemize}
   \item Für welche $x$ ist die Reihe konvergent?
   \item Welchen Wert nimmt die Reihe an?
   \item Auswertung \checkmark (Horner-Schema)
   \item Funktionswert als Potenzreihe, siehe Abschnitte \ref{sub:Taylorpolynome} (Seite \pageref{sub:Taylorpolynome}) und \ref{sub:Restglieddarstellung} (Seite \pageref{sub:Restglieddarstellung})
\end{itemize}

\subsubsection*{Konvergenz-Frage}

\begin{itemize}

   \item $x = 0$ konvergent, aber uninteressant
   
   \item für welche $x$ gibt es absolute Konvergenz?\\
   Erwartung: \\
   \begin{tabular}{cc}
      in $\mathbb{R}$ & in $\mathbb{C}$ \\
      \begin{minipage}[t]{.5\textwidth}
         \begin{center}
            \begin{tikzpicture}[line join=round,>=triangle 45,x=1.2cm,y=1cm]
              \draw[->,color=black] (-2,0) -- (2,0);
              \foreach \x in {-1,0,1}
              \draw[shift={(\x,0)},color=black] (0,0.3) -- (0,-0.3);
                    
              \draw[ultra thick,color=orange] (-2,0) -- (-1,0);
              \draw[ultra thick,color=blue](-1,0) -- (1,0);
              \draw[ultra thick,color=orange] (1,0) -- (1.9,0);
              \draw[color=orange] (-1.5,0.6) node {Dvgz.};
              \draw[color=blue]   (0,   0.6) node {Kvgz.};
              \draw[color=orange] (1.5, 0.6) node {Dvgz.};
              \draw[<->] (-1,-0.5) -- (1,-0.5);
              \draw (0,-0.5) node[below] {R};
              
             \end{tikzpicture}
          \end{center}
     \end{minipage} &
     \begin{minipage}[t]{.4\textwidth}
      \begin{center}
       \begin{tikzpicture}[line cap=round,line join=round,>=triangle 45,x=1.0cm,y=1.0cm]
         \draw[->,color=black] (-1.5,0) -- (1.5,0);
         \draw[->,color=black] (0,-1.5) -- (0,1.5);
         \draw[color=black] (1.5,0.1) node [anchor=south west] { $\Re$};
         \draw[color=black] (0.1,1.5) node [anchor=west] { $\Im$};
         
         \draw[color=evtftf,fill=evtftf,fill opacity=0.1] (0,0) circle (1cm);
         \draw[->, color=blue] (0,0) -- (0.7,0.7);
         \draw[color=blue] (1,1) node {$R$};
         \draw[color=orange] (1.5,-1) node {Dvgz.};
       \end{tikzpicture}
       Konvergenzkreis mit Radius $R$ (Konvergenzradius)
     \end{center}
     \end{minipage}
  \end{tabular}
\end{itemize}

\noindent Antwort nach \emph{Cauchy-Hadamard} (Wurzelkriterium, siehe Satz \ref{theorem:Wurzelkriterium} auf Seite \pageref{theorem:Wurzelkriterium}):
\begin{equation*}
  \sqrt[n]{|a_k x^n|} \leadsto R = \frac{1}{\limsup_{n \to \infty} \sqrt[n]{a_n}}
\end{equation*}
Ergebnis: Potenzreihe ist konvergent, falls $|x| < R$

\begin{example}[Potenzreihe $\ds\sum_{k=1}^\infty a_k x^k = \sum_{k=1}^\infty \frac{x^k}{k}$]\flush
  % harmonische Reihe $\ds\sum_{k=1}^\infty a_k = \sum_{k=1}^\infty \frac{1}{k}$ ist divergent
  % hat meiner Meinung nach nichts mit Cauchy-Hadamard zu tun -- LH
   \[
      \begin{array}{ccrl}
         |x| & > & 1 & \text{Divergenz, da Minorante divergent}\\
         x & = & 1 & \text{Divergenz, harmonische Reihe}\\
         x & = & -1 & \text{Konvergenz, Leibniz-Kriterium}\\
         |x| & < & 1 & \text{Konvergenz\footnotemark} 
      \end{array}\footnotetext{\text{benötigt für Taylorreihe $\ln(1+x) = \sum\left((-1)^\nu \cdot \frac{x^\nu}{\nu}\right)$, siehe Abschnitt \ref{ex:logarithmus_taylor} (Seite \pageref{ex:logarithmus_taylor})}}
   \]
   \begin{center}
   \begin{tikzpicture}[line join=round,>=triangle 45,x=2cm,y=1cm]
     \draw[->,color=black] (-2,0) -- (2,0); % Linie+Pfeil des Zahlenstrahls
     \foreach \x in {-1,0,1}
     \draw[shift={(\x,0)},color=black] (0,0.3) -- (0,-0.3) node[below] {\footnotesize $\x$}; % Zahlen
           
     \draw [ultra thick,color=orange] (-2,0) -- (-1,0); % Markierung links
     \draw [[-,ultra thick,color=blue](-1,0) -- (1,0); % Markierung mitte
     \draw [[-,ultra thick,color=orange] (1,0) -- (1.9,0); % Markierung rechts
     \draw[color=orange] (-1.5,0.6) node {Dvgz.};
     \draw[color=blue]   (0,   0.6) node {Kvgz.};
     \draw[color=orange] (1.5, 0.6) node {Dvgz.};
     
    \end{tikzpicture}
  \end{center}
\end{example}

\subsection{Potenzreihen spezieller Funktionen}

Funktionalgleichung
\begin{equation}
  \tag{*}\label{eq:euler} f(x+y)=f(x) \cdot f(y)
\end{equation}
Suche Funktion $f$ mit Eigenschaft (\ref{eq:euler})\\
Normierung: $\underbrace{\underbrace{f(0) = 1}_{a^x}, f'(0) = 1}_{\euler^x}$\\
Idee: Suche Lösung von (\ref{eq:euler}) in Form einer Potenzreihe:\\
$\ds f(x) = \sum_{k=0}^{\infty} a_k x^k \leadsto $ suche $a_k$, bestimme $R$ (Konvergenzradius)\\
Umsetzung:
\begin{align*}
   f(x+y) = \sum_{n=0}^\infty a_n \underbrace{(x+y)^n}_{\text{Binom.}} &= \underbrace{\left( \sum_{n=0}^\infty a_n x^n \right) \cdot \left( \sum_{n=0}^\infty a_n y^n \right)}_{\text{Cauchy-Produkt}} = f(x) \cdot f(y) \\
   \sum_{n=0}^\infty \left( a_n \cdot \sum_{k=0}^\infty {n \choose k} x^k y^{n-k}\right) &= \sum_{n=0}^\infty \sum_{k=0}^\infty a_k a_{n-k} x^k y^{n-k}
\end{align*}
%
Da die Potenzen $x^k, y^{n-k}$ passen (auf beiden Seiten), können die Koeffizienten per Induktion abgeglichen werden: $a_n = \frac{1}{n!}$\\
Normierung $\leadsto f(x) = \euler^x$
\begin{proof}
  \begin{itemize}
    \item Induktionsanfang $\equals$ Normierung: $a_0 = \frac{1}{0!}$\, $a_1=\frac{1}{1!}$
    \item Induktionsschritt $n-1 \to n$\\
   Annahme: für $0 < m < n$ gilt $\ds a_m = \frac{1}{m!}$
   \begin{align*} % Sehr grausamer und unuebersichtlicher Code
      a_n \cdot \sum_{k=0}^n {n \choose k} x^k y^{n-k} &= \sum_{k=0}^n a_k a_{n-k} x^k y^{n-k}
      \intertext{Herausziehen der Summanden für $k=0$ und $k=n$}
      a_n \left(\underbrace{\color{red}{n \choose 0} y^n}_{k=0} + \underbrace{\color{blue}{n \choose n} x^n}_{k=n} \right) + a_n \sum_{k=1}^{n-1} {n \choose k} x^k y^{n-k} &= \underbrace{\color{red}a_n y^n}_{k=0} + \sum_{k=1}^{n-1}\underbrace{a_k a_{n-k}}_{\text{Ind.-Ann.}} x^k y^{n-k} + \underbrace{\color{blue}a_n x^n}_{k=n}\\
      a_n \sum_{k=1}^{n-1} {n \choose k} x^k y^{n-k} &= \sum_{k=1}^{n-1} \frac{1}{k!} \frac{1}{(n-k)!} x^k y^{n-k}
   \end{align*}
   Forderung:
   \begin{align*}
      \forall x,y:\quad \sum_{k-1}^{n-1} \left(\underbrace{a_n {n \choose k} - \frac{1}{k!} \frac{1}{(n-k)!}}_{= 0 \text{ (Forderung)}} \right)x^k y^{n-k} &= 0 \\
      a_n {n \choose k} - \frac{1}{k!} \frac{1}{(n-k)!} &= 0\\
      \frac{a_n n!}{k! (n-k)!} - \frac{1}{k!} \frac{1}{(n-k)!} &= 0\\
      \implies a_n &= \frac{1}{n!}
   \end{align*}
  \end{itemize}
\end{proof}
